 -----------------------------------------------------------------------------
% Xic Manual
% (C) Copyright 2009, Whiteley Research Inc., Sunnyvale CA
% $Id: extract.tex,v 1.93 2017/03/22 22:27:01 stevew Exp $
% -----------------------------------------------------------------------------

% -----------------------------------------------------------------------------
% xic:extmenu 090314
\chapter{The Extract Menu: Extraction and Verification}
\label{chpext}

\index{netlist extraction}
\index{extraction}
{\Xic} contains a facility for extracting a netlist from the physical
database, and comparing it with the schematic in the electrical
database.  {\Xic} can recognize devices in the physical layout,
extract geometric and electrical data from these devices, and
correspondingly update properties of electrical device instances.  The
netlists extracted from the physical and electrical databases can be
compared.  This layout vs.~schematic (LVS) testing is a useful means
of minimizing mask errors.

The {\cb Extract Menu} contains command buttons for performing
extraction and related functions.  The commands are summarized in the
table below, which provides the internal command name and a
brief description.

\hspace*{-1.5em}
\begin{tabular}{|l|l|l|l|} \hline
\multicolumn{4}{|c|}{\kb Extract Menu}\\ \hline
\kb Label & \kb Name & \kb Pop-up & \kb Function\\ \hline\hline
\et Setup & \vt excfg & \cb Extraction Setup & Set up and control
  extraction\\ \hline
\et Net Selections & \vt exsel & \cb Path Selection Control & Select groups,
 nodes, paths\\ \hline
\et Device Selections & \vt dvsel & \cb Show/Select Devices & Select
 and highlight devices\\ \hline
\et Source SPICE & \vt sourc & \cb Source SPICE File & Update from SPICE
 file\\ \hline
\et Source Physical & \vt exset & \cb Source Physical & Update electrical from
 physical\\ \hline
\et Dump Phys Netlist & \vt pnet & \cb Dump Phys Netlist & Save physical
 netlist\\ \hline
\et Dump Elec Netlist & \vt enet & \cb Dump Elec Netlist & Save electrical
 netlist\\ \hline
\et Dump LVS & \vt lvs & \cb Dump LVS & Save physical/electrical comparison\\
 \hline
\et Extract C & \vt exc & \cb Cap Extraction & Extract capacitance
 using Fast[er]Cap\\ \hline
\et Extract LR & \vt exlr & \cb LR Extraction & Extract L/R using FastHenry\\
 \hline
\end{tabular}

In addition to the commands available from the {\cb Extract Menu}, the
extraction system provides a number of prompt-line commands which
provide additional or supplemental capability.  These include the {\cb
!antenna} command for testing the antenna effect on wire nets
connected to MOS gates, and the {\cb !netext} command for batch
extraction of physical wire nets from a layout.


% -----------------------------------------------------------------------------
% ext:method 022816
\section{Extraction System: Methodology and Overview}
\index{extraction!methodology}
\index{group number}
To use the extraction capability, one typically first designs the
circuit in electrical mode, producing a schematic, which operates
correctly in simulation.  One then produces a corresponding layout in
physical mode.  Generally, the objective and requirement is that the
layout vs.  schematic test initiated with the {\cb Dump LVS} button in
the {\cb Extract Menu} will show no errors.  This usually requires
that the subcircuits, if any, individually pass LVS.  Thus, one would
build the cell hierarchy from the bottom up, enforcing LVS to pass at
each level.

The {\Xic} extraction system differs from others in that it does not
(at least presently) provide fixed associations between electrical and
physical devices, and subcells.  The user has the ability to place
cell contact terminals in the layout, and to create net name labels in
the layout and set schematic net names by various means, and matching
names will associate.  Many cells, though, will associate with no user
intervention.  If the association initially fails, generally a placed
cell terminal or two, or judicious use of net name labels, will allow
correct association and passing of LVS testing.

There are three important diagnostic and setup tools available from
the buttons at the top of the {\cb Extract Menu}.  The {\cb Setup}
button will bring up the {\cb Extraction Setup} panel.  This panel
contains four tabbed pages.  The {\cb Views and Operations} page
contains controls that make certain extraction-related features, such
as terminals and group numbers, visible.  It also allows terminal
placement and parameter editing.  It provides means to clear and run
extraction, which is otherwise automatic.  There is also provision to
select devices and other objects that fail to associate.  The other
three pages of this panel provide controls which map to variables and
flags which control extraction system behavior.  Most of this will be
set in the technology file, and it is unlikely that there will be a
frequent need to change the parameters interactively.  The {\cb Misc
Config} page does provide some display attribute choices which may be
an exception.

The {\cb Path Selection Control} panel is obtained from the {\cb Net
Selections} button in the {\cb Extract Menu}.  This allows visualizing
conductor groups, and the corresponding nets in the schematic if there
is an association.  One can click to select the groups/nets, or enter
a group number to highlight that group.  The highlighting can follow
the net as it descends through the cell hierarchy to any depth.  The
panel provides a very useful tool for diagnosing net connectivity
problems.

The {\cb Show/Select Devices} panel appears on pressing the {\cb
Device Selections} button in the {\cb Extract Menu}.  This will list
the devices found in the current physical layout.  These devices can
be highlighted by index number, or selected by clicking on them.  When
selected, measured parameters may be printed and/or compared with the
dual device in the schematic.  The panel is a useful tool for
addressing device recognition issues.

There are cases where one starts with a layout, and it is desirable to
generate a schematic.  There are also situations where the physical
and electrical designs are generated in separate files, and it is
desirable to merge these into a single file.  {\Xic} has provisions to
assist in these cases.

Schematics can be generated in various ways.  The schematics that are
machine-generated by {\Xic} have each device individually connected to
{\et gnd} or {\et tbar} terminals, so there are no wires.  These
schematics are electrically correct, but lack human-readability and
aesthetics.  They serve, however, as a starting point if the user
wishes to rearrange the devices and add wires as in a normal
schematic.

A schematic can be generated from a SPICE file with the {\cb Source
SPICE} button in the {\cb Extract menu}.  This can create devices and
subcircuits as needed.  Existing devices will have properties updated
with values from the SPICE file.

Similarly, the {\cb Source Physical} button will create or update the
schematic from an intermediate SPICE file extracted from the physical
layout.  Existing devices will have properties updated with values
extracted from the physical layout, and missing devices and
subcircuits are added.

The {\cb Import Control} panel from the {\cb Convert Menu} is used
to copy either the electrical or physical part of another cell into
the current cell.  It is able to extract this information from cell
definitions within an archive file.  This can be used to combine
separate electrical and physical designs into a single hierarchy.

Separate commands are available for generating netlist files from the
physical and electrical data.  The {\cb Dump LVS} command in the {\cb
Extract Menu} performs the layout vs.  schematic comparison, and
prints errors in a file which may be displayed on-screen.

The {\cb Dump Phys Netlist} command in the {\cb Extract Menu}
generates a connectivity listing extracted from the physical database. 
This includes a listing of extracted devices, in various formats.  One
format is SPICE, so that the {\cb Dump Phys Netlist} command can be
used to generate a SPICE listing extracted from the physical layout.

Commands in the {\cb Extract Menu} also work with the node mapping
facility for SPICE output.  It is often necessary to know the name of
specific circuit nodes in a SPICE file, which by default is not
possible as {\Xic} assigns then internally.  The node mapping
facility, controlled with the {\cb nodmp} button in the electrical
mode side menu, allows the node tokens to be preassigned.


% -----------------------------------------------------------------------------
% ext:errlog 110513
\section{Extraction System: Logging and Error Reporting}
\index{extraction logging}
\index{log files}
If, while running the grouping, extraction, or association operations,
the operation cannot complete due to an error, or if an important
error is identified such as problems in the technology file setup, a
file browser window containing the {\vt extraction.errs} file will
appear.  This file is created in the log files area, and will contain
messages indicating serious or fatal errors encountered during
processing.  When such an error occurs, the file will be made visible
automatically, so there is usually no reason to explicitly view this
file in the absence of any error indication.

Note that lack of successful association is not considered an error. 
It is up to the user to make sure that the electrical and physical
designs are consistent, and that terminals get placed correctly.

Logging of these operations can be enabled from the {\cb Logging
Options} panel from the {\cb Logging} button in the {\cb Help Menu}. 
The {\cb Grouping/Extraction/Association} group in the lower half of
the panel contains four check boxes:  {\cb Group}, {\cb Extract}, {\cb
Assoc}, and {\cb Verbose}.  Checking any or all of the first three
will enable logging of the checked operation, producing log files
named {\vt group.log}, {\vt extract.log}, and {\vt associate.log} in
the log files area.  If the {\cb Verbose} box is checked, the files
will contain additional information.  These (and all) log files are
accessible through the {\cb Log Files} button in the {\cb Help Menu}.

The log files may be helpful to the user or to Whiteley Research in
resolving problems.  The formats are not documented, but are intended
to be reasonably suggestive if not self-explanatory to an advanced
user.


% -----------------------------------------------------------------------------
% ext:operations 110513
\section{Extraction System: Operations and Algorithms}
\index{extraction algorithm}

There are three internal operations performed by the extraction
system:  grouping, extraction, and association.  These operations are
performed as needed, and have to be performed only once, unless the
cell is modified.  This accounts for the delay and activity noted in
response to initiating many of the command buttons and other
operations related to the {\cb Extract Menu}.  The technology file
specifies the information necessary to perform these operations.

The extraction subsystem requires that a number of items be set up
properly in the technology file.  This includes the setting of
keywords in layer blocks to identify layers that serve as conductors
or vias, and definition of device blocks which allow certain devices
to be recognized by their physical structure.  Wire nets and
subcircuit connectivity are determined automatically by assembling
groups of similar objects that touch or overlap, or are connected
through a via.  Once conductor groups are established, devices are
extracted, and device terminals are assigned a conductor group index. 
This results in a description of the circuit which can be compared
with the electrical schematic for consistency.

In the present version, all device extraction is performed
automatically, thus there is no need or provision for manual placement
of device terminals.  The connection terminals to the current cell are
created in the schematic with the {\cb subct} button in the side menu. 
It is possible to manually place the corresponding physical cell
terminals in the layout with the {\cb Edit Terminals} button in the
{\cb Views and Operations} page of the {\cb Extraction Setup} panel
from the {\cb Setup} button in the {\cb Extract menu}.  However this
is not always necessary, as {\Xic} will attempt to place the terminals
at a correct position in the layout automatically.  If this fails,
manually placing terminals will assist in the association process, and
may be needed in some cases to resolve ambiguity.

Many of the devices in the device library have physical terminal
extensions in their node data structures.  These are the ``physical''
devices, such as resistors, capacitors, and transistors.  Other
devices, such as voltage and current sources, are not physically
implementable and have the {\et nophys} property assigned.  These
devices have no physical terminal extensions, thus will not appear in
netlists generated from the physical layout.  There is a third class
of ``device'' in the device library, which includes the {\et gnd}
(ground) and {\et tbar} terminals.  These have no explicit physical
implementation, but are an implicit part of the wiring net as they
assign connections by name to locations in the schematic.  All points
connected to terminal devices with the same name are logically
connected together.  The terminal devices include multi-contact
(``bus'') terminals, which again pertain to the schematic only and
have no counterpart in the layout.

When a device is placed in the electrical schematic, a physical
terminal for each device connection is associated with the physical
cell.  In physical mode, these terminals are made visible with the
{\cb All Terminals} and {\cb Cell Terminals Only} check boxes in the
{\cb Show} group in the {\cb Views and Operations} page of the {\cb
Extraction Setup} panel.  Before association, these terminals are
grouped just outside of the lower left corner of the physical cell's
bounding box.  During association, these terminals are automatically
moved to their proper locations in the physical layout, if the
corresponding physical device structure is correctly identified and
associated.  One can select and investigate physical devices in the
layout with the {\cb Show/Select Devices} panel from the {\cb Device
Selections} button in the {\cb Extract Menu}.

Separate commands are available in the {\cb Extract Menu} for
generating netlist files from the physical ({\cb Dump Phys Netlist})
and electrical ({\cb Dump Elec netlist}) data.  The {\cb Dump LVS}
command performs the layout vs.  schematic comparison, and prints
differences in a file which can be displayed on-screen.

\index{extraction grouping}
\index{grouping operation}
\subsection{The Grouping Operation}

Initial geometric processing is performed, such as implementing the
{\vt Conductor Exclude} definitions.  The extraction system maintains
a shadow cell database.  The cells in the extraction database may
contain modified objects and flattened subcell geometry, different
from the cell in the main database.  The extraction database is
created before grouping.  If the {\cb Extraction View} check box in
the {\cb Show} group in the {\cb Views and Operations} page of the
{\cb Extraction Setup} panel from the {\cb Extract Menu} is checked,
the physical drawing windows will display the cell content based on
this database, and not the main database.

One complexity that arises is that a device such as an inductor or
transmission line is often implemented simply as a strip of conducting
material.  In order to insert the device, the grouping algorithm has
to be fooled into thinking that the strip which is the device is
actually two disconnected strips, one for each terminal.  This can be
accomplished with the introduction of special layers used in layout,
and the {\vt Conductor Exclude} directive.  The directive will
logically remove parts of the conductor that intersect with the
special layer.  Perhaps a more familiar example is a MOS transistor,
whose body is defined by POLY over ACTIVE.  Both are required to be
conductors, but without ``{\vt Conductor Exclude POLY}'' in the ACTIVE
layer definition block, the source and drain would be shorted
together!.

In grouping, conducting nets are identified.  Every conducting object
is given a group number.  The group number is the same for all of the
conducting objects in a wire net, and each disjoint wire net has a
unique group number.

Grouping is done recursively, starting from the leaf subcells and
working up to the current cell.  The core of the grouping is an
algorithm for determining conductor paths due to touching objects on a
layer, and through vias between layers if the {\vt Via} keyword has
been included in the technology file for the via layers, and by
contact layers if the {\vt Contact} keyword was applied in the
technology file.  Once a cell is processed, it is not regrouped unless
the cell is modified.  The {\cb Groups} check box in the {\cb Show}
group in the {\cb Views and Operations} page of the {\cb Extraction
Setup} panel can be checked to display the group numbers of objects in
the layout in the drawing windows.  Each group (conductor net) is
assigned a number.  While the {\cb Groups} check box is checked, these
numbers are printed on-screen near the conducting objects.

\index{extraction operation}
\subsection{The Extraction Operation}

Extraction is the identification of physical devices and subcircuits,
and establishment of the connections between them.  This is the most
compute-intensive and time consuming part of the process.  Extraction
is done recursively, starting from the leaf subcells and working up to
the current cell.  It requires that the grouping operation has been
performed and the group numbering is up to date.

Initially, subcells that should be flattened into the cell are
identified, and the flattening performed.  On the initial pass, cells
that are wire-only, i.e., contain no devices and only wire-only
subcells, and cells that have been explicitly specified as flattenable
by the user (see \ref{exthier}), will be flattened.  In flattening,
objects from the master cell are transformed and added to the
containing cell, replacing the cell instance.  This is done in the
shadow cell database used by the extraction system.  The extraction
may be repeated for a cell, if it is determined subsequently the
additional subcells require flattening.  This may not be known until
the association stage.  Suffice it to say that the process is
iterative and a bit more complicated than the simple progressive flow
implied in this description.

Vias and similar wiring cells should have no electrical terminals,
and should not be placed in the schematic.

The extraction operation will look for net labels and physical cell
terminals that have been placed by the user.  These, if found, supply
text names for the group over which they reside.  The names are saved
along with the object list and other parameters for each conductor
group, for later use.

In extraction, devices are recognized by the patterns specified in the
device definitions in the technology file.  The device contact points
are identified, and connecting conductor group numbers recorded. 
Connections to and between subcells are identified and recorded.

In each subcircuit instance, each conductor group is extracted,
transformed to parent cell coordinates, and compared with the parent
conductor groups and other subcircuit conductor groups for
connectivity.  Connectivity between conductor groups can be
established through
\begin{enumerate}
\item{similar {\vt Conductor} or ground plane layers touching.}
\item{{\vt Contact} and {\vt Conductor} or ground plane layers
touching.}
\item{an area of a {\vt Via} layer exists, at any level of the
hierarchy, under which the two via layers exist (one from two
different groups) and any conjunction expression is true.}
\end{enumerate}

Conductor groups are merged when necessary due to being connected
through subcircuits, flattened and not.  When two groups merge, the
object lists are merged, and the larger of the two group numbers is
replaced by the smaller.  When extraction is finished, the groups are
renumbered so that the numbering is compact.

\index{association operation}
\subsection{The Association Operation}

The association algorithm logically links devices, subcircuits, nets,
and terminals between the schematic and layout.  If association is
successful, then LVS will pass.

At its core, the association algorithm works by comparing candidate
similar objects from the schematic and layout, and computing a
numerical score.  The pair with the highest score ``wins'' and the two
objects become duals of one another, i.e., become ``associated''. 
This is done for devices, subcircuits and nets/groups.

Unfortunately, reality is not that simple, and the actual association
algorithm is quite complex.  Some of the factors contributing to the
complexity are listed below.

\begin{itemize}
\item{The hierarchy tree may not be quite the same in the schematic
and layout.  The association algorithm has provision for detecting
when necessary and logically flattening both the schematic and the
layout.}

\item{The circuit may have topological symmetry, where the comparison
test fails because the top score is shared by two or more pairs.  The
algorithm will try various ways to break the symmetry, and if that
fails, a random choice will be made.  Finding the correct permutation
in this type of case requires examination of the context of instances
of the cell in the hierarchy.}

\item{The layout may have split nets, where a logical net (as shown in
the schematic) consists of two or more disjoint conductor groups in
the layout.  Instances of the cell have the disjoint conductor groups
connected by metal from outside of the master cell.  The association
algorithm has provision for detecting and accommodating this case.}

\item{The layout and schematic may show different connections to
permutable subcell terminals, such as permutable inputs to a logic
gate.  The association algorithm attempts to detect this type of case
and avoid flagging it as an LVS error.}
\end{itemize}

Association is done in (at least) two passes.  The first pass is done
to the complete cell hierarchy from the bottom up, and corrections
that involve comparison at different levels of the hierarchy are
skipped.  Each cell is associated as far as possible, with no attempt
to break symmetries.

On the second pass, inter-hierarchy corrections are allowed (since the
parameters to be compared have now presumably been set), and
symmetry-breaking is allowed.

During the first pass, a list of symmetries is generated if
association fails to complete due to symmetry.  If this list is found
in the second pass, ``symmetry trials'' will be initiated.  Each
symmetry trial represents one choice, or permutation, of the
symmetries.  Association proceeds with consistency tests applied.  If
a consistency test fails at a later iteration, the present symmetry
trial is aborted, and all associations made in the trial are undone. 
A new symmetry trial, using a different permutation, begins. 
Eventually, unless limits are exceeded, the ``correct'' permutation
will be found and association will complete without errors.  The {\et
MaxAssocLoops} and {\et MaxAssocIters} variables set the limits.  The
{\et MaxAssocLoops} is approximately the maximum number of
permutations that can be accommodated.  The {\et MaxAssocIters}
variable sets the maximum number of calls to the comparison functions
until no further associations are found.  Only strange cases, such as
long series arrays of identical devices, require more than a few
iterations.

The {\cb Misc Config} page of the {\cb Extraction Setup} panel from
the {\cb Extract Menu} provides a number of controls affecting
association.


% -----------------------------------------------------------------------------
% ext:exthier 110513
\section{Extraction System: Cell Hierarchy and Flattening}
\index{extraction flattening}
\label{exthier}

When associating, {\Xic} will in most cases correctly account for
differences in the cell hierarchy in electrical and physical modes. 
This is most often automatic, though it is possible for the user to
intervene if necessary.  The following examples illustrate hierarchy
differences.

\begin{itemize}
\item{An instantiated device pcell, where the physical part is
represented by a subcell (a pcell sub-master).  This differs from
normal devices, which have an empty physical part.  In this case, the
physical subcell will need to be logically flattened into its
container cell.}

\item{The schematic might contain symbols such as logic gates, where
the gates are implemented with transistors in the physical counterpart
cell.  In this case, the gate schematic must be logically flattened
into the parent schematic.}

\item{Devices and subcircuits shown in the schematic might be found in
a subcell of the physical counterpart.  There may be a container cell
containing logic gates, for example, which itself has no schematic,
and the logic gates appear in the schematic.  In this case, the
container cell will need to be flattened into its container.}
\end{itemize}

Physical cells that are wire-only, i.e., contain no devices or
non-wire-only subcircuits, are always flattened into their container. 
Physical cells that are pcell instances will always be flattened. 
Physical and electrical cell instances will be flattened as needed
during extraction.  Thus, there is typically little need for the user
to set up explicit cell flattening, much less so than in earlier
{\Xic} releases.  However, the methods for explicit flattening are
still available and can be used if {\Xic} finds a flattening situation
that isn't handled properly automatically.

When a physical subcell instance is ``flattened'', all conducting
groups, devices and sub-subcells in the subcell master are transformed
and linked into the containing cell for extraction and LVS purposes. 
References to these cells will disappear from the {\cb Dump Phys
Netlist} listing, unless the boolean variable {\et PnetListAll} is
set, in which case they are listed.  Flattening of electrical cells is
similar.  The nets, devices, and subcircuits of the flattened instance
master are linked into the containing schematic.  The containing
schematic does not display this visually, this affects only the
internal data structures.

Logical flattening can be controlled by the user in two ways:  with
the {\et flatten} property, and with the {\et FlattenPrefix} variable. 
The variable can be set from the {\cb Cell flattening name keys} group
in the {\cb Net and Cell Config} page of the {\cb Extraction Setup}
panel, from the {\cb Setup} button in the {\cb Extract Menu}.

Of these two methods, use of the property is most efficient and
flexible, and the setting is inherently persistent as the property
value is saved in the layout file.  It is the only means by which the
flattening of individual instances can be controlled.

The {\et flatten} property can be applied to electrical and physical
cells and cell instances.  It can be applied to the current cell with
the {\cb Cell Property Editor} available in the {\cb Edit Menu}.  The
{\cb Add} menu in this panel provides a {\cb flatten} choice in both
electrical and physical modes.  When the {\et flatten} property is
applied to a master cell, instances of the cell will by default always
be flattened in the extraction system.  However, {\et flatten}
properties can be applied to cell instances as well, using the {\cb
Property Editor} from the {\cb Edit Menu}.  Again, this is true in
both electrical and physical modes.  If an instance is given a {\et
flatten} property, and its master also has a {\cb flatten} property,
the instance will {\bf not} be flattened.  If an instance has the
property and the master does not, then that instance, only, will be
flattened.  Thus, when applied to a cell instance, the {\et flatten}
property inverts the flattening status implied by the master, for that
instance.

In earlier {\Xic} releases, logical flattening was controlled with the
{\et FlattenPrefix} variable.  The variable can be set to a list of
pattern-matching tokens which match the names of cells to be
flattened.  Cell names can be matched by prefix, suffix, or verbatim. 
The user is required to set this variable before extraction, a step
that can be avoided by use of the {\et flatten} property instead. 
Unlike the {\et flatten} property, the {\et FlattenPrefix} variable
applies to physical (layout) cells only.  The {\cb Cell flattening
name keys} text entry area in the {\cb Net and Cell Config} page of
the {\cb Extracting Setup} panel from the {\cb Extract Menu} can be
used to set the {\et FlattenPrefix} variable.  The variable can also
be set from a startup file or the technology file.  See the
description of the variable for the property string syntax.


% -----------------------------------------------------------------------------
% ext:netname 061916
\section{Extraction System:  Group/Net Naming}
\index{extraction net names}
\index{extraction name labels}
\label{netname}

It is possible to apply a name to a physical wire net (or group) by
use of special labels.  The group name will be used in output when
appropriate.  It will also be used in association to match to
electrical nets which have the same name (see \ref{nodmp}).

Net name labels are created in the same way as any other label in
physical mode.  There are two requirements that must be met for the
label to be applied.

\begin{enumerate}
\item{The label origin mark must reside over or touch an object on a
layer with the {\vt Conductor} attribute.  This is generally applied
in the layer blocks of the technology file to metal or otherwise
conducting layers.}

\item{The layer name of the layer-purpose pair on which the label
resides must match that of the metal object above.  The purpose name
must match the special purpose name defined for this purpose within
{\Xic}.  By default, this purpose is named ``{\vt pin}''.}
\end{enumerate}

For example, suppose we have a box on a metal layer {\vt M1}, which
has the default ``{\vt drawing}'' purpose.  To name the conductor
group containing this box, we create a label, containing the name, on
the layer-purpose ``{\vt M1:pin}'', and place this so that the label
origin mark, which is marked with a tiny ghost-drawn cross during
placement, will be located in or touching the box.

The assumed ``{\vt pin}'' purpose name can be changed with the {\et
PinPurpose} variable, or equivalently with the {\cb Net label purpose
name} text input area of the {\cb Net and Cell Config} page of the
{\cb Extraction Setup} panel from the {\cb Extract menu}.

Alternatively, the {\et PinLayer} variable can be set to a layer name,
and all netname labels must reside on this layer.  In this case, the
pin purpose is not recognized.  This is for compatibility with older
libraries and is not recommended.

Physical groups can acquire a name in the following ways, listed
in priority order.
\begin{enumerate}
\item{An element of the group contains a net name label.  Only one
name is allowed, all net labels are ignored if conflicting net names
are found.  The net may have arbitrarily many name labels, but each
must specify the same name logically.  Names are by default
case-insensitive, and different subscripting delimiters are taken as
equivalent.  Net name labels override any other name that might be
associated with the group.}

\item{If an otherwise unnamed group contains a scalar cell terminal
after association, the group name will take the terminal name.  There
can be at most one terminal, if more than one the terminal names are
ignored.}

\item{After association, otherwise unnamed groups will take any
assigned name of the corresponding wire net.  Net names can be applied
in electrical mode with the {\cb Node (Net) Name Mapping} panel from
the {\cb nodmp} button in the electrical side menu.  Electrical nets
can also be named with wire labels, or with the named terminal devices
provided in the device library.}
\end{enumerate}

During association, electrical nets and physical conductor groups with
the same logical name will be associated without further testing.  At
this point, the only names that may apply to the conductor group are
from name labels, or from cell terminals that have been placed into
the layout by the user and therefor have the {\et FIXED} flag set. 
Thus, liberal but {\bf correct} use of net name labels can speed up
the association operation.

There is a provision for automatically generating or updating net name
labels after association, with the {\cb Update net name labels after
association} check box in the {\cb Net and Cell Config} page of the
{\cb Extraction Setup} panel, which is displayed with the {\cb Setup}
button in the {\cb Extract Menu}.  This tracks the state of the {\et
UpdateNetLabels} variable, which can be set or cleared directly to set
this mode.

This mode should be used only when layout of a cell is complete and
LVS passes.  Adding additional net name labels by forcing association
with this mode active before saving the cell to a library can maximize
association efficiency wen the cell is used in the future.

It is also possible to ignore net name labels entirely by use of the
{\cb Ignore net name labels} check box in the same panel, or
equivalently by setting the {\et IgnoreNetNameLabels} variable.  It is
unlikely that a user will require this with any frequency.

The pre-association net names, as obtained from net name labels and
user-placed cell terminals, can explicitly imply connectivity in LVS,
thus accounting for ``split nets''.  A split net is a logical
conductor group that consists of two or more physically disconnected
conductor groups.  For example, the cell schematic may show a simple
wire distributing power to all parts of a cell.  The physical
implementation, though, might consist of a power ring in the parent or
another cell, that runs over and makes contact at various points in
the cell.  Without the power ring, the cell contains multiple
locations that should be, but are not, connected together. 
Association and LVS will fail in this case, however there is a way to
detect and accept this type of case.

If each piece of a split net has a separate and of course logically
equivalent net name, then setting the {\cb Merge groups with matching
net names} check box in the {\cb Net and Cell Config} page of the {\cb
Extraction Setup} panel from the {\cb Extract Menu} will cause
association to logically merge these groups, allowing LVS to pass. 
Equivalently, the {\et MergeMatchingNamed} variable, which tracks the
check box, can be set or cleared to the same effect.

Although it is required that this mode be active for successful LVS
when the top-level cell contains split nets, it is not required
otherwise, even if cells lower in the hierarchy contain split nets.


% -----------------------------------------------------------------------------
% ext:gplane 110513
\section{Extraction System:  Ground Plane Handling}
\index{extraction ground plane}

The extraction algorithm can handle the situation where there is a
single ground plane layer, either clear or dark field.  Groups
connected to ground are always assigned to group number zero.  Group
zero is only used when a layer has been identified as a ground plane
through one of the keywords.

By default, handling of a {\et GroundPlane} (clear field) layer is the
same as for other {\et Conductor} layers, however, in the top-level
cell, the largest area group extracted on this layer is assigned to
group 0, the ground group.  There an alternative mode where all areas
of the layer, in any cell, are assigned to the ground group.

There are two levels of support for a dark-field ground plane,
indicated by the presence or absence of the {\et MultiNet} keyword
following ``{\vt GroundPlaneClear}''.  The simplest situation is where
the {\et MultiNet} keyword is absent.  In this case, terminals and
contacts with no connection, which would otherwise connect to the {\et
GroundPlaneClear} layer if that layer were present, are assigned to
group 0 (ground).

For example, suppose the technology file contained the following
lines:

\begin{quote}\vt
Layer M0\\
GroundPlaneClear\\
...\\
Layer I0\\
Via M1 M0
\end{quote}

In this case, an area of {\vt I0} over an area of {\vt M1} and {\it
not} over an area of {\vt M0} would indicate a connection of the {\vt
M1} area to ground.

To repeat, if the {\et MultiNet} keyword does not appear, then all
areas {\it outside} of the {\et GroundPlaneClear} layer geometry are
assumed to be above ground.  Vias and Contacts that have been
specified for the ground plane layer will make contact to ground in
the {\it absence} of the ground plane layer.

Although this sometimes works for simple cells, it can lead to
trouble.  Suppose that an island of ground plane metal is used as part
of the metalization for the chip pads.  This would appear as a hole in
the displayed representation of the ground plane layer.  Then each pad
will be extracted as shorted to ground!

If the {\et MultiNet} keyword is given following the {\et
GroundPlaneClear} keyword, then an internal layer, which is the
inverse polarity of the ground plane layer, will be created and used
for extraction purposes.  The algorithm used for inversion can be
specified by an integer 0--2 which optionally follows ``{\vt
MultiNet}''.  There are also {\cb !set} variables which parallel the
technology file keywords.  Complete information can be found in
\ref{extsetup}.


% -----------------------------------------------------------------------------
% ext:meascache 110513
\section{Extraction System: Measurement Caching}
\index{extraction measurement cache}
\label{meascache}

During association, and when a physical netlist is being created,
measurements may be performed on devices in the layout to extract
parameter values associated with the device.  This may, for example,
be the resistance of a resistor device, or geometrical factors
associated with a MOS transistor device.

The measurement may be rather compute intensive and time consuming,
thus {\Xic} supports a means for caching measurement results.  The
caching can radically reduce the time required to associate the
circuit, but it requires that the user intervene to update the cached
values if the underlying geometry changes.  This {\bf does not} happen
automatically.  Thus, measurement caching is disabled by default.  The
caching is enabled by setting the {\et UseMeasurePrpty} variable, or
equivalent checking the {\cb Use Measurement results cache property}
check box in the {\cb Device Config} page of the {\cb Extraction
Setup} panel from the {\cb Extract Menu}.

Every device can have a ``data box''.  This is created automatically
in {\Xic} on a layer-purpose pair named ``{\vt device:xicdata}''.  The
box coordinates are set to the extracted body bounding box of the
device.

The measurement results are saved to a {\et measures} property that is
applied to the data box.  This is property number 7106.  The property
string is set to a space-separated list of numbers, or colon-separated
pairs of numbers, representing measurement results, or non-permuted
and permuted measurement results if the device has permutable contacts
and the measurement result changes on permutation.  The ordering is
the same as the order of measurement requests in the device definition
block.

When a cell is read, by default the data box and {\et measures}
property, if present, are ignored.  If the {\et UseMeasurePrpty}
variable is set, and the {\et NoReadMeasurePrpty} variable is not set,
the values from the {\et measures} property will be used when
parameters are needed, and no values will be computed if the {\et
measures} property is found.  The {\et NoReadMeasurePrpty} variable
tracks the state of the {\cb Don't read measurement results from
property} check box in the {\cb Device Config} page of the {\cb
Extraction Setup} panel from the {\cb Extract Menu}.

After association, if the {\et UseMeasurePrpty} variable is set, the
data boxes and {\et measures} properties will be created if necessary,
and updated with the current measurement values.

Thus, to globally update the cached measurement values, one can use
the following procedure.

\begin{enumerate}
\item{Press the {\cb Clear Extraction} button in the {\cb Extraction
Setup} panel.  This will invalidate the present extraction state.}

\item{Make sure that the {\cb Use measurement results cache property}
check box in the {\cb Device Config} page of the {\cb Extraction
Setup} panel is checked.}

\item{Make sure that the {\cb Don't read measurement results from
property} check box on the same page is also checked.}

\item{Press the {\cb Do Extraction} button in the same panel.  This
will run grouping, extraction and association operations.  Measured
values will be computed, since any cached values are ignored due to
{\et NoReadMeasurePrpty} being set.  After association, the {\et
measures} properties are updated to the newly computed values.}

\item{Un-check the {\cb Don't read measurement results from property}
check box.  Save the cell hierarchy.}
\end{enumerate}


% -----------------------------------------------------------------------------
% ext:setup 021615
\section{Extraction System: Setup and Configuration}
\label{extsetup}

Use of the extraction features requires setting certain keywords and
data blocks in the technology file.  There are three types of entries: 

\begin{enumerate}
\item{Keyword descriptions in the various physical layer blocks,
including the extraction (see \ref{exkwords}) and physical property
keywords see \ref{phkwords}).  These define the layers as conductors
and insulators.  See the references for complete information.}

\item{Global attribute variables (see \ref{extech}).  There are a few
such variables for the extraction system, which provide such things as
the substrate dielectric constant.}

\item{The device blocks (see \ref{devblock}), which appear after the
layer blocks in the technology file.  These define device structures
to be recognized and extracted.}
\end{enumerate}

In addition, the user may wish to further customize the technology
file by adding scripts which perform some extraction-related function,
such as generating temporary layers.

If the user wishes to define a customized format for physical or
electrical netlist output, an entry in the format library file can be
added.  The format library contains scripts which provide formatting
for the commands in the {\cb Extract Menu} that produce netlists. 
Section \ref{fmtlib} provides more information on this capability.


% -----------------------------------------------------------------------------
% deviceblock 032117
\subsection{Device Blocks}
\label{devblock}
\index{device block}
Physical characteristics of devices which are candidates for
extraction are specified in device blocks in the technology file.  The
device blocks are located in the technology file after the physical
layer definitions.  These specifications enable automated extraction
of circuits from physical layouts.

Devices are specified in the technology file through a block of lines
keyed by the word ``{\vt Device}'' and ending with ``{\vt End}''.  An
example is below:

\begin{quote}\rr\vt
Device\\
Name res\\
Prefix R\_\\
Body R2\\
Contact + M2 I1B\&R2\\
Contact - M2 I1B\&R2 ...\\
Permute + -\\
Depth 1\\
Merge S\\
Measure Resistance Resistance\\
LVS Resistance\\
Spice \%n\% \%c\%+ \%c\%- \%ms3\%Resistance\\
Cmput Resistor \%e\%, resistance = \%ms3\%Resistance\\
Value \%m\%Resistance\\
End\\
\end{quote}

There can be no text following {\vt Device} in that line.  The block
must terminate with {\vt End}.

The device block in the example specifies a resistor device:
\begin{itemize}
\item{The resistor body consists of areas of layer {\vt R2}.}
\item{Contact is made to conductor {\vt M2} through a region of
  {\vt I1B} (which represents a physical via).}
\item{The resistor can have arbitrarily many contacts ({\vt ...}
  given in second {\vt Contact} line).  This will be decomposed
  into a network of two-terminal resistors by the extraction system.}
\item{The (two) terminals are interchangeable ({\vt Permute} given).}
\item{Parts of the resistor can be found in subcells ({\vt Depth} of 1).}
\item{Two-terminal resistors in series and in parallel will be merged
 iteratively into simplified networks.}
\item{The resistance of the structure will be measured and reported.}
\end{itemize}

The keywords are described in detail below.

\begin{description}
\item{\vt Template} {\it template\_name} {\it argument} ...\\
This will access the device template with the given {\it
template\_name}.  Text from the template will be inserted into the
current device block, after macro, variable, and argument
substitution.  For argument substitution, forms like ``{\vt \$(\_}{\it
N\/}{\vt )}'', with {\it N} being a positive integer, will be replaced
by the {\it N\/}'th argument, with any quote marks stripped.  If an
argument contains white space or other strange characters, it should
be double-quoted.

The template can provide all of the keyword text for the current
device block, or additional keywords from the list below can be
provided to supplement the template.  It may not be possible to
redefine an alredy defined keyword however.

\item{\et Name} {\it device\_name}\\
The {\it device\_name} names the device, which should match a device
cell name.  The name can be that of a parameterized cell, or a regular
device cell from the device library ({\vt device.lib}) file, or a
device cell from some other source.  This line is mandatory.

Two or more device blocks can use the same name if they have different
{\et Prefix} entries.  This might be useful, for example, if there are
two resistor layers in a process.  A device block would be needed to
describe resistors on each layer.

\item{\et Prefix} {\it prefix}\\
This is a prefix that is prepended when formulating the name for the
device used in output.  This is optional, as a prefix can also be
defined in the output formatting.  The first letter of the prefix
should match the expectations of the SPICE simulator or other tools to
be used.  If two or more device definition blocks have the same {\et
Name} field, they must have different {\et Prefix} fields.  Further,
each definition must have identical contact and bulk contact names,
and order, and identical permutable contact names.

\item{\et Body} {\it expression}\\
The {\et Body} keyword specifies the ``core'' physical feature of the
device.  The specification is a layer expression.  Each individual
region where the expression is true defines a potential instance of
the device.  This keyword is mandatory.

\item{\et Contact} {\it name} {\it layer} {\it expression} [{\vt ...}]\\
For each contact of the device, there should be a {\et Contact} line. 
The first token following {\et Contact} is a name for the contact,
which should match the corresponding name used in the node property of
the device in the device library file.  The second token is a layer
name of a conductor layer which is used to contact the device.  The
remainder is an expression which identifies the contact area.  The
contact is identified as a region inside the device bounding box where
the expression is true.  Multiple contacts using the same expression
can be given, and each will select a different region.  The device
bounding box is the bounding box of the body area, after the {\et
Bloat} operation (see below).

If the {\et Contact} line ends with ``{\vt ...}'' (three periods)
there can be more than one of that type of contact.  Ordinarily, there
is a one-to-one correspondence between contacts specified and contacts
in the device instance.  With the ellipses, device instances will
include as many of that type of contact as can be found.  Thus, such
devices no longer have a fixed number of contacts.  The ellipses can
{\it not} appear in the first {\et Contact} line, but may appear in
the second and/or subsequent {\et Contact} lines.

The ellipses feature presently supports multi-contact resistors.  A
multi-contact resistor is replaced internally by a network of
two-terminal resistors, which are used in the netlist output.  To
enable multi-contact resistor support, the second contact
specification in the resistor device block should end with ``{\vt
...}'', for example
\begin{quote}\vt
Contact + M2 I1B\&R2\\
Contact - M2 I1B\&R2 ...
\end{quote}

This specifies that as many of the second type of contact as can be
found will be extracted.  Without the ``{\vt ...}'' only two contacts
would be extracted.  The ellipses can not occur on the first contact
line, but may occur on other lines, and may occur more than once,
though no standard devices use this feature presently.  In general,
this implements device extraction with arbitrary numbers of certain
contacts.

Internally, a conductivity matrix is computed from the body and
contact geometry, and this is used to compute the effective values of
the two-terminal resistors that are used to implement the
multi-contact resistor.  Resistors that would have very high values
(larger than 100 times the smallest value) are not added, so that
linear multi-contact resistors decompose as one would expect.

The decomposition occurs before the serial/parallel merging, so that
the components of the decomposition are candidates for merging, if
merging is enabled (see the {\et Merge} keyword below).

\item{\vt BulkContact} {\it tname} {\it level} [{\it name} $|$
{\it bloat} {\it layername} {\it expression\/}]\\
This is a special form of a contact specification that applies to well
and substrate connections, which may be treated differently than other
contacts.  The {\it tname} is the contact name.  This is followed by
an integer {\it level} which determines how the contact is handled
during extraction.  Possible {\it level} values are as follows.  The
remaining entries in the line depend on the {\it level}.  There can be
at most one {\vt BulkContact} line in the device description.

\begin{description}
\item{0}\\
Check in cell during device extraction.  This requires that a {\it
bloat} value, {\it layername\/}, and {\it expression} follow the level
number.  The body bounding box is (logically) bloated by the {\it
bloat} value given (in microns).  Within the bloated area, a region
where the {\it expression} (a layer expression) is dark, that is
connected to the conductor {\it layername} must exist.  If there are
multiple areas, the one closest to the bounding box center is taken as
the contact area.  If there is no such area, the device will not be
recognized.  The search hierarchy depth for the contact is to all
levels.

\item{1}\\
Ignore this in extraction.  The level value must be followed by a {\it
name\/}, which is the name of a global net.  The contact will be
assumed to connect to that global net.  Use this mode only if it is
absolutely certain that physically the bulk contact is connected
properly, as this mode does no checking.  Use this at your own risk.

\item{2}\\
Check deferred.  This requires that a {\it bloat} value, {\it
layername\/}, and {\it expression} follow the level number.  During
extraction and association, if the contact can't be resolved within
its containing cell, {\it level\/}=2 contacts are ignored as for {\it
level\/}=1.

During LVS, a special ``stamping'' test is run over the complete
hierarchy, and any errors found are reported with the top-level cell. 
This test searches for unresloved {\it level\/}=2 contacts in the
hierarchy.  It will try and resolve the contact at the top level (thus
taking account of all geometry in the hierarchy).  If unsuccessful,
the device location as reflected to the top level coordinates will be
listed in the stamping report, and LVS will not succeed.
\end{description}

\item{\et Bloat} {\it increment}\\
This will expand the body bounding box by {\it increment} (in microns)
for the purpose of identifying contacts.  For example, a MOS
transistor body is the intersection of CAA and CPG.  The source and
drain contacts can be specified as the regions of the body bounding
box after a bloat that cover CAA but not CPG.

\item{\et ContactsOverlap}\\
Ordinarily, device contact areas can not overlap.  The extracted
contact areas are clipped against one another to enforce this.  Giving
this keyword allows overlap, which is necessary for some vertical
device structures.

\item{\et Permute} {\it name1 name2}\\
The names are the names of contacts that can be permuted to enable
association when comparing to a schematic.  This applies to devices
such as resistors, and to the source and drain of MOS devices, or to
any device containing a contact pair that are geometrically identical
to the extractor.  There must be exactly two names following
``Permute'', and only one {\et Permute} line is allowed.

\item{\et Depth} {\it depth}\\
The {\it depth} is the hierarchy depth extracted for the device,
default is 0, meaning all device structure should appear is the
current cell.  The value can be an integer, or `a' to look at the full
hierarchy.

\item{\et Find} [{\it device\_name}][.{\it prefix}]\\
This will cause a device with the given name and prefix to be searched
for in the current device's bounding area, and added to an internal
list.  Any number of {\et Find} directives can be applied.  If two or
more directives look for the same name/prefix, they will return
different instances.  Currently, this is used only for identifying
inductors in a mutual inductor device.

\item{\et Merge} [{\it arg\/}]\\
This optional keyword specifies how to handle parallel and series
connected instances of the device for parameter extraction.  There is
an optional argument.  Merging implies that multiple devices are
combined internally and reported as single devices in netlists and
SPICE output.  If both parallel and series merging are enabled, the
merging process is iterative, and will continue until no further
merging is possible.

If no {\et Merge} keyword appears in the device block, no merging is
done for that device.  Only the first two characters of the {\it arg}
are tested, case insensitively, and any remaining characters are
ignored.  Series merging will be enabled only for two-terminal devices
that have the {\et Permute} keyword applied, i.e., typically
resistors, capacitors, inductors.

\begin{tabular}{ll}
\kb arg & \kb Merge\\
no {\it arg} & parallel\\
\vt "s"      &      parallel and series\\
{\vt "ns"} or {\vt "sn"} & series\\
unrecognized & error\\
\end{tabular}

\index{NoMergeParallel variable}
\index{NoMergeSeries variable}
Merging can also be controlled by the variables {\et NoMergeParallel}
and {\et NoMergeSeries} which are booleans which can be set with the
{\cb !set} command.  The variables suppress merging of the indicated
kind, parallel or series, for all devices.

Merging can also be suppressed on an individual device basis by
applying a {\et NoMerge} property to an object that is used in the
body of the device.  This property can be added with the {\cb Property
Editor}.

Merging can lead to confusion, particularly when users are
experimenting.  Unless the aggregate has external connections, it is
likely to be merged down to a single device in ways which may be
surprising.

Example:\\
The {\cb Show computed parameters of selected device} option of the
{\cb Enable Select} command mode in the {\cb Show/Select Devices}
panel from the {\cb Device Selection} button in the {\cb Extract Menu}
is useful for displaying the values of extracted devices, and shows
the effect of merging.  When resistor networks are merged, {\Xic} will
merge series resistors if there are no other connections at the common
node.  Sometimes, this will lead to a configuration that is not
intended or desired, for example if the desired end terminal of the
network is connected to two resistors only, that node might be merged
away.  {\Xic} will merge devices arbitrarily if there is insufficient
information available to uniquely define how merging is to be done.
 
One way to prevent this from happening is to use temporary virtual
terminals:
 
\begin{enumerate}
\item{Switch to electrical mode.}
\item{Enter the {\cb subct} side menu command.}
\item{Press {\kb Ctrl} and click anywhere in the drawing window.  A
terminal marker will appear.  Dismiss the {\cb Terminal Edit} pop-up,
and switch back to physical mode.}
\item{Press the {\cb Setup} button in the {\cb Extract Menu} to obtain
the {\cb Extraction Setup} panel.  Press {\cb Edit Terminals} in the
panel.  A terminal mark should appear to the lower left of the
bounding box of the current cell.}
\item{Move the terminal mark to the desired network end terminal
metal.}
\end{enumerate}

This node now has a (phony) terminal, so it won't be merged.  
Don't forget to go back and delete the terminal when done.

The way the parameters are computed upon merging is determined by the
{\et Measure} keyword (see below).  Series merging is applicable to
resistor, capacitor, and inductor-type devices.

\begin{tabular}{lp{2.5in}l}
\kb Measure Keyword & \kb Parallel Action & \kb Series Action\\
\et BodyArea & Sum & Sum\\
\et BodyPerim & Sum & Sum\\
\et BodyMinDimen & Min$^1$ & Min\\
\et CArea & Sum$^2$ & -\\
\et CPerim & Sum$^2$ & -\\
\et CWidth & - & -\\
\et CNWidth & - & -\\
\et CBWidth & Average & Sum\\
\et CBNWidth & Sum & Average\\
\et Resistance & Parallel Resistance & Sum\\
\et Inductance & Parallel Resistance & Sum\\
\et Mutual\_Inductance & not implemented & not implemented\\
\et Capacitance & Sum & Parallel Resistance\\
\end{tabular}

\begin{quote}
{\bf Notes}:\\
1) Although the minimum of the multiple sections is used, for MOS
devices each value is typically the same.\\
2) If devices of the same type share a contact, the contact
area and perimeter are divided equally between the devices.\\
The fields with `-' above are invalid, and return 0 if accessed.
\end{quote}

The ``{\vt Merge M}'' feature of earlier releases is no longer
supported.  This would average the parameters of parallel-connected
MOS devices, and automatically add the ``{\vt M=}'' (multiplier)
parameter to the SPICE output line.  Now, the merging behavior is as
described above, and no multiplier is automatically added to the SPICE
line.  The {\et Sections} measurement keyword (below) can be used to
explicitly format the SPICE output to use the multiplier parameter, if
desired.

\item{\et Measure} {\it mname expression} [{\it precision\/}]\\
The {\et Measure} keyword allows geometrical information to be
extracted from the device, which is listed with the {\cb Dump Phys
Netlist} command in the {\cb Extract Menu} and used in other
commands in the {\cb Extract menu}.

\begin{description}
\item{\it mname}\\
A name for the parameter te be extracted.  This is arbitrary but
should be unique for the device.  This is the name by which the
particular measurement result is referenced.

\item{\it precision}\\
The optional {\it precision} is a non-negative integer which applies
to comparing electrical and physical values in layout vs.~schematic
(LVS) testing.  The default is 2 if not given.  If the value given is
{\it n}, then the two values must agree to a part in $10^n$, e.g., to
within 1 percent for the default value of 2.

\item{\it expression}\\
The {\it expression} consists of an expression in the format
recognized in scripts, where the variables are either the names from
previously defined {\et Measure} lines (in the current device block)
or the keywords below.  The expression is evaluated during extraction
yielding the result of the measurement.  The math functions are
available in the {\it expression} as are all of the math operators. 
There can be arbitrarily many {\et Measure} lines.

Below is a list of the ``primitive'' measurement tokens which can
appear in the measurement expressions.  In several cases, the token
consists of three fields, separated by `.'.  The additional fields
supply modifiers to the primitive measurement indicated by the token
name (the first field).

The basic unit of length is one micron.

\begin{description}
\item{\et Sections}\\
This returns the number of components of the device, which will be
greater than one if the device is an aggregate of several series or
parallel-connected devices ({\et Merge} enabled).

\item{\et BodyArea}\\
The area of the region where the {\et Body} expression is true.

\item{\et BodyPerim}\\
The perimeter length where the {\et Body} expression is true.

\item{\et BodyMinDimen}\\
This is a minimum dimension computed using the body geometry.  There
are several ways that this can be computed, depending on other
keywords and the device type.  The default algorithm first decomposes
the body shape into a trapezoid list.  the mid-height width and height
of each trapezoid is added to a histogram, weighted by the other
value.  For example, width=2, height=3 would add to the histogram 2
with weight 3, and 3 with weight 2.  When done, the value with the
largest weight will be taken as the {\et BodyMinDimen}.  If a tie, the
smaller value is used.  This is effective on structures where a ``line
width'' is an applicable concept.

However, if the {\et SimpleMinDimen} keyword is found, the {\et
BodyMinDimen} will instead be the smallest width or height found. 
This was the default algorithm in releases prior to 3.1.6.  The result
of the simple algorithm is less useful, as, for example for a
serpentine structure, it could be the line width, or the spacing.

There is yet another {\et BodyMinDimen} algorithm, associated with MOS
devices and indicated by the {\et ContactMinDimen} keyword.  With this
keyword, and if a {\et Permutes} contact list is given, the {\et
BodyMinWidth} will be the distance between the inside edges of the two
contacts in the {\et Permutes} list.  This is the default for
recognized MOS devices, i.e., devices whose prefixes start with `m' or
`M', and overrides {\et SimpleMinDimen} if both are applicable.

For MOS devices, where it is assumed that the gate length (source to
drain) is constant, i.e., the gate is a strip that can meander
arbitrarily, even forming a loop, for device size measurements, one
can specify
\begin{quote}
{\it length} = {\et BodyMinDimen}\\
{\it width} = {\et BodyArea}/{\et BodyMinDimen}
\end{quote}
\end{description}

The next four keywords have two optional trailing fields.  The value
in each field must be a single digit.  The digit corresponds to a {\et
Contact}, in order of appearance in the device block, starting with 0. 
If one or both fields is left off, the effective entry is 0.  If both
contacts are given the same digit, the second one is incremented. 
Thus, leaving off the trailing fields is equivalent to ".0.1".  If the
indices don't point to an existing contact, or are not single digits,
the measurement will fail.  These are illustrated in Figure
\ref{devmeas}.

The ``body bounding box'' is the rectangular region encompassing
the {\et Body} objects, before any bloat.

\begin{description}
\item{\et CWidth}[.{\it n1}.{\it n2\/}]\\
The width of the first contact, along a line connecting the first
contact with the second contact.

\item{\et CNWidth}[.{\it n1}.{\it n2\/}]\\
The width of the first contact, normal to the line connecting the
first contact to the second contact, measured at the contact
bounding box midpoint.

\item{\et CBWidth}[.{\it n1}.{\it n2\/}]\\
The width between the first contact and the second contact, which
lies over the body bounding box.

\item{\et CBNWidth}[.{\it n1}.{\it n2\/}]\\
The length of the line normal to the line between the first contact and
the second contact, over the body bounding box, passing through the
center of the body bounding box.
\end{description}

Example:

\begin{quote}\rr\vt
Contact s CAA CAA \& !CPG\\
Contact d CAA CAA \& !CPG\\
Contact g CPG CAA \& CPG\\
Measure Length CBWidth.0.1 * 1e-6\\
Measure Width CBNWidth.0.1 * 1e-6\\
\end{quote}

Note that the conversion to meters is included in the {\et Measure}
lines in the example above.

The following two keywords contain two trailing fields, which are
mandatory.  The first field contains a contact index as above.  The
second field contains the name of a layer.

\begin{description}
\item{\et CArea}.{\it n1}.{\it lname}\\
Construct a single polygon from the connected objects on the named
layer, one of which intersects the bounding box of the given contact. 
The area of the polygon is returned.  Note that the constructed
polygon can extend outside of the device's bounding box.  If the
device being measured is merged, then the result is the sum of the
results from each component.

\item{\et CPerim}.{\it n1}.{\it lname}\\
Measure the perimeter length of the polygon constructed as above.  If
the device being measured is merged, then the result is the sum of the
results from each component.
\end{description}

When measuring with CArea/CPerim (for MOS ad/as/pd/ps), there is a
test to see whether the area intersects other device contacts from the
same device type.  If a contact is shared between two devices, e.g.,
common active layer for two series-connected MOS devices, the
following algorithm is invoked.

For the shared contact:
\begin{enumerate}
\item{Compute the total contact area and perimeter for both devices.}
\item{Compute the area and perimeter for the contact area common to
both devices.}
\item{Subtract 1/2 this value from the parameters computed in the
first step.}
\end{enumerate}

This algorithm should work whether or not the devices are
multi-component and merging is enabled.

Example:

\begin{quote}\rr\vt
Contact s CAA CAA \& !CPG\\
Contact d CAA CAA \& !CPG\\
Contact g CPG CAA \& CPG\\
Measure AS CArea.0.CAA\\
Measure AD CArea.1.CAA\\
Measure PS CPerim.0.CAA\\
Measure PD CPerim.1.CAA\\
\end{quote}

\begin{description}
\item{\et Resistance}\\
Extract the resistance value (see below).

\item{\et Capacitance}\\
Extract the capacitance value.  The returned capacitance value is
given by the BodyArea times the capacitance per unit area, plus the
BodyPerim times the capacitance per unit length.  The capacitance
values are specified in a {\et Capacitance} line in the layer block of
one of the layers defining the device body, i.e., the layers mentioned
in the {\et Body} line.  If no body layer contains a {\et
Capacitance} specification, or if both parameters are zero, an error
results.

\item{\et Inductance}\\
Extract the inductance value (see below).

\item{\et Mutual\_Inductance}\\
Extract the mutual inductance value.  This is not yet implemented.
\end{description}
\end{description}

\end{description}
The internal device definition structure contains a flag that if set
causes the device to be treated as a MOS transistor.  There are a few
MOS-specific tests and operations found in the extraction system,
which are enabled by the flag.  By default, the flag is set if the
device {\vt Prefix} starts with `m' or `M'.

There are also flags that are set if the device is determined to be
n-type or p-type.  Presently, we only set these for MOS devices.  By
default, if the device name begins with `p' or `P', the device is
assumed to be p-type, otherwise it is taken as n-type. 

\begin{description}
\item{\et NotMOS}\\
If given, the flag that indicates that the device is a MOS transistor
will not be set, as it would normally be if the {\vt Prefix} starts
with `m' or `M'.

\item{\et MOS}\\
If this keyword is given, the flag indicating that the device is a MOS
transistor will be set.  This overrides {\et NotMOS}.

\item{\et NMOS}\\
Flags will be set to indicate that the device is an n-type MOS
transistor.

\item{\et PMOS}\\
Flags will be set to indicate that the device is a p-type MOS
transistor.

\item{\et Ntype}\\
Flags will be set to indicate that the device is n-type.  This is
meaningful only for MOS transistors at present.

\item{\et Ptype}\\
Flags will be set to indicate that the device is p-type.  This is
meaningful only for MOS transistors at present.

\item{\et SimpleMinDimen}\\
When given, and the {\et ContactMinDimen} is not applied or not
applicable, the {\et BodyMinDimen} measurement will be the smallest
trapezoid width or height found in the decomposition of the body
shape.  This was the default algorithm in releases prior to 3.1.6, but
a better algorithm is the new default.

\item{{\et ContactMinDimen} [{\vt y/n}]}\\
This keyword has a dual purpose:  to impose a MOS-like {\et
BodyMinDimen} computation on other device types, and to turn off the
use of this algorithm in MOS devices, which use this algorithm by
default.

Recognized MOS devices are devices that have the internal flag set as
mentioned above.  The device must also have a {\vt Permutes} list for
this algorithm to apply.  Most MOS devices have permutable source and
drain contacts.

In recognized MOS devices with permutes, the default {\et
BodyMinDimen} calculation is to set this to the distance between the
inside edges of the two contacts listed in the {\et Permutes} list. 
Thus, the {\et BodyMinDimen} will always be the device length
(source/drain spacing) even if the source-drain spacing is larger than
the device width.  The simple ``line width'' algorithm normally
applied for the {\et BodyMinDimen} would be ambiguous as to whether
the {\et BodyMinDimen} is the device length or width.

If {\et ContactMinDimen n} is given for a recognized MOS device, the
line width algorithm will be used.  The ``{\vt n}'' can actually be
one of many tokens that indicate negativity, such as ``{\vt no}'',
``{\vt false}'', ``{\vt off}'', ``{\vt 0}'', etc., case insensitive,
but the token must appear.

For devices that are not recognized MOS devices, the line width {\et
BodyMinDimen} algorithm is used by default.  However, if {\et
ContactMinDimen y} is given, and the device has a {\et Permutes} list,
the {\et BodyMinDimen} will be computed as for MOS devices.  The
``{\vt y}'' can actually be missing, or can be one of many possible
tokens that indicate truth, such as ``{\vt yes}'', ``{\vt true}'',
``{\vt on}'', ``{\vt 1}'', etc., case insensitive.

\item{\et LVS} {\it measure\_expr} [{\it spice\_name}]\\
This keyword instructs {\Xic} to perform a parameter comparison as
part of LVS.  The {\it measure\_expr} is either one of the names used
for a {\et Measure} statement in the device block, or a single-quoted
expression involving constants and names from {\et Measure}
statements.  The {\it spice\_name}, if given, is the token used in
SPICE element lines to designate the parameter, e.g., ``l'', ``w'',
``area''.  This can be blank if comparing to an element value which is
given as a leading number, i.e., resistance, capacitance, etc.  The
{\et LVS} directives must appear after the referenced {\et Measure}
line.

Examples:

\begin{quote}\rr\vt
Measure Area BodyArea*1e-12\\
LVS Area area\\
\end{quote}

\begin{quote}\rr\vt
Measure Resistance Resistance\\
LVS Resistance\\
\end{quote}

Any number of {\et LVS} lines can appear in a device block.

\begin{figure}
\caption{\label{devmeas} The distances returned by the various width
measurement keywords to the device block {\et Measure} line.}
\vspace{1.5ex}
\begin{center}
\epsfbox{images/devmeas.eps}
\end{center}
\end{figure}

\item{\et Spice} {\it specification\_args}\\
The {\et Spice} keyword specifies the format for the SPICE output part
of the listing from the {\cb Dump Phys Netlist} command in the {\cb
Extract Menu}.  The specification is copied verbatim, except for the
following substitutions:

\begin{description}
\item{\et {$\backslash$}n}\\
The character sequence `{$\backslash$}n' is replaced by a newline in the
expanded text.  Note that the next token should probably begin with
the SPICE continuation character `+' for the SPICE output to be
interpreted correctly.

\item{\et {$\backslash$}t}\\
The character sequence `{$\backslash$}t' is replaced by a tab character.

\item{\et \%c\%}{\it cname}\\
The {\it cname} is a contact name (from a {\et Contact} line).  This
token is replaced with the group number of the contact.

\item{\et \%m[g {\vt |} s[{\it N\/}] {\vt |} f[{\it N\/}] {\vt |}
     e[{\it N\/}]]\%}{\it mname}\\
The {\it mname} is a name from a {\et Measure} line.  The
token is replaced with the result of that measurement.
One of the characters s, g, f, e can follow the `m'.  The s, f, e
can be followed by an optional digit.  These select the format of
the printed result.

\begin{description}
\item{g}\\
Use the ``best'' numeric format (the default if no modifier given).

\item{s}{\it N}\\
Use SPICE abbreviations, with {\it N} decimal places.

\item{f}{\it N}\\
Use fixed point notation, with {\it N} decimal places.

\item{e}{\it N}\\
Use exponential notation, with {\it N} decimal places.

\end{description}
If {\it N} is not given, the default is 5 digits.

\end{description}
Above, {\it cname} and {\it mname} can be followed directly by
`\%' and other text, for a concatenation function.  For example
``{\vt L=\%m\%Length\%u}'' might be replaced with ``{\vt L=.8u}''.

\begin{description}
\item{\et \%n\%}\\
This token is replaced with a name for the device, which consists of
the {\et Prefix} (if given) followed by an index count for the device
type.

\item{\et \%p {\it lname} {\it pnum}\%}\\
This token is replaced by the text of a physical property.  The {\it
lname} is the name of a layer, and space after the `p' is optional. 
The {\it pnum} is a non-negative integer.  Each of the objects on {\it
lname} that intersect the device bounding box is checked for a
property with number {\it pnum\/}.  The string of the first such
property found is used.  This enables property text to appear in
device output, in particular it provides a means to coerce a value or
other parameter.

\item{\et \%e\%}\\
If the electrical dual of the physical device is known, the {\et
\%e\%} is replaced by the name of the electrical device.  If no dual
is known, the behavior is the same as {\et \%n\%}.

\item{\et \%f\%}\\
The substitution {\et \%f\%} is equivalent to {\et \%e\%} except that
if the dual device is unknown, the token is simply ignored.
\end{description}

Each of the substitution tokens can take an optional integer after the
first \%, which indicates that the token refers to the device in the
n'th {\et Find} line (0 is the same as no integer).

Example:

\begin{quote}\rr\vt
Device\\
Name mut\\
Prefix K\\
...\\
Find ind\\
Find ind\\
Spice \%n\% \%1n\% \%2n\% ...\\
\end{quote}

The {\et Spice} line prints the name of the mut device, followed by
the names of the two inductors.

\begin{description}
\item{\et \%model\%}\\
Replaced by contents of the {\et Model} line (see below).

\item{\et \%value\%}\\
Replaced by contents of the {\et Value} line (see below).

\item{{\et \%param\%} or {\et \%initc\%}}\\
Replaced by contents of the {\et Param} line (see below).
\end{description}

The {\et Spice} line is used in {\cb Dump Phys Netlist} output and
internally by {\cb Source Physical} command.

\item{\et Cmput} {\it specification\_args}\\
This specifies the format used in printing the device parameters from
the {\cb Show computed parameters of selected device} option of the
{\cb Enable Select} command mode in the {\cb Show/Select Devices}
panel The substitutions are exactly as those of the {\et Spice}
keyword.  For example:
\begin{quote}\rr\vt
Device\\
Name res\\
...\\
Measure Resistance Resistance\\
Cmput Resistance = \%m\%Resistance ohms\\
End\\
\end{quote}

\item\parbox[b]{4in}{\et
Model {\it specification\_args}\\
Value {\it specification\_args}\\
Param {\it specification\_args}
}\\
These keywords specify a format string to use when creating ``property
strings'' from the extracted parameters of a physical device, to be
used for comparison or updating the properties of the corresponding
electrical device.  These are used in the {\cb Source Physical}
command, and in the {\cb Show elec/phys comparison of selected
devices} option of the {\cb Enable Select} mode in the {\cb
Show/Select Devices} panel.  This panel is brought up by the {\cb
Device Selections} button in the {\cb Extract Menu}.

The format is the same as is described for the {\et Spice} line,
however the escapes {\vt \%model\%}, {\vt \%value\%}, and {\vt
\%param\%} are not recognized. 

The {\et Model}, {\et Value}, and {\et Param} lines are used
internally when comparing physical devices to their electrical
counterparts.  This is done, for example, in the {\cb Show elec/phys
comparison of selected devices} option of the {\cb Enable Select} mode
in the {\cb Show/Select Devices} panel.  This panel is brought up by
the {\cb Device Selections} button in the {\cb Extract Menu}.

\end{description}

The {\et Set {\it varname} = {\it something}} construct in the
technology file can apply to lines in device blocks, however the
{\et Set} keyword must appear outside and before the block.  Device
block lines can contain {\vt \$(}{\it varname}{\vt )} tokens, which
are replaced with {\it something} before the line is parsed.

The format specifications for the {\et Spice}, {\et Cmput}, etc. 
lines can contain the {\et eval(...)} construct.  The argument to {\et
eval} is evaluated as a mathematical expression, and the result
replaces the entire construct.  Unlike elsewhere in the technology
file, in these lines this construct is evaluated when the line is
used, and not when the technology file is read.

The extraction mechanism can be tested with the {\cb !find} command,
or with the device listing capability in the {\cb Show/Select Devices}
panel from the {\cb Device Selections} button in the {\cb Extract
Menu}.

{\Xic} contains functionality for accurately calculating resistor
values of arbitrarily shaped resistors.  Resistance extraction is
accomplished by dividing the resistor logically into a regular grid. 
The center of each grid is a ``node'' that is connected by resistance
to adjacent nodes.  Thus, the problem becomes one of solving a large
lumped resistor mesh.

Best accuracy is obtained when the grid falls on all the resistor and
contact boundaries.  It is not possible to find such a grid in
general, however if a layout grid is used and all corners are on-grid,
and all edges are Manhattan, then tiling will be possible.  It may be
the case that tiling is possible, but the tile is so small that the
computation time is unacceptable.

For structures that can't be tiled efficiently, a set of
edge-dependent heuristics is used to modify the matrix elements to
account for the local area deficit or surplus.

There are four variables that can be used to configure the extractor. 
The default values lean toward speed over accuracy.  By default,
tiling is not attempted, and the grid spacing will be selected so that
each resistor contains 1000 grid cells.

\begin{description}
\index{RLSolverDelta variable}
\item{\et RLSolverDelta}\\
{\bf Value:} floating point $>=$ 0.01.\\
It this value is set, the resistance/inductance extractor will assume
this grid spacing, in microns.  The number of grid cells enclosed in
the device will increase for physically larger devices, so that larger
devices will take longer to extract.  If this variable is set, the
other {\et RLSolver} variables are ignored.  Setting this variable may
be appropriate if all resistors are ``small'' and dimensions conform
to a layout grid.

\index{RLSolverTryTile variable}
\item{\et RLSolverTryTile}\\
{\bf Value:} boolean.\\
If set, the extractor will attempt to use a grid that will fall on
every edge of the device body and contacts.  The device and contact
areas must be Manhattan for this to work.  If such a grid can be
found, and the number of grid cells is a reasonable number, this will
give the most accurate result.

\index{RLSolverGridPoints variable}
\item{\et RLSolverGridPoints}\\
{\bf Value:} integer 10--100000.\\
When not tiling ({\et RLSolverTryTile} is not set), this sets the
number of grid points used for resistance/inductance extraction.  This
number will be the same for all device structures, so that computation
time per device is nearly constant.  Higher numbers give better
accuracy but take longer.  The value used if not set is 1000.

\index{RLSolverMaxPoints variable}
\item{\et RLSolverMaxPoints}\\
{\bf Value:} integer 1000--100000.\\
When tiling ({\et RLSolverTryTile} is set), the maximum number of grid
cells is limited to this value.  If the tile is too small, it will be
increased in size to keep the count below this value, in which case
the tiling will not have succeeded so there may be a small loss of
accuracy.  Using a large number of grid points can take a long time. 
The value used if not set is 50,000.
\end{description}

The resistor solver is accessed through the device block {\et Measure}
keyword ``{\et Resistance}'', for example:

\vspace*{.5cm}

\begin{quote}\rr\vt
Device\\
Name res\\
 ...\\
Measure value Resistance\\
End\\
\end{quote}

By including the ``{\vt Measure value Resistance}'', all resistances
may be extracted and the values will appear in the output of the {\cb
Dump Phys Netlist} command.  When computing the resistance, the layers
in the {\et Body} specification are checked for an {\et Rsh}
specification, or alternatively a {\et Rho} or {\et Sigma}
specification along with a {\et Thickness} specification.  If the
resistance parameters are not found, an error results.  Unlike
releases prior to 3.1.6, there is no default resistance.

{\Xic} also contains functionality to measure conductor inductance
values.  Inductance is extracted using an algorithm similar to
resistance, i.e., square counting, but other factors are included to
enhance accuracy.  This assumes ``microstripline'' geometry, meaning a
conductor separated from a ground plane by a uniform dielectric.  The
{\et Measure} keyword is ``{\vt Inductance}''.  The inductance per
square is derived from the microstrip parameters for the layer, as
provided with the {\et Tline} specification.  A {\et Tline}
specification must be given to one of the device body layers, or an
error results.

Presently, the recommended way to set up inductors for extraction is
through the use of three additional layers.  These layers can have any
name but will have the following names in this discussion:

\begin{description}
\item{\et LB}\\
Used to outline the inductor, will surround the region of a
conductor where inductance is to be measured.

\item{\et LC}\\
Identifies the inductor contacts (inside {\et LB} and on the
conductor).

\item{\et LX}\\
Bisects the conductor into two areas to provide separate groups for
the two contacts.
\end{description}

\begin{figure}
\caption{\label{indx} Illustration of the configuration of layers
LB, LC, and LX for extracting inductance from a conducting strip.
The LX ensures terminal assignments to different groups.}
\vspace{1.5ex}
\begin{center}
\epsfbox{images/indx.eps}
\end{center}
\end{figure}

These layers have been added to the {\vt xic\_tech.hyp} file provided. 
The ``{\vt Exclude LX}'' clause must be added to the {\et Conductor}
specification of the conductors to be extracted.


% -----------------------------------------------------------------------------
% ext:devtmpl 032117
\subsection{Device Templates}
\index{device template}

A device template is a device block with special syntax that allows
text substitution.  The {\vt Template} line in a device block will
read the referenced template while performing the substitutions.  A
template can be used to save common keywords associated with a class
of device, for example MOS transistors, that may have several styles
in a process,

The substitution text replaces forms like ``{\vt \$(\_}{\it N\/}{\vt
)}'', where {\it N} is a positive integer.  This indicates that the
text of the {\it N\/}'th argument in the {\vt Template} line will
replace this form.  Any macros or technology file variables found in
the line will also be expanded at this time.

The first line of a device template consists of the keyword {\vt
DeviceTemplate}, followed by a name for the template.  The template
name can be just about any text word, and is used to reference the
template.  The final line of the template contains the keyword {\vt
End}.  Intervening lines are the same as device block lines, with
substitution sequences where needed.  The {\vt Device} keyword need
not appear.

Device templates can bo defined in the technology file, or in a file
named ``{\vt device\_templates}'' found in the current directory or
library path.  A default {\vt device\_templates} file is provided in
the {\vt startup} directory in the installation area.  This contains
two example templates:  {\vt NmosTemplate} and {\vt PmosTemplate}. 
These provide generic recognition of MOS transistors.  When a
technology file is written with the {\cb Save Tech} button in the {\cb
Attributes Menu}, only the device templates originally read from the
technology file will be included.  Device blocks will be written with
the {\vt Template} lines expanded.

Example:\\
Here's part of a device template definition.\\
\begin{quote}\vt
DeviceTemplate mostmpl\\
Name \$(\_1)\\
Prefix M\\
Body \$(\_2)\\
...\\
End
\end{quote}

Here's a device block in the technology file.\\
\begin{quote}\vt
Device\\
Template mostmpl nmos active\_layer\&poly\_layer\\
End
\end{quote}

Here's the post-substitution device block.\\
\begin{quote}\vt
Device\\
Name nmos\\
Prefix M\\
Body active\_layer\&poly\_layer\\
...\\
End
\end{quote}


% -----------------------------------------------------------------------------
% xic:fmtlib 062016
\subsection{Format Library File}
\index{format library}
\label{fmtlib}
{\Xic} provides a mechanism for user-specified formatting of physical
and electrical netlist output.  Such formatting is generated by
scripts found in a file named ``{\vt xic\_format\_lib}''.  This file
need not exist if user formatting is not required.

An example {\vt xic\_format\_lib} file is included in the
distributions.  This provides two examples each, for physical and
electrical output.  In either case, the first example is the Cadence
Design Exchange Format (DEF), which is an industry-standard ASCII
netlist format.  The second format in each case is a simple example,
not a ``real'' format.  The example library is found in the startup
directory, and can be used as-is or as a starting point for
customization.  The example format scripts include instructive
comments.

The {\vt xic\_format\_lib} file is searched for in the library search
path, and the first such file found will be used.

There are three types of script that can appear in the file:  those
for generating netlists from physical data, those that generate
netlists from electrical data, and those that format the output of LVS
runs (this is not supported yet).

Blank lines, and lines that start with the `\#' character, are ignored. 
There are four keywords (outside of the scripts) that are recognized:
\begin{quote}
{\vt PhysFormat} {\it name}\\
{\vt ElecFormat} {\it name}\\
{\vt LvsFormat} {\it name}\\
{\vt EndScript}
\end{quote}

One of the first three of these keywords and its argument should
appear on its own line ahead of a script, and ``{\vt EndScript}''
should appear on its own line following a script.  The {\it name} is
the name of the format, which will appear on command or menu buttons
or is given to script functions to indicate that the following script
is to be used for formatting.  This should be a short alpha-numeric
word or phrase, and must be unique among keywords of a given type.  If
the {\it name} contains white space, it should be double-quoted.

The script lines can contain any of the script library functions and
operators.  All local variables are static.  The script can call
functions that have been previously defined in a regular library file.

When the script is executed:
\begin{itemize}
\item{The ``standard output'' is to the file being generated, and not
 to the console window as for normal execution.}
\item{The script will be called for each cell in the hierarchy, to
 a depth given in the invoking command.  On each call, the ``current
 cell'' is set to the cell being processed.}
\end{itemize}

When the script is executing, the following predefined variables are
available for use in the script.

\begin{tabular}{|l|l|p{3in}|}\hline
\bf Name & \bf Type \bf Description\\ \hline
\vt \_cellname & string & name of the cell being output\\ \hline
\vt \_viewname & string & ``physical'' or ``electrical''\\ \hline
\vt \_techname & string & {\vt TechnologyName} value from technology file\\
 \hline
\vt \_num\_nets & integer & number of wire nets in cell\\ \hline
\vt \_mode     & integer & 0 if physical, 1 if electrical\\ \hline
\vt \_list\_all & integer & 1 if {\cb list all cells} active, 0 otherwise\\
 \hline
\vt \_bottom\_up & integer & 1 if {\cb list bottom-up} active, 0 otherwise\\
 \hline
\vt \_show\_geom & integer & 1 if {\cb include geometry} active, 0 otherwise\\
 \hline
\vt \_show\_wire\_cap & integer & 1 if {\cb show wire cap} active, 0
 otherwise\\ \hline
\vt \_ignore\_labels & integer & 1 if {\cb ignore labels} active, 0
 otherwise\\ \hline
\end{tabular}

The script will use functions that iterate through the cell and print
the desired information in an order and format desired.  The function
library is being expanded to provide flexibility.



% -----------------------------------------------------------------------------
% xic:excfg 103113
\section{The {\cb Misc Config} Button: Misc. Extraction Settings}
\index{Misc Config button}
The {\cb Extraction Setup} panel appears in response to pressing the
{\cb Setup} button in the {\cb Extract Menu}.  The panel has four
tabbed pages:  {\cb Views and Operations}, {\cb Net and Cell Config},
{\cb Device Config}, and {\cb Misc Config}.  These will be described
in the sections that follow.

Common to all pages are two buttons which will invalidate or initiate
extraction.  This is usually done automatically --- extraction is
invalidated if the design changes somehow, and recomputed when needed. 
The buttons can force recomputation which may be useful on occasion.

\begin{description}
\item{\cb Clear Extraction}\\
Pressing this button will clear the internal validation flags, which
will cause {\Xic} to recompute extraction when extraction results are
next needed.  This is normally done automatically if the layout or
schematic changes, or a setup variable is changed.

\item{\cb Do Extraction}\\
Pressing this button will perform the full extraction and association,
if necessary.  This is normally done automatically when needed within
commands.  Once done, flags are set indicating the validity of the
current extraction data structures, avoiding recomputation unless
something changes, or {\cb Clear Extraction} is pressed.
\end{description}

% 110613
% -------------------------------
\subsection{The {\cb Views and Operations} Page}

At the top of the page is the {\cb Show} group, containing check boxes
that make visible certain features related to extraction.

\begin{description}
\index{Extraction View button}
\item{\cb Extraction View}\\
When the {\cb Extraction View} check box is active, the display in the
main window is based on features as known to the extraction system. 
Although similar to the normal display, there are important
differences:

\begin{enumerate}
\item{Only the conducting layers are shown.}
\item{The features from internally-flattened subcells are shown as part
of the parent cell.}
\item{The geometry shown represents the processing from the
{\vt Conductor Exclude} directive.}
\item{The visible geometry includes ground-plane processing.}
\end{enumerate}

This viewing mode is compatible with most other commands, however when
active, editing is prevented.  Object and subcell selection works in
the normal way, however only visible objects can be selected, which
includes the ``phony'' objects created by the extraction system and
not present in the actual geometry database.

\index{Groups button}
\index{conductor groups}
\item{\cb Groups}\\
The {\cb Groups} check box causes the group number of each conducting
object to be displayed in physical windows.  This is similar to the
{\cb Nodes} check box, and is mutually exclusive with that check box.

\index{Nodes button}
\item{\cb Nodes}\\
The {\cb Nodes} check box causes the associated node name of each
conducting object to be displayed in physical windows.  This is
similar to the {\cb Groups} check box, and is mutually exclusive with
that check box.

\index{All Terminals button}
\item{\cb All Terminals}\\
The {\cb All Terminals} check box displays the terminals in the
physical layout which correspond to terminals in the electrical
schematic.  Electrical devices and subcircuits may have terminals
associated with their nodes.  These terminals are used to identify the
connection points in the physical database.  During association, the
terminals are automatically placed at the appropriate point in the
physical cell.

Should association fail, unplaced terminals are grouped in an array to
the lower left of the physical cell's bounding box.  Also potentially
visible after failure, to the right of the physical layout, are any
unassigned subcircuit labels.

This check box is mutually exclusive with the {\cb Cell Terminals
Only} check box.

\index{Cell Terminals Only button}
\item{\cb Cell Terminals Only}\\
This is similar to the {\cb All Terminals} display mode, however only
the cell's connection points are shown, not the connection points of
devices or subcells.  This check box is mutually exclusive with the
{\cb All Terminals} check box.
\end{description}

The {\cb Terminals} group provides buttons which initiate commands and
modes related to terminal placement and parameters.

\begin{description}
\item{\cb Reset Terms}\\
Pressing this button will move all device and subcircuit terminals out
of the layout, and array them outside the lower left of the cell.  The
instance connection points will become undefined.  This operation
cannot be undone except by re-running extraction.  If the
{\cb Recursive} check box is checked, this operation will be
performed recursively on all cells in the hierarchy.

The cell's contact terminals that have been explicitly placed by the
user, and consequently have the {\vt FIXED} flag set that locks the
position, are not touched by this operation.

This capability is mostly for debugging.  It may be entertaining to
make all terminals visible, then press this button.  Zooming out will
reveal the terminals arrayed outside the lower left corner of the
cell.  Then, on pressing {\cb Do Extraction}, the terminals will snap
back to their locations within the layout.  The locations of these
terminals are set by {\Xic}, unlike the cell connection terminals
which can be placed by hand.

The same capability is available from the {\cb !ptrms} text-mode
command.

\item{\cb Reset Subcircuits}\\
Pressing this button will move all of the subcircuit marker labels to
an array outside of the upper right of the layout.  This will undefine
the subcell associations with the schematic.  This operation can not
be undone except by re-running extraction.  If the {\cb Recursive}
check box is checked, this operation will be performed recursively on
all cells in the hierarchy.

This capability is mostly for debugging.  It may be entertaining to
make terminals visible (which also makes the instance label marks
visible), then press this button.  Zooming out will reveal the
instance marks arrayed outside the upper right corner of the cell. 
Then, on pressing {\cb Do Extraction}, the marks will snap back to
their locations within the layout.  The locations of these marks are
set by {\Xic}, and presently they can not be manually placed.

The same capability is available from the {\cb !ptrms} text-mode
command.

\item{\cb Recursive}\\
When this check box is checked, the {\cb Reset Terms} and {\cb Reset
Subcircuits} buttons will act recursively in the hierarchy of the
(physical) current cell.  If not set, the operations are performed in
the current cell only.

\index{Edit Terminals button}
\item{\cb Edit Terminals}\\
In electrical mode, this button has the same effect as the {\cb subct}
button (see \ref{subct}) in the electrical side menu.  A command is
entered enabling the user to define and edit the current cell's
connection terminals.

In physical mode, the {\cb Edit Terminals} button makes visible the
current cell's contact terminals, if any have been assigned with the
{\cb subct} command.  These terminals are automatically placed during
association (if possible) however this command allows manual placement
and editing of the properties of the terminals.

Terminals are moved and selected for editing using the same mouse and
keyboard operations as in the {\cb subct} command.  One can click
twice or drag the terminals to a new location.  Multiple terminals can
be selected and moved (unlike in electrical mode) by dragging over
them with the mouse pointer, which displays a rectangle.  The selected
terminals are ghost-drawn and attached to the mouse pointer, during
the move operation.  In this state, pressing either the {\kb
Backspace} or {\kb Esc} keys will deselect the terminals and abort the
pending move.  Terminals can not be deleted or created in this
command, these operations must be done in the {\cb subct} command.

If the {\kb Shift} key is held while the user clicks on a terminal, or
the user double-clicks on a terminal, including the case of a ``move''
to the same location, the {\cb Terminal Edit} pop-up appears, just as
in the {\cb subct} command.  The entries and effects are the same as
are described for that command.

It is usually not necessary to place terminals manually.  Exceptions
are cells with ambiguous connection points.  For example, suppose a
cell contains a single resistor, with cell contacts ``C1'' and ``C2''
to the resistor.  {\Xic} will assign the physical locations of the
terminals arbitrarily, which may not be the locations expected in a
parent cell.  For example, if the physical resistor is a vertical
strip, a parent cell may expect C1 above C2, whereas {\Xic} might have
assigned the reverse.  The user can move the terminals to the proper
locations, bypassing the assignment in {\Xic}, and the locations are
made permanent when the cell is saved.

If association fails to place a terminal, or it is placed in the wrong
location, then manual placement should be used.  If the new location
is over a conductor, that node/group association is assumed before the
association operation (so it had better be correct, or association
will not be correct).

The cell terminals have a {\vt FIXED} flag which will be saved in
cell files if set, the purpose of which is to prevent {\Xic} from
reassigning the physical location of the terminal.  This flag will be
set whenever the terminal is moved by the user.  Once moved, the
terminal should always remain in that location (which had better be
correct for extraction/LVS to succeed).

The state of the flag is indicated by the check box in the {\cb Terminal
Edit} pop-up with label ``Location locked by user placement''. 
This flag can be set or unset with the check box.

When a terminal is placed, {\Xic} searches through the conductor
groups that touch the terminal for a suitable object to associate with
the terminal.  The object must touch the terminal, be on a {\et
Routing} layer, and match the layer hint given to the terminal, if
any.  The hint layer can be supplied with the {\cb Terminal Edit}
pop-up, and is otherwise the last layer that the terminal was
associated with (if any).  If no object can be found that matches the
hint, the hint is ignored and any {\et Routing} layer will be used. 
If an association is made, the layer name is printed near the terminal
marker.  If not, no layer name will be printed, and the terminal no
longer has a hint.  If the cell has not yet been associated, the layer
name label may not appear.  The actual association will be made the
next time the cell is processed, which occurs when entering most of
the extraction commands.  In particular, activating the {\et All
Terminals} or {\cb Cell Terminals Only} check boxes is a benign way to
force a recalculation of all associations.  Better still is the {\cb
Do Extraction} button at the bottom of the panel.

\index{Find Terminal button}
\item{\cb Find Terminal}\\
The {\cb Find Terminal} button brings up the {\cb Node (Net) Name
Mapping} panel, which can also be brought up from the electrical side
menu (see \ref{nodmp}).  In physical mode, the net naming operations
are unavailable, however the facility for locating nets and terminals
is always available.
\end{description}

Below the {\cb Terminal} group are the {\cb Select Unassociated}
buttons.  These can be used after extraction to identify the objects
that failed to associate.  These are objects in physical mode that
have no identified electrical counterpart, and vice-versa.

The command works in physical and electrical modes.  Display windows
will highlight the appropriate unassociated objects for the window's
display mode.

The highlighting is removed on a deselect operation, with the menu
button or otherwise.  Mostly, the objects are simply selected, however
objects such as physical devices use other highlighting methods.

\begin{description}
\item{\cb Groups/Nodes}\\
Highlight nets and net objects (object groups) that are not associated
between layout and schematic.

\item{\cb Devices}\\
Highlight device symbols and layout elements that are not associated
between layout and schematic.

\item{\cb Subckts}\\
Highlight subcircuit symbols and placements that are not associated
between layout and schematic.
\end{description}

This functionality is also available from the {\cb !ushow} text-mode
command.

% 061916
% -------------------------------
\subsection{The {\cb Net Config} Page}

This page provides control over aspects of net identification and
naming.

At the top of the page is an input area and four check boxes.  These
relate to the net naming capability (see \ref{netname}) using text
labels in physical mode.

\begin{description}
\item{\cb Net label purpose name}\\
This group provides a text entry area and an {\cb Apply} button.  The
{\cb Apply} button must be active for the control to have any effect.

The text entered is the name of the purpose to be assumed for net
naming labels.  This tracks the {\et PinPurpose} variable.  Pressing
{\cb Apply} will set the variable to the text name provided, which can
be empty.  Pressing {\cb Apply} again will unset the variable
reverting to the default purpose name of ``{\vt pin}'', but the text
in the entry area will be retained.

If the purpose name is set to an empty string, the ``{\vt drawing}''
purpose is assumed.  One could equivalently give the name explicitly. 
This is not really recommended as it can be inefficient.  In this
case, the label, and presumably its metal, would both reside on the
same layer with the default purpose.  For example, a label on a
routing layer named ``{\vt M2}'' placed over an object on {\vt M2}
would name the net containing the object.  For efficiency, it is
better to use a separate purpose.  In the default case, the label
would be on a layer purpose pair named ``{\vt M2:pin} in this example.

\item{\cb Net label layer}\\
This works similarly to {\cb Net label purpose name}, though sets the
{\et PinLayer} variable.  This is set to a layer name (or
layer-purpose pair name) on which all net labels must reside.  When
set, the purpose mechanism is not used.  This is for compatibility
with older cell libraries, and is otherwise not recommended.

\item{\cb Ignore net name labels}\\
If this check box is checked, existing net name labels will be
ignored.  This is probably only useful for debugging.  The {\et
IgnoreNetLabels} variable tracks the state (set or not set) of this
check box.

\item{\cb Find old-style net (term name) labels}\\
Earlier releases of {\Xic} recognized ``term labels'' in the layout as
identifying the conductor groups associated with cell terminals. 
These are labels, optionally created by the user.  They must be
created on a conducting layer, and placed over an object on the same
layer.

If this box is checked, {\Xic} will search for these labels, and treat
them as net labels.  These labels are formally identical to net labels
if the {\cb Net label purpose name} is applied as an empty string, or
the purpose name ``{\vt drawing}''.  In this case, the search for term
labels, which is a separate search from the search for net labels,
would be redundant.

This check box links to the set/unset status of the {\et
FindOldTermLabels} variable. 

\item{\cb Update net name labels after association}\\
When checked, net name labels will be updated, and new net name labels
possibly created, after association completes.  The label text is
obtained from corresponding electrical net names.

{\bf WARNING:  don't use this feature unless you understand the
potential consequences}.  It is essential that association be correct
when labels are created or updated.  Incorrect labels in a layout will
prevent correct association and will cause LVS errors.  These would
probably need to be removed or corrected by hand.

To use this, once LVS passes for a cell, one can clear extraction with
the {\cb Clear Extraction} button, check this check box, then press
{\cb Do Extraction}.  The layout will now contain the new and updated
labels.  Uncheck this check box, and save the cell.  This should only
be done as a final step when creating a new cell, not while the cell
is likely to change.

The presence of the net labels should reduce the time needed to
associate the cell.  Otherwise the added labels are redundant.

The {\et UpdateNetLabels} variable tracks the state (set or not set)
of this check box.

\item{\cb Merge groups with matching net names}\\
If two physically unconnected conductor groups have the same logical
net name (see \ref{netname}), if this box is checked the groups will
be logically merged and treated as a single group.  This allows
successful top-level LVS of cells containing split nets.  Below the
top level, split nets are detected by other means so checking this box
is not required for successful LVS if the top-level contains no split
nets.

The group names that apply are obtained from net name labels, or from
cell terminals that have been placed by the user.  By default, net
name matching is case-insensitive, though this can be changed with the
{\et NetNamesCaseSens} variable.  The name matching also treats as
equivalent various subscripting delimiters, as listed in the
description of the {\et Subscripting} variable.

The {\et MergeMatchingNamed} variable tracks the state of this check box.
\end{description}

\subsubsection{Via Detection}
\index{via detection}
\label{viafind}

Below the check boxes mentioned above is the {\cb Via Detection}
group.  This controls how vias are searched for and identified.  As
most layouts contain large numbers of vias, the algorithms used to
detect vias can have a significant impact on extration time.  The
default settings provide the least testing and the speediest
performance, however they assume that a certain layout methodology has
been followed.  If all vias in the layout are separate cell instances,
and the via masters contain patches of the two conductors along with
the via layer, then the default settings will always apply.  {\bf If
this is not the case, it is possible for connections to be missed, the
user must understand the rest of this section}.

The recognition algorithm is as follows.

\begin{enumerate}
\item{An object on a via layer is found, usually by iterating through
the spatial database over a region.}

\item{If the object is not a box object, and the {\cb Assume convex
vias} check box is checked, it is replaced temporarily by a box of
half the width/height of the bounding box.  This is for efficiency --
the geometrical computations are much faster for box objects.  In
practice, a via is almost always a square, but other shapes, mostly
circular approximations, are occasionally seen.  In particular, users
of the Hypres process often use circular vias concentric with circular
Josephson junctions.  If the check box is not set, the via shape is
decomposed into a trapezoid list.}

\item{We look for objects on the upper and lower conductor layers
whose intersection intersects the assumed via shape with nonzero area. 
If found, a connection is indicated.}
\end{enumerate}

The initial ``grouping' phase establishes networks of metal objects in
the cells, which contact by proximity or through vias.  The next
``extraction'' phase modifies the network to account for device
connections, and establishes connections to and between subcircuits. 
In this process, ``wire-only'' subcircuits are logically flattened
into the parent cells.  There are two things to keep in mind about
this flattening process.

\begin{enumerate}
\item{Only conductors are promoted into the parent cell.  In
particular, via material is not promoted, and must be accessed through
the original cell hierarchy}.

\item{Consider a via cell, consisting of metal caps and a via pattern. 
The metal caps are connected, trivially.  When the metal caps are
promoted, they will tie together any contacting metal in the parent
cell.  Thus, we never have to recognize the via again, it has imposed
its connectivity constraint and logically disappeared.}
\end{enumerate}

Connecting to and between subcells is a complicated and labor
intensive operation.  We have to iterate through all wire nets of all
subcells, and test for connectivity.  Connectivity can occur by direct
contact, or, conceivably by a via.  However, if we know that all of
the vias are separate cells as described, we know that they have all
been flattened away, and the hugely expensive test of looking for via
connections between nets can be skipped entirely.  If in the original
layout a via (cell) is making the connection, in extraction the
proximity test will discover the promoted, shorted via caps, and make
the connection.

What if not all vias are in separate cells?  In theory, the via
material, and the two connected metal objects, may each occur in any
subcell at any hierarchy level.  In the most general case, we would
need to search the entire hierarchy depth for via material, which can
be painfully slow.  However, other rules can apply.  For example, to
make a contact, one could place a square of via material on the
current cell, over the metal layer intersection area of nets contained
in the cell or subcells.  If this method is used to connect between
subcircuits, then this test must be enabled, however the search depth
can remain at zero.  If the via material is found in a subcell, then
the search depth would have to be set appropriately.

The {\cb Via Detection} group contains the following elements.

\begin{description}
\item{\cb Assume convex vias}\\
This applies when checking connectivity through a via during
extraction.  When set, vias that are not rectangular are assumed to be
convex polygons, and connectivity testing is performed in a small
rectangular region near the center of the bounding box.  This is
specifically for Hypres-style circular vias.  This simplifies testing
and might speed extraction when circular vias are present.  It should
{\cb not} be used if vias can take arbitrary polygonal shapes.  This
will have no effect on rectangular vias.

\index{ViaConvex variable}
This sets/tracks the state of the {\et ViaConvex} variable.

\item{\cb Via search depth}\\
If we have intersecting areas of top and bottom conductor, and we are
searching for an area of via material that would connect the two metal
objects, this sets the depth in the current cell hierarchy to search. 
The default is zero, indicating to search the current cell only. 
Generally, layout methodology can easily ensure that this value can be
safely zero, but there may be cases that require extraction where such
methodology was not practiced.  In such a case, where the methodology
is completely unknown, this value should be set to a large number
(internally it is limited to 40, the maximum cell hierarchy depth)
which will ensure that all via-induced connections are found.  This
can dramatically increase extraction time.

\index{ViaSearchDepth variable}
This tracks the value of the variable {\et ViaSearchDepth}, and
defaults to zero if the variable is not set.

\item{\cb Check for via connections between subcells}\\
By default, it is assumed that connections between subcells will be
made by touching metal only.  This includes the case where the metal
is from a flattened wire-only cell, as would be provided by via cells
as described previously.  One can easily adapt layout methodology
where this is true.  Otherwise, this box can be checked, which will
cause explicit testing for the presence of vias between subcircuit
nets.  This is a very expensive operation.

\index{ViaCheckBtwnSubs variable}
This tracks the whether or not the {\et ViaCheckBtwnSubs} variable is
set.
\end{description}

\subsubsection{Ground Plane Handling}
\index{ground plane handling}

The {\cb Ground Plane Handling} group controls how the extraction
system treats a ground plane.  The is only required if the technology
file defines a ground plane layer, which is unlikely to be true in
semiconductor processing.  The ground plane handling features were
included specifically for Josephson junction process support, but can
be applied to other technologies should the need arise.

\begin{description}
\item{\cb Assume clear-field ground plane is global}\\
If a clear-field ground plane has been identified in the technology
file, when this box is checked, all areas of this layer are assigned
group 0, the ground group.  When not checked, only the largest area
group in the top-level cell is assigned group 0.  This tracks the
{\et GroundPlaneGlobal} variable.

\item{\cb Invert dark-field ground plane for multi-nets}\\
If a dark-field ground plane layer has been identified in the
technology file, if this box is checked, the ground plane layer will
be polarity inverted internally for extraction purposes.  The inverted
layer will be used to establish connectivity.  This tracks the state
of the {\et GroundPlaneMulti} variable.

\item{\cb Inversion method menu}\\
When using an inverted ground plane, this menu provides a choice of
methods.  The default is to invert in each cell, then clip out the
area occupied by subcells.  The second choice will create the inverted
layer in the top-level cell only, using the entire hierarchy as the
source for geometry to invert.  The third choice is similar, but will
create the inverted layer in each cell, using as the source all
geometry in that cell and its subcell hierarchy.  This tracks the
state of the {\et GroundPlaneMethod} variable.
\end{description}

% 063014
% -------------------------------
\subsection{The {\cb Device Config} Page}

This page contains controls that correspond to device-related
extraction variables.

\begin{description}
\item{\cb Device Block}\\
Pressing the {\cb Device Block} button produces a drop-down list of
device blocks from the technology file, plus three additional buttons: 
{\cb New}, {\cb Delete}, and {\cb Undelete}.  The device blocks are
listed in order of their definition, as the block name followed by the
prefix, if any.  Pressing {\cb New} or any of the device block name
entries brings up a text editor loaded with the indicated block, or
empty for {\cb New}.  The text for the device block can be entered
into the editor.  Adding a block with the same name and prefix (or
lack of a prefix) as an existing block will overwrite the existing
block.  Saving with the {\cb Save} button in the editor will update an
existing block or add a new block to the internal device list.  The
{\cb Save} button in the editor {\it does not} save to disk.  The {\cb
Save Tech} command can be used to generate a new technology file which
will contain the new block, or the {\cb Save As} button in the editor
can be used to save the block as a file.

To delete a device block, press the {\cb Delete} button in the {\cb
Device Block} menu, then select a device block from the same menu. 
That block will be removed from the menu.  The name will disappear
from the menu, and it is removed from consideration in extraction. 
The block can be restored with the {\cb Undelete} menu entry, but only
one deletion is remembered.

\item{\cb Don't merge series devices}\\
If checked, series-connected devices will never be merged during
extraction, overriding any {\vt Merge} directive in the corresponding
device blocks of the technology file.  This tracks the state of the
{\et NoMergeSeries} variable.

Similar devices may be series-merged if the common connection has no
other connection.  It is occasionally useful to suppress merging to
individually measure the components of segmented devices, or in cells
where the common contact may not have a connection that is currently
in scope.

\item{\cb Don't merge parallel devices}\\
When checked, parallel-connected devices are never merged during
extraction, overriding any {\vt Merge} directive in the device blocks
of the technology file.  This tracks the state of the {\et
NoMergeParallel} variable.

If is occasionally useful to suppress parallel merging to individually
measure segments of a multi-component device.

\item{\cb Include devices with terminals shorted}\\
By default, if an extracted device is found to have all terminals
shorted together, it is ignored.  If this check box is checked, such
devices are kept, allowing their parameters to be reported.  This
tracks the state of the {\et KeepShortedDevs} variable.

\item{\cb Don't merge devices with terminals shorted}\\
When including devices with all terminals shorted, checking this box
will prevent such devices from being merged as parallel devices, if
parallel merging is enabled for the device type.  This tracks the
state of the {\et NoMergeShorted} variable.

\item{\cb Skip device parameter measurement}\\
When this box is checked, device parameter measurement will not be
performed during extraction.  This may save time if the user is
interested only in topology.  This tracks the state of the {\et
NoMeasure} variable.

\item{\cb Use measurement results cache property}\\
When this box is checked, the extraction system will read and update
(creating if necessary) the {\et measures} property (property number
7106) which is used to cache measurement results (see
\ref{meascache}).  The measurement of device parameters can be time
consuming, and the caching can speed up the extraction process
significantly.  However, using the measurement cache may require user
intervention to maintain coherency.  If a device layout changes, the
user will have to manually update the cache in order to obtain updated
parameters.  With this box not checked, the default condition will
force actual computation of device parameters, and avoid all use of
the caching mechanism.  This is appropriate while a cell is under
development, to avoid cache coherency issues. 

The {\et UseMeasurePrpty} variable tracks the state of this check
box.

\item{\cb Don't read measurement results from property}\\
This setting is ignored unless {\cb Use measurement results cache
property} is checked.  When this box is checked, the extraction system
will not read the {\et measures} property (property number 7106) which
is used to cache measurement results.  When measurement results are
required, they will be computed.  The property will still be updated,
after association, provided that {\cb Use measurement results cache
property} is set.  Thus, by setting this check box and forcing
association, one can get a fresh set of measurement results into the
{\et measures} properties.

The {\et NoReadMeasurePrpty} variable tracks the state of this check
box.
\end{description}

The remaining controls apply to the numerical solver used to extract
resistance and (microstripline) inductance.

The default mode of the solver is to divide the device body into a
grid such that the number of grid cells is fixed, independent of
device size.  This gives a predictable and constant measurement time
per device, however it may provide less accuracy than other methods.

One can also choose to use a fixed grid size, in which case physically
larger devices will take longer to extract, but computed values may be
more accurate.

A third choice is to tile the structure, if possible, by using the
largest grid such that all body and contact boundaries fall on grid. 
This is likely to provide the best accuracy if tiling succeeds.

\begin{description}
\item{\cb Set/use fixed grid size}\\
If the check box is checked, the solver will use a fixed grid size as
given in the numeric entry area.  When checked, other controls in this
group are grayed and their states ignored.  This tracks the state and
value of the {\et RLSolverDelta} variable.

\item{\cb Try to tile}\\
When checked, the solver will try to use a grid that falls on every
edge of the contacts and device body.  This tracks the state of the
{\et RLSolverTryTile} variable.

\item{\cb Maximum tile count per device}\\
When tiling is enabled, this entry area will set the maximum number of
tiles allowed in a device.  This tracks the state of the {\et
RLSolverMaxPoints} variable, and defaults to 50,000.

\item{\cb Set fixed per-device grid cell count}\\
This entry area supplies a number of grid cells to use per device.  In
this mode, the time required for extraction is close to constant,
independent of device size.  This mode is used when not tiling, and
not using a fixed grid size.  This tracks the state of the {\et
RLSolverGridPoints} variable, and the default value is 1000;
\end{description}


% 091614
% -------------------------------
\subsection{The {\cb Misc Config} Page}

This page contains some miscellaneous settings and controls, to be
described.  The topmost controls on this page apply to cell handling.

\begin{description}
\item{\cb Cell flattening name keys}\\
This group contains a text entry area and an {\cb Apply} button.  When
the {\cb Apply} toggle button is in the pressed state, the {\et
FlattenPrefix} variable is set to the text in the entry area.  When
the {\cb Apply} button is not pressed, the text in the text area is
retained, but the {\et FlattenPrefix} variable is unset.

This is one means by which the user can force a physical cell to
always be logically flattened (see \ref{exthier}) in the extraction
system.  It is a bit archaic, as the extraction system will do most
required flattening automatically, and use of the {\et flatten}
property is the recommended way to force instance flattening.

\item{\cb Extract opaque cells, ignore OPAQUE flag}\\
Checking this will cause the extraction system to ignore the OPAQUE
flag applied to subcells, and attempt to extract the contents as for a
normal cell.  This tracks the state of the {\et ExtractOpaque}
variable.
\end{description}


\begin{description}
\item{\cb Be very verbose on prompt line during extraction}\\
When set, during extraction lots of progress messages are displayed on
the prompt line.  However, this can actually slow down the process
considerably, particularly if running remotely.  By default, the
messages are much more terse, and there is very little progress
indication.

\item{\cb Global exclude layer expression}\\
This group provides an {\cb Apply} button and text entry area.  The
text entry area should contain the name of a layer, or layer
expression.  When the {\cb Apply} button is pressed, the {\et
GlobalExclude} variable will be set with the text from the entry. 
When the {\cb Apply} button is unset, the text is ignored, and the
{\et GlobalExclude} variable is unset.

If an expression is given and active, any object that touches an area
where the layer expression is ``dark'' will be ignored in extraction,
as if it didn't exist.

For example, one might create a layer named ``{\vt NOEX}''.  Then, if
the layout contains features that should be ignored by the extraction
system, one can enclose the features in {\vt NOEX} boxes, and if
``{\vt NOEX}'' is entered into the text entry area and the {\cb Apply}
button pressed, the marked features will be ignored. 
\end{description}

The {\cb Association} group contains settings that apply during the
association operation, when electrical and physical matching occurs.

\begin{description}
\item{\cb Don't run symmetry trials in association}\\
When checked, the association computation will not break symmetry and
run symmetry trials.  This is mostly for debugging.  The {\et
NoPermute} variable tracks the state of this check box.

\item{\cb Logically merge physical contacts for split net handling}\\
When set, some additional association logic is employed to detect and
account for split nets in instance placements.  A ``split net'' is a
logical net consisting of two or more disjoint physical conductor
groups.  The disjoint parts of the net are connected when instances
are placed, through parent cell metalization.  If the schematic shows
the net fully connected in the master, LVS will fail on the parent
unless this check box is checked.

The {\et MergePhysContacts} variable tracks the state of this check
box.

\item{\cb Apply post-association permutation fix}\\
Checking this check box enables additional association logic.  It
applies when there is perfect topological matching between layout and
schematic, but LVS is failing due to different permutations of
permutable subcell contacts being assumed in the electrical and
physical parts.  Checking the box will enforce the electrical
permutation on the physical solution, which will allow LVS to pass if
the permutation difference was the only issue.

This should no longer be needed, as the two-pass association algorithm
in current use should resolve these cases automatically.  This check
box should therefor not be checked in general, but it is possible that
it might allow successful LVS in some obscure case.

The {\et SubcPermutationFix} variable tracks the state of this check
box.

\item{\cb Maximum association loop count}\\
This sets the maximum number of ``loops'' allowed around the
association logic.  There is no reason to touch this unless directed
by Whiteley Research.  The {\et MaxAssocLoop} variable tracks this
entry.

\item{\cb Maximum association iterations}\\
This sets the maximum number of iterations allowed per loop.  There is
no reason to touch this unless directed by Whiteley Research.  The
{\cb MaxAssocIters} variable tracks this entry.
\end{description}


% xic:exsel 090914
% -----------------------------------------------------------------------------
\section{The {\cb Net Selections} Button: Path Selection Control Panel}
\index{Net Selections button}
The {\cb Net Selections} button in the {\cb Extract Menu} brings up
the {\cb Path Selection Control} panel.  This panel enables
extraction-specific selection modes for groups, nodes, and connected
conductor paths (wire nets).  It is separate and distinct from the
normal object and subcell selection.  Command buttons in the panel
replace menu buttons and modes found in other commands in earlier
releases of {\Xic}, in particular the group/node selection found in
the {\cb View Extraction} mode, and the {\cb Show Paths} and {\cb
Quick Paths} commands found in the {\cb Extract Menu} of {\Xic}
releases 3.1.4 and earlier are now available from this panel.

There are three basic selection modes available, which are set from
the top row of buttons in the panel.  Similar to normal selections,
one clicks on an object in a drawing window.  The object must be on a
visible and selectable layer.  Other selection specifications as found
in the {\cb Selection Control Panel} apply as well.  In particular,
one can choose the types of objects that are selectable, and the
search-up mode.  In search-up mode, layers are searched from bottom to
top, rather than the default top to bottom, in physical mode.  This
affects the conductor chosen if the user clicks over more than one
conductor layer.

\begin{description}
\item{\cb Select Group/Node}\\
When active, clicking on a conducting object in the current cell in a
physical window will highlight all objects of the current cell in that
conductor group.  In electrical windows displaying the same cell, the
corresponding node connections and wires will be highlighted. 
Clicking on a connection point or wire in an electrical window will
highlight that node, and also highlight the corresponding group in
physical windows.

Physical objects are ``conducting'' if the {\vt Conductor} keyword was
applied (directly or by inference) to the layer of the object.

If the {\cb Select Path} button is also pressed while in this mode,
the conducting path, as it recurses into subcells, will also be shown
in physical windows.

Pressing the {\kb p} key will toggle the state of the {\cb Select
Path} button and the display of recursive conductor paths.

With the mouse pointer in a physical window, typing a group number
followed by {\kb Enter} will switch to the display of that group and
corresponding node.  Similarly, in an electrical window, entering a
node name or number followed by {\kb Enter} will switch the display to
the entered node and corresponding group.  However, entering a name
will probably trigger all kinds of accelerators, including those for
this command, so there is a trick.  Type a single or double quote
character, followed by the node name.  The quote character will
inhibit recognition of accelerators.  Since the keypress buffer length
is only 16 characters, long node names can't be entered in this
manner, the equivalent node number can be entered or some other method
used to select the node.

In electrical mode, the command works with the {\cb Node Mapping
Editor} when visible.  The currently selected node will always be
highlighted in the node list panel of the editor.  Selecting a node in
the editor will highlight that node/group in the display windows.

\item{\cb Select Path}\\
When the {\cb Select Group/Node} button is also pressed, physical
windows will display the conducting path of the currently selected
group descending into subcells.  Otherwise, this button initiates a
different command for displaying conductor paths recursively.  This
mode is available in physical mode only.  Clicking on a conducting
object will highlight the conductor path containing that object. 
There is no selection or indication of the corresponding electrical
node, nor will clicking in an electrical window have any effect in
this mode.  The clicked-on object need not be in the current cell (as
is required for group/node selection), but must be within the search
depth.  The path generation algorithm makes use of the extraction
system, and observes extracted devices and exclude directives as
provided to the extraction system.

Only one path can be shown at a time.  Clicking on another object will
rebuild a path from the second object, erasing the original path, or
it is possible to select a sub-path, it that feature is enabled.

If a dark-field ground plane is used, clicking on the painted areas
(holes in the ground plane) will select the ground group, as will
clicking on any other object which is connected to ground (group 0).

\item{\cb "Quick" Path}\\
This command is similar to the {\cb Select Path} command, but does not
use the extraction system, except for establishing conducting layers
and connections through vias.  In particular, there is no information
about devices and other extraction constraints established at higher
levels.  It may be useful for tracing wire nets, while skipping the
sometimes lengthly extraction operation.

The {\cb "Quick" Path} algorithm, unlike {\cb Show Paths}, will ignore
layers that are set invisible.

Since extraction is not used, there is no concept of devices, so that
results may not be as expected, and not be as seen with the {\cb Show
Paths} mode.  For example, consider MOS devices.  Since, the source
and drain are connected to a common area of the ``active'' layer,
which is (usually) a {\vt Conductor}, the simple algorithm used in
this mode will interpret the source and drain as being connected
together, since it does not recognize the MOS device.  As a
consequence, all wire nets are likely shorted together in this mode!

In order to get meaningful results, it may be necessary in this case
to temporarily remove the {\vt Conductor} keyword from the active
layer.  This can be accomplished with the {\cb Tech Parameter Editor}
in the {\cb Attributes Menu}.
\end{description}

The remaining buttons and controls in the panel provide options or
modes while the selections are active.

\begin{description}
\item{\cb "Quick" Path ground plane handling}\\
This menu applies only to the {\cb "Quick" Path} selection mode, and
sets the ground plane handling method.  This tracks the setting of the
{\et QpathGroundPlane} variable.  If a dark-field ground plane ({\et
GroundPlaneClear} keyword) has been specified in the technology file,
the implied connectivity to ground is similar to that in force for the
extraction system.  There are three choices for handling the ground
plane.

\begin{description}
\item{\cb Use ground plane if available}\\
This is the default.  If an inverted ground plane has already been
created and is current, it will be used when determining paths.  If
the ground plane does not have a current inversion, the absence of the
layer will imply a ground contact, as in extraction without the {\et
MultiNet} keyword.  This choice avoids the sometimes lengthly
inversion computation, but makes use of the inversion if it has
already been done.  The inversion is performed during extraction.

\item{\cb Create and use ground plane}\\
If the extraction system would use an inverted ground plane, it will
be created if not already present and current.  The path selection
will include the inverted layer.

\item{\cb Never use ground plane}
The {\cb "Quick" Path} mode will never use the inverted ground plane.
\end{description}

\item{\cb Search path depth}\\
This control and associated buttons apply when the {\cb Select Path}
or {\cb "Quick" Path} modes are in effect.  It determines the depth to
recurse to when the conductor path is being constructed.  If 0, only
objects in the current (top-level) cell will be considered.  The depth
can be entered directly, or by clicking the up/down buttons, or by
pressing the {\cb 0} or {\cb All} buttons.

While the command is active, the expansion depth can also be changed
with the {\kb -}, {\kb +}, {\kb n}, and {\kb a} keys.  These
decrement, increment, set to 0, and set to maximum, the depth,
respectively.

When the depth changes, the path, if one is being shown, will be
redrawn, if possible (the original object must be above the new
depth).

\item{\cb "Quick" Path use Conductor}\\
If this check box is not checked, only objects on layers with the {\et
Routing} attribute applied will be considered for inclusion in the
extracted path.  If checked, objects on layers with the {\et
Conductor} attribute will be allowed.  The {\et Routing} attribute
implies {\et Conductor}, but may be more restrictive.

The {\et QpathUseConductor} variable tracks the state of this check
box.

\item{\cb Blink highlighting}\\
{\bf Accelerator:} {\kb h}\\
When this box is checked, the highlighting in physical windows will
blink.  When unset, the highlighting will use the static highlighting
color.  Associated highlighting in electrical windows will always
blink.

With a path being displayed, pressing the {\kb h} key will toggle the
blinking status.

\item{\cb Enable sub-path selection}\\
This check box enables sub-path selection while in the {\cb Select
Path} or {\cb "Quick" Path} modes.

When a path is displayed, the user can click on two objects in the
path, and only the ``sub-path'' connecting the two objects will be
highlighted.  If the two objects are connected in multiple ways, the
algorithm will select one (which may not be the most direct).  If {\kb
Shift} is held while clicking on an object in the path, the object
will be deselected and not considered as part of the path.  This can
be used to coerce a desired sub-path.  When a sub-path is displayed,
clicking on any non-selected object will display the full path
containing that object.

\item{\cb Load Antenna file}\\
{\cb Accelerator:} {\kb f}\\
This button applies to the {\cb Select Path} mode only.  Pressing this
button will load a previously-generated antenna report file (from the
{\cb !antenna} command) for the current cell, and ask the user for a
net number found in the file.  The conductor path corresponding that
that net number will be highlighted.

Pressing the {\kb f} key while in {\cb Select Paths} mode will also
query an antenna report file in a similar manner.

\item{\cb To trapezoids}\\
{\cb Accelerator:} {\kb t}\\
Pressing this button will decompose the geometric objects which
comprise the currently shown physical conductor path into trapezoids. 
This has no effect on ``real'' objects in the database or in the
extraction system, only the temporary objects used to display the
selected path.

This can be useful in conjunction with the sub-path selection
capability, to enable breaking a path by deselecting parts of an
object that are separate as trapezoids.  It may also be useful as a
prelude to the {\cb Save} operations in some cases.

Pressing the {\kb t} key with a path displayed will also convert the
path to trapezoid representation.

\item{\cb Save path to file}\\
{\cb Accelerator:} {\kb s}\\
If a physical conductor path is being displayed, this button enables
saving the objects that comprise the path to a native cell file.  Only
the selected objects will be exported.  If the path has been converted
to trapezoids, the trapezoid representation will be exported. 
Pressing the button brings up a small pop-up where the user can give a
cell/file name.  The resulting file can be read into {\Xic} at a later
time for further processing, or for conversion to another file format.

By default, the via layers are not included in the file, only the
conductors.  The two check boxes below the button allow saving the
vias and other associated layers as well.

Pressing the {\kb s} key with a path displayed will also save the path
to a file in a similar manner.

\item{\cb Path file contains vias}\\
This check box applies when the {\cb Save path to file} button is
used.  When checked, the via objects that connect layers will be
included in the generated path.  If not checked, only the metal layers
that constitute the path will be included in the file.  The via layers
are those that have the {\et Via} keyword defined in the technology
file.  The file will included the objects on the via layers, clipped
to the intersection area of the two associated conductors.

\item{\cb Path file contains via check layers}\\
This check box applies when {\cb Save path to file} is used, and
the {\cb Path file contains vias} check box is checked.

The {\et Via} keyword line in the technology file contains an optional
layer expression, which must be ``true'' for an actual connection to
be indicated.  For example

\begin{quote}
{\vt Via SBST MET1 DIFF\&PPLS}
\end{quote}

This line would indicate that the layer containing this line forms a
via between conductors SBST and MET1 only in the presence of layers
DIFF and PPLS.

When this check box is checked, the file will contain the layers
needed for the checking expression (DIFF and PPLS), clipped to the via
layer objects.  If not checked, the file will contain only the vias
that meet the check criteria, but the layers needed for checking (DIFF
and PPLS) will not appear.

With this box checked, the file can be loaded into {\Xic} and
extraction run, and the (single) net will be completely identified. 
This may not be the case if check layers are missing, and certainly
won't be the case if via layers are omitted.
\end{description}

The two via inclusion check boxes track the state of the {\et
PathFileVias} variable.  If this variable is set as a boolean (i.e.,
to no value), then vias will be included, and check layers will not be
included.  If the variable is set to any text, the check layers will
also be included.

\subsection{Resistance Measurement}
\index{resistance measurement}

The {\cb Resistance Measurement} buttons allow the user to measure the
resistance between two points of the currently highlighted path.

Caveat:  This is a new capability.  The algorithm seems to have
difficulty with some, usually complex, paths, meaning that a ``pivot
too small'' or other error message will appear indicating lack of a
solution.

All layers used in the path should have a sheet resistance specified. 
If no sheet resistance is specified, a value of 1 ohm/square is
assumed.  The sheet resistance can be specified directly with the {\et
Rsh} keyword, or can be obtained if {\et Rho} or {\et Sigma} and {\et
Thickness} have been given.  If {\et Rsh} is not given, the value is
taken as {\vt 1e6*}{\et Rho}/{\et Thickness}, where {\et Rho} has
units of ohm-meter and {\et Thickness} is given in microns.  The
conductivity {\et Sigma} is equal to 1.0/{\et Rho}.  These keywords
can be set in the technology file, or with the {\cb Tech Parameter
Editor} in the {\cb Attributes Menu}.

To perform a measurement, the {\cb Define Terminals} button should be
used first to define two terminal locations.  With the button pressed,
drag mouse button 1 to define a rectangular area over some part of the
displayed path.  A box will be shown.  Note that one must drag, with
the mouse button pressed, to define the terminal area.  Simply
clicking has no effect.  Repeat the process over another part of the
displayed path, and a second box will be shown.  These boxes represent
the equipotential terminal areas assumed in the solver.

Once the terminals have been defined, pressing the {\cb Measure}
button should display the measured resistance on the prompt line. 
Diagnostic messages from the solver will be printed in the console
window. 

The algorithm does not include contact resistance between 
different metal layers.


% -----------------------------------------------------------------------------
% xic:dvsel 103013
\section{The {\cb Device Selections} Button: Show/Select Devices}
\index{Device Selections button}
The {\cb Device Selections} button in the {\cb Extract Menu} brings up
the {\cb Show/Select Devices} panel, from which devices can be made
visible, and certain operations can be performed.  There are three
basic control groups.

The top control group contains a window which lists all of the devices
extracted from the physical layout of the current cell.  The listing
has three columns.  The {\cb Name} and {\cb Prefix} columns provide the
values supplied in the technology file device block for the device. 
The third column gives the range of index values assigned for the
device instances extracted.  Each instance of a device has a unique
index in this range.

The list is actually shown in response to pressing the {\cb Update
List} button.  This will perform extraction/association on the current
cell, if necessary, and list the devices found.  With entries listed,
the buttons above the listing become active.  These buttons allow
devices to be highlighted in the display windows.

Devices are highlighted is all windows showing the physical layout of
the current cell.  In addition, the corresponding electrical devices
are also highlighted in windows showing the electrical schematic of
the current cell.

Lines in the listing can be selected by clicking on the text.  The
buttons and other controls above the listing have the following
functions.

\begin{description}
\item{\cb Show All}\\
All devices will be highlighted in the drawing windows.

\item{\cb Erase All}\\
Erase all device highlighting in the drawing windows.

\item{Show}\\
The devices corresponding to the current selection in the list will be
highlighted in the drawing windows.  These are the devices that match
the {\cb Name} and {\cb Prefix} selected, and whose indices are
matched in the {\cb Indices} entry text.

\item{\cb Erase}
The devices corresponding to the current selection in the list will be
un-highlighted in the drawing windows.  These are the devices that
match the {\cb Name} and {\cb Prefix} selected, and whose indices are
matched in the {\cb Indices} entry text.

\item{\cb Indices}\\
This is a text entry area where the user can provide a list of index
integers and ranges to specify the index values of devices to
highlight or un-highlight with the {\cb Show} and {\cb Erase} buttons. 
If the entry contains no text, all indices are used.  The text
consists of space or comma separated integers, or ranges of integers
where the minimum and maximum values are separated by a hyphen (minus
sign).  For example:  ``{\vt 1,2-5,7,9-12}''.
\end{description}

The second basic control group appears below the devices list, and
enables devices to be selected by clicking on the device structure in
the drawing windows.  The command mode in initiated by pressing the
{\cb Enable Select} button.  When in this mode, clicking on a device
in a physical window showing the current cell will apply blinking
highlighting to the device.  The corresponding electrical device (if
any) will also be shown with blinking highlighting in windows showing
the electrical schematic of the current cell.  In such windows,
electrical device symbols can be clicked on, which will select the
corresponding physical device.

Only one device can be selected at a time.

When the mode is active, the two check boxes to the right of the {\cb
Enable Select} button become active.  When checked, information about
new selections will be presented.

\begin{description}
\item{\cb Show computed parameters of the selected device}\\
When this box is checked, when a device is selected, the parameters
extracted for the device will be printed on the prompt line.  The
format of the output is defined in the device block following the {\et
Cmput} keyword.

\item{\cb Show elec/phys comparison of selected device}\\
When this check box is active, clicking on a device will show a
comparison of the extracted parameters and the corresponding
electrical values for the device obtained from the schematic.

If one clicks on a device with the {\kb Shift} key held, the
electrical device properties will be set from the parameters extracted
from the corresponding physical device.  In an electrical window
showing the device symbol, the device property labels will appear or
change when the properties are set or updated.

The physical device must be specified in a device block, and have at
least one parameter with the {\et LVS} keyword specified.
\end{description}

The third basic control group allows electrical parameters for the
current layer to be measured for a rectangular region.  If also allows
rectangular regions to be painted, and can be used as an alternative
to the {\cb box} command in the side menu.  Unlike the other two basic
control groups, this group is only active in physical mode.

Electrical information is applied to layers in the with the {\et Rsh}
keyword for resistance (or alternatively {\et Rho} or {\et Sigma}
along with {\et Thickness}), the {\et Capacitance} keyword for
capacitance, and the {\et Tline} keyword for transmission line
parameters.  The electrical specifications may be added or edited with
the {\cb Edit Tech Params} command in the {\cb Attributes Menu}.

When the {\cb Enable Measure Box} button is pressed, the command mode
becomes active.  The user can drag or click twice in physical windows
to define a rectangular area.  This area will be outlined with a
highlighting box.  During the creation, and after the box is created,
the electrical parameters, if any, from the current layer are applied
to the box dimensions, and the electrical parameters are displayed.

Once the box is created, pressing the {\cb Paint Box} button, or
pressing the {\kb p} key, will paint the highlighting box with the
current layer, creating a box object in the cell.

This is useful for creating simple rectangular resistors, for example,
as the readout facilitates creating the proper size for the desired
resistance.  The command can also be used to measure the values of
existing rectangular resistors.

The mouse operations can be repeated, as long as the command remains
active.  Only one highlighting rectangle is available at a time.


% -----------------------------------------------------------------------------
% xic:sourc 030715
\section{The {\cb Source SPICE} Button: Update From SPICE File}
\index{Source SPICE button}
The {\cb Source SPICE} button in the {\cb Extract Menu} allows
electrical information in a schematic to be updated or generated by
reading a SPICE file.  Pressing {\cb Source SPICE} brings up a small
pop-up containing an entry area for the name of a SPICE file to read,
and three check boxes.  The entry area is active as a drop receiver,
so that the {\cb File Selection} panel (or another file manager
program) can be used to locate the file, and the name can be dragged
into the entry area.  The {\cb Go} button will actually initiate the
operation.

Node name mapping is turned on after the operation completes.  Since a
schematic produced in this way has every node name defined by a
terminal, using the defined names, which correspond to the original
SPICE file, is convenient.

The three choice buttons are:
\begin{description}
\item{\cb all devs}\\
If set, all devices in the cell which match a name in the SPICE file
will be updated.  If not set, only the devices that have names that
were set explicitly by the user (by applying a name property) are
updated.
\item{\cb create}\\
If set, devices specified in the SPICE file that are not found in the
schematic are created.  If not set, only the properties of existing
devices are updated.  If the current cell is empty, {\cb create} is
taken as set.
\item{\cb clear}\\
If set, the electrical part of a cell is cleared before reading the
SPICE input.  This implies {\cb create}.
\end{description}

If {\cb create} is set, or the target cell is empty, this command will
create a schematic hierarchy from the SPICE file.  The function may be
used as follows:  open a new cell and go to electrical mode.  Use the
{\cb Source SPICE} button to read in a SPICE file.  The devices and
subcircuits referenced in the file will be arrayed in the drawing,
each with the appropriate properties applied.  Named terminals are
placed at each device contact point, which establish connectivity
(wires are not used).  The drawing can be used for simulation or any
purpose just as a schematic entered in the standard way.  The created
schematic can be modified by the user to replace the named terminals
with wires and reset the device locations, to make a ``real''
schematic that is aesthetically decent.

Subcircuits are created as needed.  They must be written out later
(e.g., with the {\cb Save} command).  If a file exists in the search
path with the same name as a subcircuit, it is ignored, as the
subcircuit cells are created internally.  When writing, therefore, it
is possible to replace an existing cell file, but the previous version
is retained with a ``{\vt .bak}'' extension.

Devices are instantiated as needed, and given an assigned name from
the SPICE file.

If {\cb create} is not active, no new devices or subcells will be
instantiated in a non-empty cell, though devices in the drawing with
names which match those in the SPICE file will have their properties
updated.  Properties of existing devices are updated whether or not
{\cb create} is active.  Similarly, if a subcircuit already exists,
its devices will be updated, but no new devices will be created in the
subcircuit.

If {\cb all devs} is not set, only devices that have been assigned a
name by the user will have properties updated.  Devices with
internally assigned names are skipped.  This is to avoid problems due
to the fact that internally assigned names will change when the
circuit is edited, and updating from an out-of-sync SPICE file could
be a disaster.

If {\cb clear} is set, then the electrical part of the cell and
subcells will be cleared before the SPICE information is read.  This
ensures that the cells contain only information supplied in the SPICE
file.

In order to determine if a semiconductor device is a p-type or n-type,
{\Xic} will look for a corresponding model in the source file, or the
model library if not found.  If still not found, if the model name
starts with ``n'' or ``p'', or if the model name contains ``n'' but
not ``p'' or {\it vice-versa\/}, {\Xic} will infer the type.  If none
of this succeeds, the operation is aborted, and the user must provide
access to the device model.

{\Xic} will also test the consistency of MOS models defined in the
technology file (used when extracting physical data) and the MOS model
assumed for use in the device library (usually the {\vt device.lib}
file).  If the node count differs, a warning will be issued.  The
warning indicates that LVS will fail.  See the discussion in the
description of the {\et DeviceKey} global device library property in
\ref{devglobprop} for more information.

All ``dotcards'' that are not otherwise handled are written verbatim
in the top-level schematic as labels on the SPTX layer.  Recall that
the labels on this layer are ``spicetext'' labels (see
\ref{spicetext}), so that the label text is included in SPICE output
generated from the cell.  Labels will not be created if a label with
matching text already exists.

There is no inclusion of text from {\vt .include} or {\vt .lib} lines
or similar, these become labels on the SPTX layer in the top-level
schematic.

Subcircuit calls that have no subcircuit definition will be written as
spicetext labels, and a warning will be given.  It is likely that they
are resolved at simulation-time through {\vt .include} or {\vt .lib}
inclusions.

All {\vt .model} lines found in the SPICE source are written to a
file in the current directory named ``{\it cellname\/}{\vt
\_models.inc}'', where {\it cellname} is the top-level cell name.  In
the schematic of the top-level cell, a label is created on the SPTX
layer containing a {\vt .include} line for the models file (if the
label does not already exist).  Models that are defined within
subcircuits are given a new hierarchical name to ensure uniqueness.

Nested subcircuit definitions are handled by assigning a new
hierarchical name to the subcircuit (cell) which is unique.

Parameter definitions from {\vt .param} lines, subcircuit
definitions, and subcircuit instances are applied verbatim as {\et
param} properties of cells and instances, or as labels on the SPTX
layer for {\vt .param} lines, within the cell corresponding to the
subcircuit where found.

Although parameterization of subcell instances is allowed and works
fine for simulation and other purposes, these parameters are
effectively ignored in LVS.  LVS requires that a unique master be
created for each instantiation parameter set, and the parameterized
instances be replaced by instances of the appropriate master.

Each of the option buttons has a corresponding {\cb !set} variable. 
If the variable is changed while the pop-up is visible, the pop-up
will be updated.  Conversely, changing the state of the option buttons
will set or unset the corresponding variables.  The pop-up check box
will be checked if the corresponding variable is set.  The names of
the corresponding variables are given in the table below.

\begin{tabular}{|l|l|} \hline
\cb all devs & \et SourceAllDevs\\ \hline
\cb create   & \et SourceCreate\\ \hline
\cb clear    & \et SourceClear\\ \hline
\end{tabular}

There are two additional variables that are used by this command. 
These specify the names of the ground and terminal devices, as
provided by the device library file, that this command will use. 
Generally, it is not necessary to set these variables, as the defaults
should always be appropriate.  The user may, however, prefer to use an
alternative terminal style, or may have a custom device library with
different names for these devices from those found in the {\vt
device.lib} file distributed with {\Xic}.

\begin{description}
\item{\et SourceGndDevName}\\
This variable specifies the name of the ground terminal device to use. 
If not set, the name ``{\vt gnd}'' will be assumed.  If this variable
is set to a name, a ground device of that name must appear in the
device library file.

\item{\et SourceTermDevName}\\
This variable specifies the name of the terminal device to use.  If
not set, the name ``{\vt tbar}'' will be assumed, if that name is
found for a terminal device in the device library.  If not found, the
name ``{\vt vcc}'' will be assumed.  If this variable is set to a
name, that name must match the name of a terminal device in the device
library file.
\end{description}


% -----------------------------------------------------------------------------
% xic:exset 113009
\section{The {\cb Source Physical} Button: Update Electrical From Physical}
\index{Source Physical button}
The {\cb Source Physical} button in the {\cb Extract Menu} will update
the electrical part of a design from parameters extracted from the
physical part.  The command works by writing a temporary SPICE file
from the physical database, then updating the electrical database from
the SPICE file.  When the {\cb Source Physical} button is pressed, a
small pop-up appears, which is similar to the pop-up seen with the
{\cb Source SPICE} command, but has no text entry area, and has an
additional {\cb Depth} choice menu which sets the depth into the
hierarchy to process.  The {\cb Go} button initiates the operation.

Node name mapping is turned on after the operation completes.  Since a
schematic produced in this way has every node name defined by a
terminal, using the defined names, which correspond to the physical
group numbers, is convenient.

The first three check boxes have similar functions as in the {\cb
Source SPICE} command.  The remaining check box enables inclusion of
wire-net capacitors.

\begin{description}
\item{\cb all devs}\\
If set, all devices in the cell will be considered for updating If not
set, only the devices that have names that were set explicitly by the
user (by applying a name property) are updated.

\item{\cb create}\\
If set, missing devices are created.  If not set, only the properties
of existing devices are updated.

\item{\cb clear}\\
If set, the electrical part of a cell is cleared before updating. 
This implies {\cb create}.

\item{\cb include wire cap}\\
If set, capacitors that represent routing net capacitance will be
updated, or created if they don't exist and {\cb create} is set. 
These capacitors are given a special name prefix ``{\vt C@NET}'' which
has significance to {\Xic}, i.e., it identifies them as routing
capacitances.  The capacitors are added between the wire nets and
ground.  In order for wire capacitance to be computed, the {\et
Capacitance} keyword must be supplied in the technology file for the
routing layers.

\item{\cb ignore labels}\\
From some tools, cell terminals may be indicated by the presence of a
label on a {\et Routing} layer, positioned such that the label
reference point touches an object on the same layer.  Such labels, if
found, will be used to generate a terminal list for the top-level cell
in the extracted hierarchy, if the existing electrical cell contains
no terminals (or the electrical cell doesn't exist).  If this box is
checked, such labels will always be ignored.
\end{description}

Each of the option buttons has a corresponding {\cb !set} variable. 
If the variable is changed while the pop-up is visible, the pop-up
will be updated.  Conversely, changing the state of the option buttons
will set or unset the corresponding variables.  The pop-up check box
will be checked if the corresponding variable is set, unless the
variable name has a ``No'' prefix, in which case the logic is
reversed.  The names of the corresponding variables are given in the
table below.

\begin{tabular}{|l|l|} \hline
\cb all devs         & \et NoExsetAllDevs\\ \hline
\cb create           & \et NoExsetCreate\\ \hline
\cb clear            & \et ExsetClear\\ \hline
\cb include wire cap & \et ExsetIncludeWireCap\\ \hline
\cb ignore labels    & \et ExsetNoLabels\\ \hline
\end{tabular}


% -----------------------------------------------------------------------------
% xic:pnet 062016
\section{The {\cb Dump Phys Netlist} Button: Dump Physical Netlist}
\index{Dump Phys Netlist button}
\index{physical netlist}
\index{netlist!physical}
The {\cb Dump Phys Netlist} button in the {\cb Extract Menu} creates a
netlist file from the physical connectivity information in the current
cell.  Upon pressing this button, a small pop-up appears, which
provides a number of format options.  The options include the names
from the {\vt PnetFormat} blocks in the format library file, if any. 
The format library provides a mechanism for user-specified formatting
of netlist output.  The supplied {\vt xic\_format\_lib} file contains
a formatter for the Cadence DEF (Design Exchange Format) format, as
well as a simple example format.

There are three built-in format choices:  {\cb net}, {\cb devs}, and
{\cb spice}.  Any combination of the formats can be selected, and the
output will contain a block for each selected format, for each cell.

In addition, there are a number of options which modify the
presentation.  These include {\cb list all cells} and {\cb list
bottom-up}, which apply to all formats, and {\cb show geometry} and
{\cb include wire cap}.  The latter options are enabled when {\cb net}
and {\cb spice} are enabled, respectively, or when a format library
choice is active.

The format options will be described in more detail below.  Below the
format check boxes there is a {\cb Depth} choice menu which allows
setting of the depth into the hierarchy to process.  The user is given
the option of creating the netlist to an arbitrary depth in the
hierarchy.  If the given depth is greater than zero, the subcells
above the indicated depth will also be added to the file.  If ``all''
is selected, the full hierarchy will be output.

Below the depth menu is a text entry area for the name of the file to
be generated.  The default name is the base name of the current cell,
suffixed with ``{\vt .physnet}'', to be created in the current
directory.  The entry area is sensitive as a receiver for drag/drop.

Any combination of the four format options may be selected.  The
states of the option check boxes track the status of the variables
described below.  The listing from the {\cb Dump Phys Netlist} command
will have a field of output for each selected format, from each cell. 
Pressing the {\cb Go} button will produce the output file.

The format option check boxes are described below.  The first two are
options that apply to all formats.

\begin{description}
\item{\cb list all cells}\\
Subcells that are wire-only or otherwise internally flattened or
ignored are normally not listed.  If set, these cells are included in
the listing, which may be useful for debugging.

\item{\cb list bottom-up}\\
When the depth is larger than zero, this check box controls the
ordering of cells in the file.  When selected, the deepest cells (the
``leaf cells'') are listed ahead of their parent cells, thus the
current cell will be listed last.  When not selected, the listing is
top-down.  The current cell is listed first, followed by subcells.
\end{description}

The next three rows of option check boxes specify the internal formats
and options for these formats.

\begin{description}
\item{\cb net}\\
A netlist consisting of the terminal names associated with each
conductor group is generated.

\item{\et show geometry}\\
If this is selected, the {\cb net} part of the output file will
include a listing of the physical objects that comprise the wire net. 
This includes objects from the present cell, and objects that have
been promoted from wire-only subcells.  The objects may not exactly
correspond to the physical objects, for example if the {\et Conductor
Exclude} directive is given.  The objects are listed in a modified CIF
syntax, where units correspond to internal database units.

\item{\cb devs}\\
A list of extracted devices, with information about the device,
including {\et Measure} results, is generated.

\item{\cb spice}\\
A list of the SPICE lines for extracted devices which have a {\et
Spice} specification in the device block is generated.

\item{\cb include wire cap}\\
When active, the SPICE listing will contain capacitors for nonzero
computed wire net capacitance.  These capacitors are given a special
prefix ``{\vt C@NET}'' which has significance to {\Xic}, when applying
LVS.  The capacitors are added between the wire nets and ground.  In
order for wire capacitance to be computed, the {\et Capacitance}
keyword must be supplied in the technology file for the routing
layers.

\item{\cb ignore labels}\\
From some tools, cell terminals may be indicated by the presence of a
label on a {\et Routing} layer, positioned such that the label
reference point touches an object on the same layer.  Such labels, if
found, will be used to generate a terminal list for the top-level cell
in the listed hierarchy, if the existing electrical cell contains no
terminals (or the electrical cell doesn't exist).  If this box is
checked, such labels will always be ignored.

\item{\cb devs verbose}\\
This check box is active when the {\cb devs} check box is checked. 
When checked, it enables printing of additional information in the
device report in the output file.  At present, it will print
information about the individual components of multi-component (series
or parallel merged) devices.
\end{description}

Additional option buttons, if any, correspond to formats specified in
the format library file.  If selected, a text block containing the
output from the format generator will be appended to the file, for
each cell.  The following are available from the stock distribution
format library file.

\begin{description}
\item{\et DEF} \\
This uses a formatting script in the {\vt xic\_format\_lib} file to
generate DEF output.  DEF is a common portable netlisting format.  See
the comments in the {\vt xic\_format\_lib} file in the startup
directory for more information.

\item{\et phys-example}\\
This uses a formatting script in the {\vt xic\_format\_lib} file to
generate output in simple example format.
\end{description}

Each of the option buttons that correspond to an internal format or
option (not the formats from the library) has a corresponding {\cb
!set} variable.  If the variable is changed while the pop-up is
visible, the pop-up will be updated.  Conversely, changing the state
of the option buttons will set or unset the corresponding variables. 
The pop-up check box will be checked if the corresponding variable is
set.  The names of the corresponding variables are given in the table
below.

\begin{tabular}{|l|l|l} \hline
\cb list all cells   & \et PnetListAll\\ \hline
\cb list bottom-up   & \et PnetBottomUp\\ \hline
\cb net              & \et PnetNet\\ \hline
\cb show geometry    & \et PnetShowGeometry\\ \hline
\cb devs             & \et PnetDevs\\ \hline
\cb spice            & \et PnetSpice\\ \hline
\cb include wire cap & \et PnetIncludeWireCap\\ \hline
\cb ignore labels    & \et PnetNoLabels\\ \hline
\cb devs verbose     & \et PnetVerbose\\ \hline
\end{tabular}


% -----------------------------------------------------------------------------
% xic:enet 062016
\section{The {\cb Dump Elec Netlist} Button: Dump Electrical Netlist}
\index{netlist}
\index{Dump Elec Netlist button}
\index{netlist!electrical}
\index{electrical netlist}
The {\cb Dump Elec Netlist} button in the {\cb Extract Menu} creates a
netlist file from the electrical connectivity information in the
current cell.  Upon pressing this button, a small pop-up appears,
which provides a number of format options.  The options include the
names from the {\vt EnetFormat} blocks in the format library file, if
any.  The format library provides a mechanism for user-specified
formatting of netlist output.  The supplied {\vt xic\_format\_lib}
file provides a formatter for Cadence DEF (Design Exchange Format),
and a simple example format.

There are two built-in format choices:  {\cb net} and {\cb spice}. 
Any combination of the formats can be selected, and the output will
contain a block for each selected format, for each cell.  In addition,
there is one format option, {\cb list bottom-up}, which applies to all
formats.

The format options will be described in more detail below.  Below the
format check boxes there is a {\cb Depth} choice menu which allows
setting of the depth into the hierarchy to process.  The user is given
the option of creating the netlist to an arbitrary depth in the
hierarchy.  If the given depth is greater than zero, the subcells
above the indicated depth will also be added to the file.  If ``all''
is selected, the full hierarchy will be output.

Below the depth menu is a text entry area for the name of the file to
be generated.  The default name is the base name of the current cell,
suffixed with ``{\vt .elecnet}, to be created in the current
directory.  The entry area is sensitive as a receiver for drag/drop.

Any combination of the format options may be selected.  The states of
the option check boxes track the status of the variables described
below.  The listing from the {\cb Dump Elec Netlist} command will have
a field of output for each selected format, from each cell.  Pressing
the {\cb Go} button will produce the output file.

The format option check boxes are described below.  The first option
applies to all formats.

\begin{description}
\item{\cb list bottom-up}\\
When the depth is larger than zero, this check box controls the
ordering of cells in the file.  When selected, the deepest cells (the
``leaf cells'') are listed ahead of their parent cells, thus the
current cell will be listed last.  When not selected, the listing is
top-down.  The current cell is listed first, followed by subcells.
\end{description}

The next two option check boxes specify the internal formats.

\begin{description}
\item{\cb net}\\
A netlist consisting of the terminal names associated with each
wire net is generated.

\item{\et spice}\\
A SPICE listing is generated.
\end{description}

Additional option buttons, if any, correspond to formats specified in
the format library file.  If selected, a text block containing the
output from the format generator will be appended to the file, for
each cell.  The following are available from the stock distribution
format library file.

\begin{description}
\item{\et DEF} \\
This uses a formatting script in the {\vt xic\_format\_lib} file to
generate DEF output.  DEF is a common portable netlisting format.  See
the comments in the {\vt xic\_format\_lib} file in the startup
directory for more information.

\item{\et elec-example}\\
This uses a formatting script in the {\vt xic\_format\_lib} file to
generate output in simple example format.
\end{description}

Each of the option buttons that correspond to an internal format or
option (not the formats from the library) has a corresponding {\cb
!set} variable.  If the variable is changed while the pop-up is
visible, the pop-up will be updated.  Conversely, changing the state
of the option buttons will set or unset the corresponding variables. 
The pop-up check box will be checked if the corresponding variable is
set.  The names of the corresponding variables are given in the table
below.

\begin{tabular}{|l|l|} \hline
\cb list bottom-up & \et EnetBottomUp\\ \hline
\cb net   & \et EnetNet\\ \hline
\cb spice & \et EnetSpice\\ \hline
\end{tabular}

If the variable {\et CheckSolitary} is set with the {\cb !set} command
then warnings are issued if nodes are encountered with one connection
only.


% -----------------------------------------------------------------------------
% xic:lvs 030412
\section{The {\cb Dump LVS} Button: Test Layout vs. Schematic}
\label{lvs}
\index{Dump LVS button}
\index{LVS}
\index{layout vs. schematic}
The {\cb Dump LVS} (Dump Layout Vs.  Schematic) button in the {\cb
Extract Menu} compares the netlists obtained from the physical and
electrical data for the hierarchy of the current cell, and lists
topological and electrical differences.  When the {\cb Dump LVS}
button is pressed, a small pop-up appears, which contains a field for
setting the name of the output file, and has provision for setting the
depth into the hierarchy to compare.  The default name for the output
file is the base name of the current cell, with a ``{\vt .lvs}''
extension, and this will be written in the current directory unless a
path is given to the file name.  Entering 0 for the depth compares the
current cell only, 1 compares the current cell and immediate subcells,
and so on.  The user is given a chance to view the output file upon
completion.

If computed wire capacitance is included in the electrical data, the
capacitors will be recognized by virtue of having a special name
prefix ``{\vt C@NET}''" and treated specially.  Unlike other devices,
there is no corresponding physical device.  If found, the values will
be compared with the corresponding computed net capacitance in the
physical data, and an error will be reported if the two numbers differ
by 1 percent or more.  Wire net capacitance is considered only for the
capacitors that are found in the electrical data, i.e., if they are
missing no error is generated.

When the LVS data are printed out, the hierarchy of the
electrical (schematic) part is used as the basis.  This means that
\begin{enumerate}
\item{any physical structures that are not connected to the
top-level cell (directly or indirectly) and are not represented in
the schematic are ignored.}
\item{the reverse is not true: anything in the schematic that doesn't
have a physical counterpart is an error.}
\end{enumerate}

Thus, the schematic is favored, as anything not in the schematic and
not connected physically is considered to be a ``test structure'' and
is generally ignored.  One of the reasons for this behavior is the
potential existence of test cells and structures that might contain
real devices or circuits, which aren't connected to anything but are
used for process analysis.  Generally, one would expect these to be
ignored for LVS purposes.

However, unconnected physical subcells (cell instances) that contain
extracted devices or subcircuits are explicitly checked for and
listed.  If the {\cb fail if unconnected physical subcells} check box
in the {\cb LVS} panel is checked, the presence of unconnected
physical subcircuits will force LVS failure of the cell.  This check
box tracks the state of the {\et LvsFailNoConnect} variable.

\subsection{Parameterization Limitation}

Although electrical subcircuit instance parameterization is allowed
and works fine when generating simulation files for SPICE, it is
ignored in LVS.  The LVS system implicitly assumes that a cell and
its instances are precisely similar, that an instance of a cell is in
all respects defined by the master cell of the instance.  Instance
parameterization is therefor not recognized (but parameters defined
in the cell itself are fine).

One has similar issues with parameterized physical cells.  With
parameterized cells, a unique master is created for each unique set
of instantiation parameters used in the design.  The template cell
``instance'' is not really an instance of the template cell, but is
actually an instance of a master created for a particular parameter
set.

Within LVS, each physical template master would correspond to an
electrical master, and likewise there would be correspondence between
instances.  Presently, all of this must be configured manually.  Work
is ongoing to fully support parameterization through SPICE, physical
and electrical cells, and LVS, in a transparent manner.

\subsection{Using the nophys Property}
\index{nophys property}

The {\et nophys} property can be applied to electrical devices and
subcircuits, causing them to be ignored in the extraction system,
notably in LVS.  Devices that have no physical representation, such as
voltage sources, have this property set by default.

By ``ignoring'' these devices, the device terminals are considered as
open circuits.  However, there are times when it would be useful to
consider these devices as shorted.  For example, suppose that one
wishes to include parasitic series inductance in a resistor during
simulation.  However, this inductance would cause LVS to fail, since
the series inductor added to the schematic has no explicit physical
counterpart.

It is possible to configure the {\et nophys} property to indicate that
when the electrical netlist is generated for use by the extraction
system, the flagged {\et nophys} devices will be forced to have all
terminals connected to the same net, i.e., the terminals are
effectively shorted together.  Thus, the inductor in the example
above, if given this property, would disappear properly during LVS. 
However, when generating a SPICE netlist for simulation, these devices
will be included in the netlist.

There are a number of aspects to using the {\et nophys} property.

\begin{enumerate}
\item{The cached internal electrical netlist can be in one of two
states, respecting shorted {\et nophys} or not.  If there are no
shorted {\et nophys} devices, both representations are the same. 
Functions that require one representation or another will invisibly
rebuild this when needed.}

\item{All operations in the extraction system, including the {\cb
Extract Menu} functions and extraction script functions, will respect
the shorted {\et nophys} property.  This includes the SPICE format
listings from electrical data in the {\cb Extract Menu}.

The {\cb run}, {\cb deck}, and other similar functions in the side
menu that relate to SPICE simulation will {\it never} respect the {\et
nophys} property, these devices will be treated as other devices.}

\item{In electrical mode, {\et nophys} devices are shown in a different
color on-screen (yellow by default, the ``Terminal Color'').}

\item{The {\cb Property Editor} will query the user whether to set
the shorted option when a {\et nophys} property is added.}

\item{There is a {\cb Use nophys} button in the {\cb Node Name
Mapping} editor from the side menu.  This button selects whether or
not to respect shorted {\et nophys} devices in the node listings. 
Shorted devices can obviously change the node numbering.}

\item{The string stored in the {\et nophys} property can either be
``{\vt nophys}'' or ``{\vt shorted}''.  {\Xic} sets these values
according to the state.}

\item{There is an {\vt IncludeNoPhys}
script function which can be used with the existing electrical
netlist access functions to provide the {\et nophys} recognition
state desired.}
\end{enumerate}

\subsection{LVS Output File Format}

For each cell comparison, the LVS system reports four levels of
success.

\begin{description}
\item{CLEAN}\\
Everything was measurable and matched.

\item{PASSED - AMBIGUITY}\\
There were device parameters which could not be compared, but all
comparisons that were done matched.

In the electrical schematic, if component values are
parameterized (i.e., use a token defined in a {\vt .param}
line or similar), or perhaps use {\WRspice} shell
expansion, the value was unavailable.  In earlier releases, a
value was available only if it was a numeric constant.
{\Xic} now provides limited parameter substitution during
LVS (see below).

\item{PASSED - PARAM DIFFS}\\
There were device parameters that differ outside of the
tolerance between electrical and physical.  So actually, only
the circuit topological check passed.

\item{FAILED}\\
Differences in circuit topology were detected.
\end{description}

The overall result for the run is the lowest level in this hierarchy
reported for any cell.

The parameter database and substitution code was imported from
{\WRspice} for use during LVS and elsewhere.  However, not all
capability can be provided.

\begin{enumerate}
\item{Parameters given in subcircuit call lines are ignored in LVS,
making LVS meaningless if these are given in the schematic.
Parameterized instances must be remastered to unique master cells
for the current LVS system.}

\item{There is presently no support for macros defined in
{\vt .param} lines.  However, single-quoted expressions are
fully supported, all math operations and all relevant functions
are available.}
\end{enumerate}

Parameter expansion works as follows:

\begin{enumerate}
\item{When an LVS run starts, the parameters defined in the top-level
cell as {\et param} properties, and all parameters defined in {\vt
.param} lines found in labels on the SPTX layer in the top level
cell, are placed in a table.

In addition, the labels on the SPTX layer are searched for {\vt
.option} lines, and these lines are searched for a {\et parhier}
option, and if found, its setting is saved.  This option can be set
to one of ``{\vt global}'' (the default if not found) or ``{\vt
local}''.}

\item{When comparing devices in the top-level cell, the parameter
table is used to parameter substitute the {\et value} and {\et param}
property strings.  The resulting string should provide numerical
values for comparison to the extracted physical values.}

\item{When comparing in a subcell/subcircuit, the subcircuit {\et
param} properties and {\vt .param} labels are tabulated as for the
top-level cell.  This is merged with the top-level table, and is used
to expand the {\et param} and {\et value} property strings of devices
in the cell.

If the {\vt parhier} option was found, and it was set to {\vt local},
then parameters defined in the subcircuit table will override
conflicting definitions in the top level table.  If {\vt parhier}
wasn't found or was set to {\vt global}, the reverse is true --
top-level definitions will override conflicting definitions in the
subcell.}
\end{enumerate}

The output file produced by LVS contains a block of lines for each
cell in the hierarchy where there is both electrical and physical
information.  Each block may contain several tables, which provide
information about the cell and the electrical/physical associations. 
These tables are described below.

\subsubsection{Conductor group and electrical node mapping}

{\Xic} assigns an integer to every physical wire net (called a
``group'') and to every electrical wire net (called a ``node'', as in
SPICE).  These numbers are in general different.  In addition, a node
may have a text name that was assigned by the user.

This table displays the group to node and node to group mappings.
The entries under the ``node'' heading display the internal node
number in parentheses, followed by the actual node name (which
will simply be the number again if no node name was assigned).

\subsubsection{Formal terminal group associations}

In this listing, the first column is the terminal name, the second
column is the associated group number (you can find the electrical
node from the group/node mapping table).  If association failed for
the terminal, i.e., {\Xic} was unable to place the terminal in the
layout, the word ``UNINITIALIZED'' will appear in the third column. 
This will cause LVS to fail for the cell.

\subsubsection{Physical device associations}

If the physical cell contains devices, then this table will
appear.  Each device of a given type in the schematic is
assigned a number, and devices extracted from the physical layout
are assigned a (generally different) number.

An entry appears for each device extracted from the physical data. 
The first line for the device contains the device name and the
physical index number.  If the device has an electrical counterpart,
the electrical device type (same as the physical name) and electrical
name are printed on the same line, following a colon.  The electrical
name uses the SPICE convention.  This line is followed by a listing of
the device terminals, one line per terminal.  The terminal name and
group number are to the left of the colon.  If the group is
associated, the associated electrical node number (in parentheses) and
name are given to the right of the colon.  These lines are optionally
followed by a listing of extracted parameter values for the device. 
The actual format and displayed parameter set is defined in the
corresponding device block in the technology file.

\subsubsection{Physical subcircuit associations}

If the physical cell contains subcells, then this table will appear. 
The first column gives that name of a subcell found in the physical
cell.  If the cell is actually an array, each element of the array
will be listed, with the array indices in parentheses following the
name.  The second column is the internal index assigned to the subcell
for physical mode.  If there is a corresponding electrical subcell,
the electrical subcell type and name will be shown, following a colon. 
The subcell type is the same as the physical subcell name.  The
subcell name is the subcircuit name in the schematic.  This usually
follows the SPICE convention of using 'X' as the leading character. 
This is followed by a listing of the subcircuit terminals, one line
per terminal.  The physical group numbers in the cell and subcell are
printed to the left of the colon.  To the right of the colon, the
electrical node numbers (in parentheses) and names in the electrical
cell and subcell are printed.  If a group number is not associated,
the corresponding node number is shown as ``-1'' and the node name is
``???''.

\subsubsection{Checking for unconnected physical subcircuits}

Physical subcells that contain extracted devices or subcells that have
no connection to the circuit may be present.  Since the electrical
hierarchy is used for recursion, these are not detected in the
traversal, since they have no representation in the schematic and no
connection to the circuit.  However they are checked for explicitly. 
If any such subcells are found, they will be listed, but otherwise
ignored, unless the {\et LvsFailNoConnect} variable is set, in which
case LVS will fail on the presence of such cells.

\subsubsection{Checking per-group/node terminal references}

For each group/node association, {\Xic} will compare the list of
terminals connected to the physical group with the list of terminals
connected to the electrical node.  The lists should be the same.  This
header may be followed by a list of terminal referencing errors. 
Possible errors are device, subcircuit, and formal terminals that are
connected to the physical group but not the electrical node, or
vice-versa.  Such errors will cause LVS to fail for the cell.

\subsubsection{Summary}

The final table, which always appears, is the summary.  This will
report nonassociations, and will indicate whether the cell passed
or failed the LVS test.

A pass indication is reported for a cell if all of the following
are true:
\begin{enumerate}
\item{All electrical nets, devices, and subcircuits are associated,
meaning that {\Xic} has identified the corresponding object in
the physical layout.}
\item{No associated physical device or subcircuit is connected to an
unassociated group.}
\item{No unassociated physical device or subcircuit has a connection
to an associated group other than the ground group (0).}
\item{Parameter value comparisons between corresponding electrical
and physical devices match.}
\end{enumerate}

Note that having unassociated physical groups, devices, or subcircuits
does not automatically cause failure.  Unassociated groups (random
pieces of conductor material) do no harm, but all groups connected to
associated devices or subcircuits must be associated (have a
corresponding node in the schematic).  It is also possible to have
unassociated physical devices or subcircuits, but none of these can
have a connection to associated groups other than the ground group
(the ground group is used when a ground plane layer is specified).
Thus, the physical layout can have structure not represented in the
schematic, but only if this structure is topologically disjoint from
the associated circuit.

% -----------------------------------------------------------------------------
% xic:exc 090314
\section{The {\cb Extract C} Button: Capacitance Extraction}
\index{Extract C button}

The {\cb Extract C} button in the {\cb Extract Menu} brings up {\cb
Cap Extraction} panel which controls the interface to an external
program used for capacitance extraction.


% -----------------------------------------------------------------------------
% fcinterf 053015
\subsection{The Capacitance Extraction Interface}
\index{capacitance extraction interface}
\index{FastCap program}
\index{FasterCap program}
\label{fcinterf}

The interface uses an external program to extract capacitance values
between conducting features in the layout.  The interface supports the
following capacitance extraction programs:

\begin{enumerate}
\item{The {\it FasterCap} program from {\vt FastFieldSolvers.com}. 
This commercial program is recommended for users with capacitance
extraction as an important workflow element.  The auto-refinement
capability provides the best accuracy with the least amount of setup.}

\item{The {\it FastCap} program from Whiteley Research.  The interface
also provides a crude, linear panel refinement capability which can be
used with this free version of {\it FastCap\/}, which is available
from the Whiteley Research free software archive.  We will refer to
the Whiteley Research program as {\it FastCap-WR} to distinguish it
from the MIT original.}
\end{enumerate}

The interface generates a unified list file, which is compatible with
the programs listed above.  It is {\bf not} directly compatible with
the original MIT {\it FastCap} program, or its derivatives, that
require multiple input files.  There are accessory programs {\vt
lstpack} and {\vt lstunpack} available which convert between the
formats, so the MIT {\it FastCap} can be used in a two-step flow
involving unpacking.

This is the second generation capacitance extraction interface.  The
original capacitance extraction interface, found in releases 4.0.8 and
earlier, was quite a bit more complicated.  The present interface
affords at least the following simplifications:

\begin{itemize}
\item{The interface presently takes material from the current cell
as input, there is no need to select and save things into the
interface.}

\item{There is no ``dataset name'', the file names use the current
cell name as a base name.}

\item{The output file is always a unified list file, there is no ``old
format'' support except via the separate {\vt lstpack} and {\vt
lstunpack} utilities.}

\item{There is no graphical ``partition editor'' as {\it FasterCap}
does not need external refinement.}
\end{itemize}

The new interface, however, is much more flexible and powerful
than the original interface.

\begin{itemize}
\item{There is no longer a fixed assumption that layers are planar, or
that layer ordering must begin with a conductor and alternate with
insulators.  Layers can appear in any order, and any layer can be
planarizing, or not.  If a layer is planarizing, it will have variable
thickness such that the top surface is in one plane.}

\item{There is a new layer-sequencing engine (see \ref{ldb3d}) that is
also used by the {\cb Cross Section} display command (in the {\cb View
Menu}), as well as for inductance/resistance extraction.  Thus, the
cross section display will always faithfully represent the assumptions
used in the interface.  Layer ordering is basically that shown in the
layer table, though {\et Via} layers are allowed to be out of sequence
(likely for drawing visibility reasons).  The ``real'' position of a
{\et Via} layer can be obtained from the layers it references.}

\item{Geometry is taken from the current cell, to all levels of the
hierarchy.  There is provision for use of a special masking layer.  If
this layer is found in the layout, geometry will be clipped to the
patterning on this layer.}

\item{The substrate is now more accurately included in the
calculation, taking into account the actual thickness and lateral
extent.}
\end{itemize}

%-----------------------------------
\subsubsection{Geometry Construction}
\label{subsaoi}

By default, a layer named ``{\vt FCAP}'' will serve as the masking
layer.  If, however, the {\et FcLayerName} variable is set to a
layer name, that layer will provide the masking function.  We will
refer to this layer as the ``mask layer''.

If no objects are found on the mask layer, all geometry in the current
cell will be treated in the interface.  Be advised that the
capacitance extraction will very rapidly become untenable if too much
geometry is included.  The interface itself is not designed to handle
large object collections, though it will remain snappy while
generating files that may take weeks to run.  More than 100 objects is
probably pushing things.  The effective area of interest (AOI) is the
bounding box of the current cell.

If objects are found on the mask layer, then the mask layer pattern is
{\bf and}ed with the other layers, and the resulting geometry is
processed by the interface.  The bounding box of the mask layer
patterning becomes the effective AOI, which can be much smaller than
the cell bounding box.  In any case, the AOI bounds all geometry in
the problem.

The geometry is as shown in the drawing window, though geometry is
saved in an internal representation that removes any overlap of
objects in the original layout.  Outside of the AOI, and above all
geometry and below the substrate, vacuum (relative permittivity of
1.0) is assumed.

The substrate is included in the calculation as follows.  A variable
named {\et SubstrateThickness} can be set to specify the assumed
substrate thickness in microns.  If not set, a thickness of 75 microns
is assumed.  Typically, the substrate thickness would be set in the
technology file with the {\vt SubstrateThickness} keyword, which sets
the variable.  It can be set interactively from the {\cb Params} page
of the {\cb Cap Extraction} panel (see \ref{fcpanel}).

{\bf If the substrate has nonzero thickness}:

\begin{quote}
The boundary of the substrate is taken as the AOI, bloated by the
value given by the {\et FcPlaneBloat} variable.  This is generally
desirable to move the substrate edge effectively away from structures
of interest.  If not set, a value of 0.0 micron is assumed.

Interface panels will be created on the sides and bottom of the
substrate when the input list file is generated.  If a positive {\et
FcPlaneBloat} is given, dielectric interface panels will also cover
the top of the substrate outside of the AOI.
\end{quote}

{\bf If the substrate has zero thickness}:

\begin{quote}
This is obtained by setting the {\et SubstrateThickness} variable to
0.  We attempt to treat the substrate as filling the infinite half
space, though is is not clear how to convey this to {\it FasterCap\/}. 
Outside of the AOI, the substrate/vacuum interface extends to
infinity.  We approximate this with finite panels extending a distance
given by {\et FcPlaneBloat} out of the AOI.
\end{quote}

NOTE:  the original interface made no attempt to deal with the
substrate.  This is reasonable, as the different substrate treatments
should have little effect on results in most cases.

Note also that the {\et FcPlaneBloat} parameter extends the substrate
only, and not the geometry.  This is different from the original
interface, which would also extend the dark-field layers.  To
effectively bloat the geometry as well as the substrate, one can use
the {\vt FCAP} layer in most cases to enlarge the AOI.

%-----------------------------------
\subsubsection{Technology File Setup}

Setup parallels setup of the three-dimensional layer sequence database
(see \ref{ldb3d}), which in turn follows setup of the extraction
system (see \ref{extsetup}).  These sections should be consulted for
detailed setup information, here we provide some supplemental
information.

To the interface, there are two different materials:
\begin{description}
\item{conductors}\\
Conductors will have one of the following:
\begin{enumerate}
\item{Any of the {\vt Conductor}, {\vt Routing}, {\vt GroundPlane},
{\vt GroundPlaneClear}, or {\vt Contact} technology file keywords (or
their aliases) applied.  All of these implicitly give the {\et
Conductor} keyword.}

\item{Any of the {\et Rsh}, {\et Rho}, {\et Sigma}, or {\et Lambda}
keywords applied with a positive value.}
\end{enumerate}

\item{insulators}\\
Insulators will have one of the {\et Dielectric} or {\et Via} keywords
applied, and also the {\et EpsRel} keyword applied with a value of 1.0
or larger.
\end{description}

In addition, layers that are to be used in the interface as conductors
or insulators must have all of the following:

\begin{itemize}
\item{A {\et Thickness} keyword applied with a value greater than
zero.}

\item{Must be visible in the layer table.}

\item{Must not have the {\et Symbolic} keyword applied.}
\end{itemize}

The {\et Dielectric} technology file keyword was added to support this
interface.  This is intended to model an explicit capacitor
dielectric, and differs from {\et Via} layers in the following ways.

\begin{itemize}
\item{Unlike {\et Via}, it is not assumed to be dark field (but {\et
DarkField} can be applied to the layer explicitly).}

\item{Only one {\et Dielectric} keyword can appear per layer (multiple
{\et Via} keywords are allowed).}

\item{Stacking order is as shown in the layer table ({\et Via} layers
are allowed to appear out of order).}

\item{{\et Dielectric} layers are not planarizing by default, {\et
Via} layers are.}
\end{itemize}

The present interface can take layers in any order.  This is in
contrast with the original interface, that required layers to
alternate conductor/insulator starting with a conductor, and ordering
was obtained entirely from {\et Via} references and not the layer
table order.

After all possible layers from the layer table are sequenced, layers
that are not used in the extracted geometry are discarded.  Note that
dark-field layers are inverted, as we are interested in representing
the physical material.  Thus, for example no structure in a {\et Via}
layer (i.e., no vias) in the layout implies the presence of a
continuous film of insulating material, so the layer is actually
present.

The same layer sequencer is used in the {\cb Cross Section} command in
the {\cb View Menu}.  The cross section display and the interface will
always agree on the ordering and planarization of the layers.  It is
therefor a useful visualization tool when setting up the layers in the
technology file.

The {\cb Cross Section} command is also useful for finding errors. 
One possible error occurs when not planarizing, and a thin metal layer
runs over the edge of a thicker dielectric layer.  This will
disconnect the metal between the two sides of the step, which will
cause a failure in the extraction.  Presently, the interface assumes
that the number of conductor groups remains the same before and after
3D processing.  The disconnection can be easily seen in the {\cb Cross
Section} view.

In addition to the layers that describe material geometry, the
interface can make use of a masking layer.  This allows only certain
specified parts of the current cell to be evaluated.  When present,
geometry is clipped to objects on this layer before being processed in
the interface.

By default, a layer named {\vt FCAP} with purpose {\vt drawing} is
assumed for the masking layer.  Such a layer should be defined in the
technology file.  It should be given a GDSII mapping to allow saving
of work containing the layer to GDSII or OASIS files.  As an
alternative, the {\et FcLayerName} variable can be set to the name of
another layer, which will instead provide the masking function.


%-----------------------------------
\subsubsection{Output File}

The output file is a unified {\it FasterCap} list file.  At the top of
the file is a comment containing the layer sequence.  For example:

\begin{verbatim}
* Layers           Plane        Thickness    EpsRel
* Substrate                     75.000       11.900
* Insulator CO     0.000        0.190        4.200
* Insulator VIA1   0.190        0.095        2.900
* Insulator VIA2   0.285        0.095        2.900
* Insulator VIA3   0.380        0.095        2.900
* Conductor M4     0.475        0.220
* Insulator VIA4   0.695        0.095        2.900
* Conductor M5     0.790        0.220
* Insulator VIA5   1.010        0.095        2.900
* Conductor M6     1.105        0.220
* Insulator VIA6   1.325        0.095        2.900
* Insulator VIA7   1.420        0.610        4.200
* Insulator VIA8   2.030        0.610        4.200
\end{verbatim}

The {\vt Plane} is the base elevation of the mask objects of a
planarizing layer, that is, the top surface of the layer minus the
layer thickness value.  This field will be empty for non-planarizing
layers.


% -----------------------------------------------------------------------------
% fcpanel 070814
\subsection{The Cap Extraction Panel}
\index{Cap Extraction panel}
\label{fcpanel}

This panel, brought up by the {\cb Extract C} button in the {\cb
Extract Menu}, controls the interface to external capacitance
extraction programs described above.  The interface can also be
controlled to a large extent with the {\cb !fc} prompt line command.

The panel functionality is divided into three pages, selectable
through the tabs along the top of the window.  Common to all pages is
a {\cb Help} button, status line, and {\cb Dismiss} button.  The
status line indicates the number of background extraction jobs
currently running.

%-----------------------------------
\subsubsection{The Run Page}

The {\cb Run} page contains controls for running the supported
programs, or creating unified list format input files for these
programs.  This is the default page, shown when the panel appears.

\begin{description}
\item{\cb Run in foreground}\\
At the top of the page is the {\cb Run in foreground} check box.  When
checked, the program will run synchronously in the foreground, rather
than asynchronously in the background.  Aside from possibly being
helpful when debugging problems, it is not clear that this mode has
any value.

This check box sets, and is set by, the {\et FcForeg} variable.

\item{\cb Out to console}\\
When the {\cb Out to console} check box is checked, the program output
will be printed in the console window, i.e., the shell window from
which {\Xic} is running.  This is most useful with {\it FasterCap},
which iterates to a solution, and the user can verify that all is well
by watching this output.

This check box sets, and is set by, the {\et FcMonitor} variable.

\item{\cb Show Numbers}\\
When the interface is run to produce an input file, the mutually
connected conducting shapes are identified, and each disjoint group is
assigned a conductor number.  These numbers are used in the input list
file to specify the conductors, and in the output file to identify the
capacitance matrix indices.

When this check box is checked, the conductor numbers will be shown
on-screen, so that the user can easily determine the conductor numbers
associated with the layout objects.  Each number is shown as a
cross-mark, which will appear at a corner of an object in the
conductor group.  The conductor number, and the layer name of the
associated object, are printed next to the cross-mark.

Note that due to the possibility of clipping by an {\vt FCAP} layer,
objects that are connected in the cell may be disconnected when used
in the interface.  In that case, two or more marks may appear over the
same object, or different objects that are touching.  If this is
confusing, one can use the {\cb Layer Expression} panel from the {\cb
Edit Menu} to create temporary layers for visualization, consisting of
the conducting layers {\bf and}ed with the {\vt FCAP} layer.  The
resulting shapes will have unique conductor number marks.

\item{\cb Run File}\\
This button and adjacent text entry allows an arbitrary input file to
be run by the capacitance extraction program currently configured. 
The text area should contain a path to a valid input file for the
configured program.  The program will run, and results will appear, as
for a normal extraction run.

\item{\cb Run Extraction}\\
This button will dump a temporary input file, run the program, and
display the results.  The result file is named {\it cellname\/}-{\it
pid\/}{\vt .fc\_log}, where {\it cellname} is the name of the current
cell, and {\it pid} is the process id of the spawned process used to
run the program.  The file contains listings of the input file
produced by the interface and the output file produced by the program.

By default, the program is run in the background.  The label at the
bottom of the panel will indicate that the job is running.  When
complete, a {\cb File Browser} window containing the result file will
appear.  While jobs are running in the background, one can continue
using {\Xic}.

If the {\et FcForeg} variable is set, from the {\cb Run in foreground}
check box or with the {\cb !set} command, then the program will
instead run in the foreground.  In this case, the result file is named
{\it cellname\/}{\vt .fc\_log}, and {\Xic} will be unresponsive until
the run completes.

\item{\cb Dump Unified List File}\\
This button allows an input file to be generated, which is in a
unified list format compatible with the supported programs.  The
default name for this file is {\it cellname\/}{\vt .lst}, where {\it
cellname} is the current cell name.

\item{\cb FcArgs}\\
This text entry area can be given a string, which will be included in
the argument list when the program is run with the {\cb Run
Extraction} button.  This allows specialized command line options to
be provided during the run, which the user may require.  This entry
field is tied to the {\et FcArgs} variable.

If the interface detects that {\it FasterCap} from {\vt
FastFieldSolvers.com} is being used, and this entry is empty, the
default argument string
\begin{quote}\vt
-b -a0.01
\end{quote}
will be imposed.  A ``{\vt -b}'' option will always be added if
missing from the {\it FasterCap} arguments list, as this argument is
necessary for correct {\it FasterFap} operation in this mode.  The
``{\vt -a}'' option is almost always used, as it specifies
auto-refinement, however it is technically not necessary and won't be
imposed if not given, except in the case where no arguments are given
at all.

\item{\cb Path to FasterCap or FastCap-WR}\\
Near the bottom of the page is an entry area where the path to the
executable program can be edited.  This entry area displays and sets
the {\et FcPath} variable.
\end{description}

%-----------------------------------
\subsubsection{The Params page}

The upper half of this page provides entry areas for parameters used
by the interface related to the substrate, plus a menu for choosing
the units to use in the list file.

\begin{description}
\item{\cb SubstrateThickness}\\
This sets the assumed substrate thickness in microns.  When the
thickness is nonzero, the substrate bottom and sides are assumed to
abut vacuum permittivity.  When the thickness is set to zero, the
substrate is assumed to completely fill the half-space below the
extraction area.

This entry sets, and is set by, the {\et SubstrateThickness} variable.

\item{\cb FcPlaneBloat}\\
This entry contains a length, in microns.  If nonzero, horizontal
dielectric/vacuum interface panels will extend outside of the area of
interest (AOI, see \ref{subsaoi}) along the top surface of the
substrate.  The extension distance is the {\et FcPlaneBloat} distance.

When the {\et SubstrateThickness} is nonzero, the substrate bounding
box, which is the AOI, will be bloated by this value before writing of
the substrate bottom and side interface panels to the list file.  This
will move the abrupt dielectric change at the substrate edge away from
the area of interest.

If the {\et SubstrateThickness} is zero the {\et FcPlaneBloat}
distance should be large enough to represent ``infinity'', but making
it too large will slow down computation.  The model is approximating
the entire half-space filled with substrate dielectric material. 

This entry sets, and is set by, the {\et FcPlaneBloat} variable.

\item{\cb SubstrateEps}\\
This entry supplies the relative dielectric constant asssumed for the
substrate.  This sets, and is set by, the {\et SubstrateEps} variable.

\item{\cb FcUnits}\\
This is an option menu which is used to set the length units used in
files produced by the interface.  Choices are meters, centimeters,
millimeters, microns (the default), inches, and mils.  The selection,
if not the default, will set the {\et FxUnits} variable.  Similarly,
setting the variable with the {\cb !set} command will update the state
of the menu.  The choice currently in effect will be applied when
input files are generated.  The choice of units will not affect the
computed capacitance.
\end{description}

The lower half of the page allows one to crudely refine the raw panels
while being written to the list file.  This is specifically for {\it
FasterCap-WR}, which requires refined panelization for accuracy.  The
{\it FasterCap} program does not require external refinement, which is
a major advantage.  In fact, the refinement provided here should {\bf
not} be used with {\it FasterCap}, as it may interfere with {\it
FasterCap\/}'s refinement.

The refinement is ``crude'' due to each refined panel being
approximately the same size.  If the size is small enough, sufficient
spatial resolution for accurate capacitance calculation is achieved. 
This resolution is needed along edges, and at corners, where there are
strong field gradients, but is gross overkill for most areas.  Since
the solving time is related to the total number of refined panels,
this type of refinement is very inefficient with respect to memory use
and execution speed.

The refinement works as follows.  First, the interface computes the
total area of all conductor and dielectric raw panels that would be
output to the list file.  This area is divided by the {\cb
FcPanelTarget} number provided by the user.  This is a number
approximating the total refined panel count that {\it FastCap-WR} will
need to process.  The solution time should be approximately the same
for the same panel count, independent of the actual geometry.  The
square root of the divided area is used when writing the panels to the
list file.  The raw panels are subdivided so that no panel edge is
longer than this value.

A number like 10000 is probably about right for the {\cb
FcPanelTarget} in providing decent accuracy in a reasonable execution
time.  Larger numbers provide more accuracy, but require larger files
and have longer solution time.  The list file will contain a line for
each refined panel.  The entry area will take numbers up to 1e6, which
is probably unreasonable for a normal computer.
 
\begin{description}
\item{\cb Enable}\\
This check box will enable or disable the refinement.  This should
not be active when using {\it FasterCap}.  When pressed, the {\cb
FcPanelTarget} entry will become un-grayed, and internally the {\et
FcPanelTarget} variable will be set to the number shown in the
{\cb FcPanelTarget} entry area.  When the {\cb Enable} button is set
inactive, the {\et FcPanelTarget} variable is unset.  The {\cb Enable}
button state will reflect whether or not the {\et FcPanelTarget}
variable is set.

\item{\cb FcPanelTarget}\\
This entry area is seensitive only when the {\cb Enable} check box is
checked.  It tracks the value of the {\et FcPanelTarget} variable,
which can be set to a real value of 1e3 -- 1e6.  This will be the
approximate number of refined panels generated in the list file. 
\end{description}

%-----------------------------------
\subsubsection{The Jobs page}

The {\cb Jobs} page contains a list of running background jobs.  Each
entry provides the process identification number (PID), the name of
the executing program, and the local date and time when started. 
Entries can be selected by clicking with the mouse.

When an entry is selected, the {\cb Abort job} button below the list
becomes un-grayed.  Clicking this button will terminate the selected
process.  The user should consider that there is no confirmation and
no ability to resume the run.


% -----------------------------------------------------------------------------
% xic:exlr 090314
\section{The {\cb Extract LR} Button: Inductance/Resistance Extraction}
\index{Extract LR button}
The {\cb Extract LR} button in the {\cb Extract Menu} brings up the
{\cb LR Extraction} panel which controls the interface to an external
program used for inductance and resistance extraction.  Presently, the
{\it FastHenry} program is supported.  The interface should be
compatible with any version of {\it FastHenry} or derivatives.  A
guaranteed-compatible version is distributed as free software on the
Whiteley Research web site ({\vt http://wrcad.com}).


% -----------------------------------------------------------------------------
% fhinterf 032015
\subsection{The Inductance/Resistance Extraction Interface}
\index{inductance extraction interface}
\index{FastHenry program}
\label{fhinterf}
The interface uses an external program to extract inductance and
resistance values along conducting features in the layout.  The
interface currently supports the original {\it FastHenry} program from
MIT, and (presumably) all input format compatible successor and
derivative programs that can be run from a command line.

\begin{itemize}
\item{{\bf FastHenry-WR} from Whiteley Research.  The program is
available available in the free software archive on the Whiteley
Research web site.  This package has been updated to build easily with
newer compilers.  It has been extended to support superconductors, and
incorporates features to accelerate computation.}

\item{{\bf FastHenry} from MIT.  This is the original {\it FastHenry}
three-dimensional inductance extraction program.}
\end{itemize}

This is the second generation interface to {\it FastHenry}.  The
original extraction interface, found in releases 4.0.8 and earlier,
was quite a bit more complicated.  The present interface affords at
least the following simplifications:

\begin{itemize}
\item{The interface presently takes material from the current cell
as input, there is no need to select and save things into the
interface.}

\item{There is no ``dataset name'', the file names use the current
cell name as a base name.}

\item{There is no longer a graphical ``partition editor''.  This was
too cumbersome.  Instead, a simple automatic refinement provision is
included.  It is hoped that one day auto-refinement will be built into
a {\it FastHenry} successor program, as was done in the {\it
FasterCap} program from {\vt fastfieldsolvers.com}.}

\item{There is no longer a graphical terminal definition editor. 
Instead, a special layer is used to define terminal areas.}
\end{itemize}

The new interface, however, is much more flexible and powerful than
the original interface.

\begin{itemize}
\item{There is no longer a fixed assumption that layers are planar, or
that layer ordering must begin with a conductor and alternate with
insulators.  Layers can appear in any order, and any layer can be
planarizing, or not.  If a layer is planarizing, it will have variable
thickness such that the top surface is in one plane.}

\item{There is a new layer-sequencing engine (see \ref{ldb3d}) that is
also used by the {\cb Cross Section} display command (in the {\cb View
Menu}), as well as for capacitance extraction.  Thus, the cross
section display will always faithfully represent the assumptions used
in the interface.  Layer ordering is basically that shown in the layer
table, though {\et Via} layers are allowed to be out of sequence
(likely for drawing visibility reasons).  The ``real'' position of a
{\et Via} layer can be obtained from the layers it references.}

\item{Geometry is taken from the current cell, to all levels of the
hierarchy.  There is provision for use of a special masking layer.  If
this layer is found in the layout, geometry will be clipped to the
patterning on this layer.}
\end{itemize}

%-----------------------------------
\subsubsection{Geometry Construction}

By default, a layer named ``{\vt FHRY}'' will serve as the masking
layer.  If, however, the {\et FhLayerName} variable is set to a layer
name, that layer will provide the masking function.  We will refer to
this layer as the ``mask layer''.

If no objects are found on the mask layer, all geometry in the current
cell will be treated in the interface.  Be advised that the
inductance/resistance extraction will very rapidly become untenable if
too much geometry is included.  The interface itself is not designed
to handle large object collections, though it will remain snappy while
generating files that may take weeks to run.  More than 100 objects is
probably pushing things.  The effective area of interest (AOI) is the
bounding box of the current cell.

If objects are found on the mask layer, then the mask layer pattern is
{\bf and}ed with the other layers, and the resulting geometry is
processed by the interface.  The bounding box of the mask layer
patterning becomes the effective AOI, which can be much smaller than
the cell bounding box.  In any case, the AOI bounds all geometry in
the problem.

The geometry is as shown in the drawing window, though geometry is
saved in an internal representation that removes any overlap of
objects in the original layout.  Outside of the AOI, and above all
geometry and below the substrate, empty space is assumed.

The geometric specification of conductors to {\it FastHenry} consists
of ``node'' definitions, and the definition of ``segments'' that
connect the nodes.  The nodes are points in three-dimensional space,
and touch or are enclosed in conducting objects.  Once the node
locations are assigned, the connecting segments are created.  Nodes
and segments that dead-end are removed, the remaining nodes and
segments can be thought of as a three-dimensional SPICE net, where the
segments represent inductors.  We compute the overall inductance
between different ``terminals'' of the network.

The interface works by tiling.  This assumes that current can flow in
any direction.  A tile is a rectangular prism, with a node in the
center, and a node at the center of each of the six faces.  Segments,
whose dimensions are set by the tile size, connect the central node
with each of the face nodes.  In order for adjacent tiles to make
contact, the face nodes of the adjacent faces must coincide.  This
will be true if the conducting objects are decomposed into tiles
properly, avoiding corners that are adjacent to an edge.  The
decomposition is performed along the X, Y, and Z directions.  Tiles
with a side longer than a given maximum dimension will be subdivided.

This is the appropriate way to handle three-dimensional current flow
through constrictions such as vias, and account for penetration or
skin depth.  However, {\it FastHenry} was originally set up to handle
long, thin conductors where the current flow is assumed to be in a
fixed direction.  Most of the examples use this approach, and use
flags that internally subdivide the segments transverse to current
flow.  This provides suitable accuracy and good efficiency for certain
types of problems.

When the full three-dimensional decomposition is used, the problem
size can quickly grow to the point where {\it FastHenry} can not
provide a solution in a reasonable amount of time.  Approximations
must be employed at this point to reduce segment count.  To a much
greater extent that for capacitance extraction, the user may have to
intervene to fine-tune the process.

The ability to tile requires that the geometry be Manhattan.  Support
for non-Manhattan geometry is provided by first internally
Manhattanizing non-Manhattan objects before processing.  The
granularity of the Manhattanization is controlled by the {\et
FhMinRectSize} variable or the corresponding text entry field in the
{\cb LR Extraction} panel {\cb Params} page.  Note that this can
dramatically increase the segment count, and therefor {\it FastHenry}
run time.  When possible, non-Manhattan features should be avoided
when using this interface to {\it FastHenry}.

%-----------------------------------
\subsubsection{Terminal Definition}

Terminals are the assumed external contact points used when extracting
the inductance matrix.  Unlike capacitance extraction, inductance and
resistance extraction requires terminal definitions, and the results
will depend fundamentally on the terminal locations.

Terminals are specified by creating boxes or polygons and labels on
special layers.  The feature will define as equivalent all nodes that
touch or are enclosed in the shape, for any Z coordinate (the features
are in the X-Y plane).

Each terminal feature must have at least one overlapping text label on
the same layer, that provides the terminal name.  Terminals must
resolve to pairs, where each pair is a ``port'', taking the inductance
matrix as an N-port network.  The pair is ordered, with one terminal
being the ``plus'' terminal, the other the ``minus'' terminal.

This is all accomplished by adherence to the following rules.

\begin{enumerate}
\item{The features (boxes or polygons) and labels which equivalence
nodes and define terminals are created on special layers.  The layer
name is the same as the layer name of the conductor which provides the
nodes.  The purpose name is the special keyword ``{\vt fhterm}''. 
Thus for example, for a metal layer named ``{\vt M1}'' (which has the
default ``{\vt drawing}'' purpose) the corresponding special layer has
the full layer-purpose pair (LPP) name ``{\vt M1:fhterm}''.  Such a
LPP should be defined for each conducting layer in the technology
file.}

\item{Terminal features must touch or enclose at least one node. 
Nodes can be found at the center of each edge of each tile.  When
connecting to the end of a metal strip, for example, the entire
transverse width of the strip end should be enclosed in or touch the
terminal feature, so that current flow is uniform.}

\item{Each terminal feature must have at least one overlapping text label
on the same layer giving a terminal name.}

\item{Each terminal is one of a pair, the pair representing a port. 
The terminals of each pair must contact the same conductor group,
i.e., be connected.}

\item{It is possible for a terminal to be used in more than one port,
in which case the terminal will have more than one overlapping label.}

\item{Port-terminal association is by name.  Name labels must follow
these rules:
\begin{enumerate}
\item{Terminal names consist of a port name and a suffix.  If the name
string contains punctuation or white space, the first occurrence of
such is stripped, and the port name is taken as the characters to the
left, and the suffix is taken as the characters to the right, of where
the punctuation or white space resided.  If there is no punctuation or
white space, the port name is the name string with the rightmost
character stripped, and the suffix is this character.  The port name
and suffix must each contain at least one printable character or a
fatal error results.  The port name is arbitrary, but must be unique
among the ports.}

\item{Both terminals of a port must have the same port name, case
sensitive.  It is a fatal error if a terminal can not be paired.}

\item{Both terminals of a port must have different suffixes.  The
suffix is used only to order the terminals in the port.  This is done
using lexicographic ordering of the suffix strings.  Beyond ordering,
the suffix is ignored.  It is a fatal error if the suffixes are the
same.}

\item{Both terminals of a port must contact the same conductor group.}
\end{enumerate} }

\item{For each terminal feature, a list of intersecting nodes is
created internally.  The first node in the list is taken as the
reference.  If there are additional nodes, they are equivalenced to
the reference node, using the {\it FastHenry} ``{\vt .equiv}''
construct.}

\item{For each port, a {\it FastHenry} ``{\vt .extern}'' construct is
used to provide the reference nodes of the two terminals in order,
followed by the port name.}
\end{enumerate}

%-----------------------------------
\subsubsection{Technology File Setup}

Setup parallels setup of the three-dimensional layer sequence database
(see \ref{ldb3d}), which in turn follows setup of the extraction
system (see \ref{extsetup}).  These sections should be consulted for
detailed setup information, here we provide some supplemental
information.

To the interface, there are two different materials:
\begin{description}
\item{conductors}\\
Conductors will have one of the following:
\begin{enumerate}
\item{Any of the {\vt Conductor}, {\vt Routing}, {\vt GroundPlane},
{\vt GroundPlaneClear}, or {\vt Contact} technology file keywords (or
their aliases) applied.  All of these implicitly give the {\et
Conductor} keyword.}

\item{Any of the {\et Rsh}, {\et Rho}, {\et Sigma}, or {\et Lambda}
keywords applied with a positive value.}
\end{enumerate}

\item{insulators}\\
Insulators will have one of the {\et Dielectric} or {\et Via} keywords
applied, and also the {\et EpsRel} keyword applied with a value of 1.0
or larger.
\end{description}

In addition, layers that are to be used in the interface as conductors
or insulators must have all of the following:

\begin{itemize}
\item{A {\et Thickness} keyword applied with a value greater than
zero.}

\item{Must be visible in the layer table.}

\item{Must not have the {\et Symbolic} keyword applied.}
\end{itemize}

The conductor layers can be given a resistivity or conductivity with
the {\et Rho} and {\et Sigma} keywords, respectively.  Additionally,
the {\et Lambda} parameter, which specifies the London penetration
depth for superconductors, can be specified.  This is for the
convenience of {\Xic} users in the superconducting electronics R\&D
community.  In this case, {\et Rho}/{\et Sigma} specify the unpaired
conductivity from the two-fluid model.

The {\et Dielectric} technology file keyword was added to support
capacitance extraction.  This is intended to model an explicit
capacitor dielectric, and differs from {\et Via} layers in the
following ways.

\begin{itemize}
\item{Unlike {\et Via}, it is not assumed to be dark field (but {\et
DarkField} can be applied to the layer explicitly).}

\item{Only one {\et Dielectric} keyword can appear per layer (multiple
{\et Via} keywords are allowed).}

\item{Stacking order is as shown in the layer table ({\et Via} layers
are allowed to appear out of order).}

\item{{\et Dielectric} layers are not planarizing by default, {\et
Via} layers are.}
\end{itemize}

The present interface can take layers in any order.  This is in
contrast with the original interface, that required layers to
alternate conductor/insulator starting with a conductor, and ordering
was obtained entirely from {\et Via} references and not the layer
table order.

After all possible layers from the layer table are sequenced, layers
that are not used in the extracted geometry are discarded.  Note that
dark-field layers are inverted, as we are interested in representing
the physical material.  Thus, for example no structure in a {\et Via}
layer (i.e., no vias) in the layout implies the presence of a
continuous film of insulating material, so the layer is actually
present.

The same layer sequencer is used in the {\cb Cross Section} command in
the {\cb View Menu}.  The cross section display and the interface will
always agree on the ordering and planarization of the layers.  This
was not true with the original interface.

In addition to the layers that describe material geometry, the
interface can make use of a masking layer.  This allows only certain
specified parts of the current cell to be evaluated.  When present,
geometry is clipped to objects on this layer before being processed in
the interface.

By default, a layer named {\vt FHRY} with purpose {\vt drawing} is
assumed for the masking layer.  Such a layer should be defined in the
technology file.  It should be given a GDSII mapping to allow saving
of work containing the layer to GDSII or OASIS files.  As an
alternative, the {\et FhLayerName} variable can be set to the name of
another layer, which will instead provide the masking function.

Finally, the layers used in terminal definitions should be configured
into the technology file.  Generally, there is an {\Xic} layer
corresponding to each conducting layer, with a purpose name ``{\vt
fhterm}'' and the same base layer name.  In order to save the layout
with terminals as GDSII or OASIS, a GDSII layer mapping should be
applied for these layers.

%-----------------------------------
\subsubsection{Output File}

The output file is a {\it FastHenry} input file, with format as
documented in the original {\it FastHenry} manual, potentially with
the extensions for superconductivity support ({\vt lambda}
specifications) if the {\et Lambda} parameter is given to any layer
used (making it a superconductor).  Such constructs are {\bf not}
compatible with the original MIT {\it FastHenry}, but require the
Whiteley Research version, or a derivative.

At the top of the file is a comment containing the layer sequence. 
For example:

\begin{verbatim}
* Layers           Plane        Thickness    EpsRel
* Substrate                     75.000       11.900
* Insulator CO     0.000        0.190        4.200
* Insulator VIA1   0.190        0.095        2.900
* Insulator VIA2   0.285        0.095        2.900
* Insulator VIA3   0.380        0.095        2.900
* Conductor M4     0.475        0.220
* Insulator VIA4   0.695        0.095        2.900
* Conductor M5     0.790        0.220
* Insulator VIA5   1.010        0.095        2.900
* Conductor M6     1.105        0.220
* Insulator VIA6   1.325        0.095        2.900
* Insulator VIA7   1.420        0.610        4.200
* Insulator VIA8   2.030        0.610        4.200
\end{verbatim}

The {\vt Plane} is the base elevation of the mask objects of a
planarizing layer, that is, the top surface of the layer minus the
layer thickness value.  This field will be empty for non-planarizing
layers.

%-----------------------------------
\subsubsection{Tips and Hints}

\begin{itemize}
\item{It is easy to generate input files that take an excessively
long time to run.  This is probably the most common pitfall.  The
run time increases with increasing segment count.  The segment  
count is approximately proportional to the number of lines in the
input file.}

\item{Non-Manhattan features can be very bad news.  These are
``Manhattanized'' by converting non-Manhattan edges into stepwise
approximations.  The large number of edges that can be produced by
this may lead te excessive run time.  The {\cb FhMinRectSize} entry in
the {\cb Params} page of the control panel can be increased to reduce
the number of segments.}

\item{Too-few segments is also to be avoided, as the calculation may
not be suitably accurate.  The {\cb Volume Element Refinement} check
box in the {\cb Params} page should always be checked.  The entered
number can be varied, the user should experiment.  Larger values
should provide better accuracy at the expense of longer computation
time.}

\item{One should not try to extract ``too much'', due to the FastHenry
limitations.  How much is too much depends on a lot a factors, the
user should experiment to gain a feel for their process and computer
hardware.}
\end{itemize}


% -----------------------------------------------------------------------------
% fhpanel 090614
\subsection{The LR Extraction Panel}
\index{LR Extraction Panel}
\label{fhpanel}
This panel, brought up by the {\cb Extract LR} button in the {\cb
Extract Menu}, controls the interface to external
inductance/resistance extraction programs described therein.  The
interface can also be controlled to a large extent with the {\cb !fh}
prompt line command.

The panel functionality is divided into three pages, selectable
through the tabs along the top of the window.  Common to all pages is
a {\cb Help} button, status line, and {\cb Dismiss} button.  The
status line indicates the number of background extraction jobs
currently running.

%-----------------------------------
\subsubsection{The Run Page}

The {\cb Run} page contains controls for running the supported
programs, or creating input files for these programs.  This is the
default page, shown when the panel appears.

\begin{description}
\item{\cb Run in foreground}\\
At the top of the page is the {\cb Run in foreground} check box.  When
checked, the program will run synchronously in the foreground, rather
than asynchronously in the background.  Aside from possibly being
helpful when debugging problems, it is not clear that this mode has
any value.

This check box sets, and is set by, the {\et FhForeg} variable.

\item{\cb Out to console}\\
When the {\cb Out to console} check box is checked, the program output
will be printed in the console window, i.e., the shell window from
which {\Xic} is running.  It may be useful to verify that all is well
by watching this output.

This check box sets, and is set by, the {\et FhMonitor} variable.

\item{\cb Run File}\\
This button and adjacent text entry allows a compatible input file
to be run by the inductance/resistance extraction program
currently configured.  The text area should contain a path to a
valid input file for the configured program.  The program will
run, and results will appear, as for a normal extraction run.

\item{\cb Run Extraction}\\
This button will dump a temporary input file, run the program, and
display the results.  The result file is named {\it cellname\/}{\vt
-}{\it pid\/}{\vt .fh\_log}, where {\it cellname} is the name of the
current cell, and {\it pid} is the process id of the spawned process
used to run the program.  The file contains listings of the input file
produced by the interface and the output file produced by the program.

By default, the program is run in the background.  The label at the
bottom of the panel will indicate that the job is running.  When
complete, a {\cb File Browser} window containing the result file will
appear.  While jobs are running in the background, one can continue
using {\Xic}.

If the {\et FhForeg} variable is set, from the {\cb Run in foreground}
check box or with the {\cb !set} command, then the program will
instead run in the foreground.  In this case, the result file is named
{\it cellname\/}{\vt .fh\_log}, and {\Xic} will be unresponsive until
the run completes.

\item{\cb Dump FastHenry File}\\
This button allows an input file to be generated, which is in a format
compatible with the {\it FastHenry} program.  The default name for
this file is {\it cellname\/}{\vt .inp}, where {\it cellname} is the
current cell name.

\item{\cb FhArgs}\\
This text entry area can be given a string, which will be included in
the argument list when the program is run with the {\cb Run
Extraction} button.  This allows specialized command line options to
be provided during the run, which the user may require.  This entry
field is tied to the {\et FhArgs} variable.

\item{\cb FhFreq}\\
This consists of three entry areas, which take the starting and ending
evaluation frequencies for {\it FastHenry} runs, and the number of
intermediate frequencies to evaluate.  This corresponds to the {\vt
.Freq} specification line in {\it FastHenry} input files.  The
frequencies are given in hertz.  If the third field in empty, then
evaluation is at the specified frequencies only.  This variable is
tied to the {\et FhFreq} variable, which can also be set with the {\cb
!set} command.

\item{\cb Path to FastHenry}\\
Near the bottom of the page is an entry area where the path to the
{\it FastHenry} executable program can be edited.  This entry area
displays and sets the {\et FhPath} variable.
\end{description}

%-----------------------------------
\subsubsection{The Params page}

The upper half of this page provides entry areas for parameters used
by the interface.

\begin{description}
\item{\cb FhUnits}\\
This is an option menu which is used to set the length units used in
files produced by the interface.  Choices are meters, centimeters,
millimeters, microns (the default), inches, and mils.  The selection,
if not the default, will set the {\et FhUnits} variable.  Similarly,
setting the variable with the {\cb !set} command will update the state
of the menu.  The choice currently in effect will be applied when
input files are generated.  The choice of units will not affect the
computed inductance/resistance.

\item{\cb FhMinRectSize}\\
This numerical entry field is used to specify the minimum rectangle
width or height used when non-Manhattan objects are converted to a
Manhattan approximation for {\it FastHenry}.  The value is entered in
microns, and the default value is 1.0.  This is tied to the {\et
FhMinRectSize} variable, which can also be set directly with the {\cb
!set} command.
\end{description}

The lower half of the page allows one to crudely refine the raw
segmentation.  By default, this is not enabled, so that only tiling
provides refinement.  This may be a good starting point for a
third-party refinement algorithm, but with present {\it FastHenry}
programs is unlikely to provide accurate results as-is.  When enabled,
the crude refinement should provide somewhat better results.

The refinement algorithm works as follows.  First, the total volume of
all conductors is calculated.  This is divided by the {\cb
FhVolElTarget} as entered.  The cube root is taken, which provides a
length which is used as a maximum.  If a side of a tile exceeds this
length, it is subdivided.  This is repeated until no tiles have sides
larger than the calculated length.  The total number of tiles (or
``volume elements'') is approximately the target value entgered.  The
total number of segments is approximately six times larger.

The refinement is ``crude'' due to each refined volume element being
approximately the same size.  If the size is small enough, sufficient
spatial resolution for accurate calculation is achieved.  This
resolution is needed along edges, and at corners, where there are
strong field gradients, but is gross overkill for most areas.  Since
the solving time is related to the total number of segments, this type
of refinement is very inefficient with respect to memory use and
execution speed.

\begin{description}
\item{\cb Enable}\\
This check box will enable or disable the refinement.  When pressed,
the {\cb FhVolElTarget} entry will become un-grayed, and internally
the {\et FhVolElTarget} variable will be set to the number shown
in the {\cb FhVolElTarget} entry area.  When the {\cb Enable} button
is set inactive, the {\et FhVolElTarget} variable is unset.  The {\cb
Enable} button state will reflect whether or not the {\et
FhVolElTarget} variable is set.

\item{\cb FhVolElTarget}\\
This entry area is seensitive only when the {\cb Enable} check box is
checked.  It tracks the value of the {\et FhVolElTarget} variable,
which can be set to a real value of 1e2 -- 1e5.  This will be the
approximate number of refined tiles generated in the input file.
\end{description}

%-----------------------------------
\subsubsection{The Jobs page}

The {\cb Jobs} page contains a list of running background jobs.  Each
entry provides the process identification number (PID), the name of
the executing program, and the local date and time when started. 
Entries can be selected by clicking with the mouse.

When an entry is selected, the {\cb Abort job} button below the list
becomes un-grayed.  Clicking this button will terminate the selected
process.  The user should consider that there is no confirmation and
no ability to resume the run.

