% -----------------------------------------------------------------------------
% Xic Manual
% (C) Copyright 2009, Whiteley Research Inc., Sunnyvale CA
% $Id: modifymenu.tex,v 1.14 2015/12/06 23:39:16 stevew Exp $
% -----------------------------------------------------------------------------

% -----------------------------------------------------------------------------
% xic:modifymenu 020113
\chapter{The Modify Menu:  Modify Geometry}
\index{Modify Menu}
The {\cb Modify Menu} contains commands which alter the current
design, supplemental to the side menu.  Most of these commands have
keyboard or mouse motion shortcuts, so an experienced user may not
often use this menu.

The table below summarizes the commands that appear in the {\cb Modify
Menu}, including the internal command name and the command function.

\begin{tabular}{|l|l|l|l|} \hline
\multicolumn{4}{|c|}{\kb Modify Menu}\\ \hline
\kb Label & \kb Name & \kb Pop-up & \kb Function\\ \hline\hline
\et Undo & \vt undo & none & Undo last operation\\ \hline
\et Redo & \vt redo & none & Redo last undo\\ \hline
\et Delete & \vt delet & none & Delete objects\\ \hline
\et Erase Under & \vt eundr & none & Erase under objects\\ \hline
\et Move & \vt move & none & Move objects\\ \hline
\et Copy & \vt copy & none & Copy objects\\ \hline
\et Stretch & \vt strch & none & Stretch objects\\ \hline
\et Change Layer & \vt chlyr & none & Move object to new layer\\ \hline
\et Set Layer Chg Mode & \vt mclcg & {\cb Layer Change Mode} &
 Set layer change mode for move/copy\\ \hline
\end{tabular}


% -----------------------------------------------------------------------------
% xic:undo 061608
\section{The {\cb Undo} Button: Undo Operation}
\index{Undo button}
\index{undo}
The {\cb Undo} button in the {\cb Modify Menu} reverses the operations
performed in the current cell.  These operations can be undone as long
as the present cell is the current cell.  Undone operations can be
redone with the {\cb Redo} command.  Pressing the {\kb Tab} key has
the same effect as clicking on the {\cb Undo} button, and {\kb
Shift-Tab} is equivalent to {\cb Redo}.

\index{undo!list length}
By default, the last 25 operations can be undone.  This can be changed
with the variable {\et UndoListLength}, which can be set to a
non-negative integer with the {\cb !set} command.  This sets the
number of operations that are remembered.  If set to zero, the list
length is unlimited.

When {\it Xic} is waiting for text input to the prompt line, the {\cb
Undo} and {\cb Redo} commands are disabled.


% -----------------------------------------------------------------------------
% xic:redo 061608
\section{The {\cb Redo} Button: Redo Last Undo}
\index{Redo button}
The {\cb Redo} button in the {\cb Modify Menu} will redo the last
undone operation performed with {\cb Undo}.  This can also be
accomplished by holding the {\kb Shift} key and pressing the {\kb Tab}
key.  Each undone operation is added to an internal list for possible
redo.  This list is cleared after any database-modifying operation
which is not an undo.

When {\Xic} is waiting for text input to the prompt line, the {\cb
Undo} and {\cb Redo} commands are disabled.


% -----------------------------------------------------------------------------
% xic:delet 061608
\section{The {\cb Delete} Button: Delete Objects}
\index{Delete button}
\index{object deletion}
The {\cb Delete} button in the {\cb Modify Menu} may be used to delete
the selected objects.  This is redundant, as selected objects can be
deleted by pressing the {\kb Delete} key.


% -----------------------------------------------------------------------------
% xic:eundr 100412
\section{The {\cb Erase Under} Button: Erase Under Objects}
\index{Erase Under button}
The {\cb Erase Under} button in the {\cb Modify Menu} will erase the
intersection area of non-selected objects with selected objects.  The
selected objects are not affected.  This allows non-Manhattan holes to
be cut in dark areas, for example.  Suppose one needs a circular hole
in a ground plane.  Using this command, the task is simple.  One would
create the disk on some arbitrary layer where the hole is desired,
select it, press {\cb Erase Under}, then the {\kb Delete} key to erase
the disk object.


% -----------------------------------------------------------------------------
% xic:move 082024
\section{The {\cb Move} Button: Move Objects}
\index{Move button}
\index{move objects}
\index{object movement}
The {\cb Move} button in the {\cb Modify Menu} is used to move
objects.  This command is redundant, as objects can be moved with a
basic button 1 operation.  If objects are previously selected, the
group will be moved.  If no object has been selected, the user is
requested to select an object to move.  Responding to the prompts, the
user points to a reference point, then to a destination point, using
either hold and drag, or two clicks.  If either the {\kb Shift} or
{\kb Ctrl} key is held, the angle of translation is constrained to
multiples of 45 degrees.  The object is moved such that the reference
point falls on the destination point.  The orientation is altered
according to the current transformation.

When the {\cb Move} command is at the state where objects are
selected, and the next button press would initiate the move operation,
if either of the {\kb Backspace} or {\kb Delete} keys is pressed, the
command will revert the state back to selecting objects.  Then, other
objects can be selected or selected objects deselected, and the
command is ready to go again.  This can be repeated, to build up the
set of selections needed.

At any time, pressing the {\cb Deselect} button to the left of the
coordinate readout will revert the command state to the level where
objects may be selected to move.

The current transform is saved in Register 0 then cleared after every
move operation.  Pressing the {\kb Forward Slash} key will swap the
current and saved transforms, allowing the last-used transform to be
retrieved.  The current transform is saved and cleared when
{\cb Deselect} is pressed, and cleared when the {\cb Move} command
exits.

The undo and redo operations (the {\kb Tab} and {\kb Shift-Tab}
keypreses and {\cb Undo}/{\cb Redo} in the {\cb Modify Menu}) will
cycle the command state forward and backward when the command is
active.  Thus, the last command operation, such as initiating the
move by clicking, can be undone and restarted, or redone if
necessary.  If all command operations are undone, additional undo
operations will undo previous commands, as when the undo operation is
performed outside of a command.  The redo operation will reverse the
effect, however when any new modifying operation is started, the redo
list is cleared.  Thus, for example, if one undoes a box creation,
then starts a move operation, the ``redo'' capability of the box
creation will be lost.

The substructure of cell instances being moved is highlighted to the
depth shown in the main window.  This facilitates alignment with other
objects.  One can change the display depth to reveal more or less of
the substructure.

While in a move operation in physical mode, while the objects are
ghost-drawn and attached to the pointer, pressing {\kb Enter} causes
the reference point to shift to the lower left corner of the bounding
box containing the objects being moved.  Pressing {\kb Enter} will
cycle the reference point through the corners of the bounding box, and
back to the original reference location.  Note that this allows
objects that have somehow gotten off grid to be returned to the grid.

It is possible to change the layer of objects during a move operation. 
During the time that objects are ghost drawn and attached to the mouse
pointer, if the current layer is changed, the objects that are
attached can be placed on the new layer.  Subcells are not affected. 

How this is applied depends on the setting of the {\et
LayerChangeMode} variable, or equivalently the settings of the {\cb
Layer Change Mode} pop-up from the {\cb Set Layer Chg Mode} button
in the {\cb Modify Menu}.  The three possible modes are to ignore the
layer change, to map objects on the old current layer to the new
current layer, or to place all objects on the new current layer.  If
the current layer is set back to the previous layer before clicking to
locate the new objects, no layers will change.  Note that layer change
is only possible for ``click-click'' mode and not ``press-drag''.

Move operations can be also performed through the command line
interface with the {\cb !mo} command.


% -----------------------------------------------------------------------------
% xic:copy 082024
\section{The {\cb Copy} Button: Copy Objects}
\index{Copy button}
\index{object copying}
\index{copy objects}
The {\cb Copy} button in the {\cb Modify Menu} is used to copy
objects.  In its simplest form, this command is redundant, as copies
of an object can be made with basic button 1 operations.  However,
this command has an important and useful feature not available with
the basic mouse operations:  it is possible to copy objects from cells
other than the one being edited.

Initially, the user is prompted for a replication count.  This can be
any positive integer.  When the copy is performed, the replication
specifies the number of copies made, with the translation incremented
for each new copy.  Thus, this facilitates creating many
equally-spaced structures.

If objects are previously selected, the group will be copied to new
locations.  If no objects have been selected, the user is asked to
select objects to copy.

Responding to the prompts, the user first clicks on a reference point,
then to a destination, using a hold and drag, or two clicks.  If
either the {\kb Shift} or {\kb Ctrl} key is held, the angle of
translation is constrained to multiples of 45 degrees.  The copy is
produced such that the reference point falls on the destination point. 
The orientation of the copied object is altered according to the
current transformation.  Multiple copies are made by simply clicking
on additional destinations.

When the {\cb Copy} command is at the state where objects are
selected, and the next button press would initiate the copy operation,
if either of the {\kb Backspace} or {\kb Delete} keys is pressed, the
command will revert the state back to selecting objects.  Then, other
objects can be selected or selected objects deselected, and the
command is ready to go again.  This can be repeated, to build up the
set of selections needed.

Ordinarily, if a sub-window is displaying a cell other than the
current cell being edited, it is not possible to select objects in
that sub-window.  However, while the {\cb Copy} command is active, if
the sub-window has the same electrical/physical mode as the current
cell, selections are allowed in the foreign sub-window.

Selections in a foreign window can be ``picked up'' just like objects
selected in the main window.  Outlines of the selected objects will be
attached to the mouse pointer.  They can be copied into the main
window or a sub-window displaying the current editing cell by dragging
or clicking twice.

Objects can be selected in various sub-windows and the main window
simultaneously.  Selections in sub-windows showing the current cell
are in all respects equivalent to the main window.  Use of the {\kb
Backspace} or {\kb Delete} key method above is necessary to obtain
selections in both the main window (and equivalent sub-windows), and
sub-windows showing other cells.  When the copy in initiated, only
the objects from the cell in the clicked-in window (when objects are
picked up) will participate in the copy operation.

Once objects have been picked up, whether copies have been placed or
not, pressing either of the {\kb Backspace} or {\kb Delete} keys will
revert the command state to the level before the objects were picked
up.  The user can then click in another window to pick up that
window's selected objects, or in the same window to pick up the
previous objects but with a different reference location.

At any time, pressing the {\cb Deselect} button to the left of the
coordinate readout will revert the command state to the level where
objects may be selected to copy (in any mode-compatible window).

The current transform is saved in Register 0 after each copy is placed.
Pressing the {\cb Forward Slash} key will swap the current and saved
transforms, allowing the last-used transform to be retrieved.  The
current transform is saved and cleared when {\cb Deselect} is pressed,
and cleared when the {\cb Copy} command exits.

The undo and redo operations (the {\kb Tab} and {\kb Shift-Tab}
keypreses and {\cb Undo}/{\cb Redo} in the {\cb Modify Menu}) will
cycle the command state forward and backward when the command is
active.  Thus, the last command operation, such as initiating the
copy by clicking, can be undone and restarted, or redone if
necessary.  If all command operations are undone, additional undo
operations will undo previous commands, as when the undo operation is
performed outside of a command.  The redo operation will reverse the
effect, however when any new modifying operation is started, the redo
list is cleared.  Thus, for example, if one undoes a box creation,
then starts a copy operation, the ``redo'' capability of the box
creation will be lost.

The substructure of cell instances being copied is highlighted to the
depth shown in the main window.  This facilitates alignment with other
objects.  One can change the display depth to reveal more or less of
the substructure.

The replication count feature is not available when copying objects
from a foreign window, since the reference point is from another cell
and is unlikely to be valid in the current cell.  One copy of each
selected object is created, at the click location or where dragging
terminated, ignoring the replication count.

While in a copy operation in physical mode, while the objects are
ghost-drawn and attached to the pointer, pressing {\kb Enter} causes
the reference point to shift to the lower left corner of the bounding
box containing the objects being copied.  Pressing {\kb Enter} will
cycle the reference point through the corners of the bounding box, and
back to the original reference location.

It is possible to change the layer of objects during a copy operation. 
During the time that objects are ghost drawn and attached to the mouse
pointer, if the current layer is changed, the objects that are
attached can be placed on the new layer.  Subcells are not affected. 

How this is applied depends on the setting of the {\et
LayerChangeMode} variable, or equivalently the settings of the {\cb
Layer Change Mode} pop-up from the {\cb Set Layer Chg Mode} button
in the {\cb Modify Menu}.  The three possible modes are to ignore the
layer change, to map objects on the old current layer to the new
current layer, or to place all objects on the new current layer.  If
the current layer is set back to the previous layer before clicking to
locate the new objects, no layers will change.  Note that layer change
is only possible for ``click-click'' mode and not ``press-drag''.

When the {\cb Copy} command terminates, any selected objects in
foreign sub-windows are deselected.

Copy operations can be also performed through the command line
interface with the {\cb !co} command.

{\bf Example}:\\
Suppose that you are editing cell B, and you would like to add a     
set of complicated polygons that you have already created in cell A.

\begin{enumerate}
\item{Use the {\cb Viewport} command in the {\cb View Menu} to bring up
a sub-window.}
\item{Use the {\cb Load New} command in the sub-window {\cb View} menu
to display cell A.}
\item{Deselect any selected objects and instances.}
\item{Press the {\cb Copy} button in the main window {\cb Modify Menu},
and press {\kb Enter} at the ``Replication count'' prompt.}
\item{Select the desired objects in the sub-window.}
\item{Click on a selected object in the sub-window, and drag or click
again to copy the selected objects into the main window.}
\end{enumerate}

% -----------------------------------------------------------------------------
% xic:strch 012815
\section{The {\cb Stretch} Button: Stretch Objects}
\index{Stretch button}
\index{object stretching}
The {\cb Stretch} button in the {\cb Modify Menu} operates on
polygons, wires, boxes, and labels.  It enables moving of polygon and
wire vertices, and box and label bounding box corners and sides.  This
command is somewhat redundant, as stretching operations can be
initiated with basic button 1 manipulation, however the ability to
select specific vertices to stretch is available only in the menu
version of the command.

If no geometry has been selected, the user is asked to select objects
to stretch.  Otherwise, the stretch will be applied to currently
selected objects.

After objects have been selected, specific vertices can be selected in
boxes, polygons, and wires.  The selection of vertices, which is
available only in the menu version of the command, is accomplished by
holding the {\kb Shift} key, and clicking over a vertex, or dragging
over one or more vertices.  This operation can be repeated.  Selecting
a vertex a second time will deselect it.  When a vertex is selected it
is marked with a small highlighting box.  When there are selected
vertices, all selected vertices can be moved by clicking twice or
dragging.  The selected vertices will be translated according to the
button-down location and the button up location, or the next
button-down location if the pointer didn't move.  While the
translation is in progress, the new borders are ghost-drawn.  While
moving vertices, holding the {\kb Shift} key will enable or disable
constraining the translation angle to multiples of 45 degrees.  If the
{\cb Constrain angles to 45 degree multiples} check box in the {\cb
Editing Setup} panel from the {\cb Edit Menu} is checked, {\kb Shift}
will disable the constraint, otherwise the constraint will be enabled. 
The {\kb Shift} key must be up when the button-down occurs which
starts the translation operation, and can be pressed before the
operation is completed to alter the constraint.

If a box is selected in the {\cb Stretch} command, and one or more
vertices of the box are selected by holding {\kb Shift}, the vertices
can be moved as for a polygon, and the box is converted to a polygon.

If no vertices are selected, the stretch operation applies to the
nearest vertex of selected wires or polygons, or the nearest corner of
a box.  In this mode, boxes are stretched in a mode which preserves
their rectangular shape.  The user clicks on or drags to the new
location, and the stretch is performed.  If there are several objects
selected, then the vertex closest to where the user points is taken as
the reference vertex.  This vertex is translated to the new location. 
In each of the other objects, the same transformation is applied to
the vertex closest to the reference vertex.  Thus, a group of wires,
for example, can all be extended at once.  During the operation, the
{\kb Shift} key and the {\cb Constrain angles to 45 degree multiples}
check box in the {\cb Editing Setup} panel can be used to constrain
the stretch angle as described above.

When the {\cb Stretch} command is at the state where objects are
selected, and the next button press would initiate the stretch
operation or select a vertex, if either of the {\kb Backspace} or {\kb
Delete} keys is pressed, the command will revert the state back to
selecting objects.  Then, other objects can be selected or selected
objects deselected, and the command is ready to go again.  This can be
repeated, to build up the set of selections needed. 

At any time, pressing the {\cb Deselect} button to the left of the
coordinate readout will revert the command state to the level where
objects may be selected to stretch.

The undo and redo operations (the {\kb Tab} and {\kb Shift-Tab}
keypreses and {\cb Undo}/{\cb Redo} in the {\cb Modify Menu}) will
cycle the command state forward and backward when the command is
active.  Thus, the last command operation, such as initiating the
stretch by clicking, can be undone and restarted, or redone if
necessary.  If all command operations are undone, additional undo
operations will undo previous commands, as when the undo operation is
performed outside of a command.  The redo operation will reverse the
effect, however when any new modifying operation is started, the redo
list is cleared.  Thus, for example, if one undoes a box creation,
then starts a stretch operation, the ``redo'' capability of the box
creation will be lost.

The stretch operation works differently on Manhattan polygons than
polygons containing nonorthogonal angles.  For non-Manhattan polygons,
a single vertex is moved, all others remain fixed.  The stretch
operation on Manhattan polygons is similar to the operation as applied
to boxes, i.e., the corner and adjacent vertices are changed so as to
keep the polygon Manhattan.  A single vertex can be stretched
arbitrarily either by selecting the vertex in the {\cb Edit Menu} {\cb
Stretch} command, or by using the vertex editor in the {\cb polyg}
command.

If a wire end vertex is stretched to be coincident with the end vertex
of another wire on the same layer with the same width, the wires will
be merged, but only if the second wire is not selected.


% -----------------------------------------------------------------------------
% xic:chlyr 061608
\section{The {\cb Change Layer} Button: Change Layer}
\index{Change Layer button}
\index{layer change of object}
The {\cb Change Layer} button in the {\cb Modify Menu} allows the user to
change the layer of the selected objects.  All selected objects will
be moved to the current layer.  Objects must be selected before this
command button is pressed.


% -----------------------------------------------------------------------------
% xic:mclcg 101212
\section{The {\cb Set Layer Chg Mode} Button: Set Change Mode for Move/Copy}
\index{Set Layer Chg Mode button}
\index{layer change of object}
This button brings up a panel which sets the layer change mode that
applies to all move and copy operations, and to the {\cb spin} command
in the physical side menu.  In these commands, when objects being
moved or copied are ghost drawn as attached to the mouse pointer, it
is possible to change the current layer.  The operation is then
completed by clicking at the new location in a drawing window.
 
The {\cb Layer Change Mode} pop-up contains three ``radio'' buttons,
which determine the response to the mid-command layer change.

\begin{description}
\item{\cb Don't allow layer change}\\
The layer change is simply ignored as it relates to the move/copy
operation in progress.  This is the default.

\item{\cb Allow layer change for objects on current layer}\\
Any of the objects being moved/copied that are on the previous current
layer will be moved or copied to the new current layer.  Other objects
are moved/copied normally.

\item{\cb Allow layer change for all objects}\\
All objects will be moved or copied to the new layer.
\end{description}

The pop-up sets and tracks the state of the {\et LayerChangeMode}
variable.

This is a tri-state variable.  If not set, there will be no layer
change in these commands.  If set to the string ``{\vt all}'' (case
insensitive), then a layer change will apply to all objects being
moved or copied.  If set to anything else, including to nothing (i.e.,
as a boolean) then only objects on the previous current layer will be
changed to the new layer.

