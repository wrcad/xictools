% -----------------------------------------------------------------------------
% Xic Manual
% (C) Copyright 2009, Whiteley Research Inc., Sunnyvale CA
% $Id: attrimenu.tex,v 1.67 2016/10/10 18:29:59 stevew Exp $
% -----------------------------------------------------------------------------

% -----------------------------------------------------------------------------
% xic:attrmenu 012715
\chapter{The Attributes Menu:  Set Display Attributes}
\index{Attributes Menu}
The {\cb Attributes Menu} contains commands primarily for modifying
the presentation format of {\Xic}.  Each sub-window has an independent
set of attributes, which are initially set to those of the main
window.

The table below summarizes the command buttons found in the {\cb
Attributes Menu}, listing the internal name and command function.

\hspace*{-1.5em}
\begin{tabular}{|l|l|l|l|} \hline
\multicolumn{4}{|c|}{\kb Attributes Menu}\\ \hline
\kb Label & \kb Name & \kb Pop-up & \kb Function\\ \hline\hline
\et Save Tech & \vt updat & none & Save technology file\\ \hline
\et Key Map & \vt keymp & none & Create keyboard mapping file\\ \hline
\et Define Macro & \vt macro & none & Define a keyboard macro\\ \hline
\et Main Window & & Attributes sub-menu & Set main window attributes\\ \hline
\et Set Attributes & \vt attr & \cb Window Attributes &
    Set misc. attributes for drawing windows\\ \hline
\et Connection Dots & \vt dots & \cb Connection Points &
    Show connection dots in schematics\\ \hline
\et Set Font & \vt font & \cb Font Selection & Set text fonts used\\ \hline
\et Set Color & \vt color & \cb Color Selection & Set layer and other colors\\
    \hline
\et Set Fill & \vt fill & \cb Fill Pattern Editor & Set layer fill patterns\\
    \hline
\et Edit Layers & \vt edlyr & \cb Layer Editor & Add or remove layers\\ \hline
\et Edit Tech Params & \vt lpedt & \cb Tech Parameter Editor & Edit
    technology parameters\\ \hline
\end{tabular}

The {\cb Connection Dots} button appears only when the full {\Xic}
feature set is enabled, not in the {\XicII} or {\Xiv} virtual
products.

The sub-menu brought up by the {\cb Main Window} button is identical
to the {\cb Attributes} menu in the sub-windows produced from the {\cb
Viewport} button in the {\cb View Menu}.  The settings apply to the
main window.

\begin{tabular}{|l|l|l|p{2.25in}|} \hline
\multicolumn{4}{|c|}{\kb Sub-Window Attributes Menu}\\ \hline
\kb Label & \kb Name & \kb Pop-up & \kb Function\\ \hline\hline
\et Freeze Display & \vt freez & none & Suppress redisplay\\ \hline
\et Show Context in Push & \vt cntxt & none & Show context in subedit\\ \hline
\et Show Phys Properties & \vt props & none & Show physical properties\\ \hline
\et Show Labels & \vt labls & none & Show labels\\ \hline
\et Label True Orient & \vt larot & none & Show labels transformed\\ \hline
\et Show Cell Names & \vt cnams & none & Show cell names\\ \hline
\et Cell Name true Orient & \vt cnrot & none & Show cell names transformed\\
  \hline
\et Don't Show Unexpanded & \vt nouxp & none & Don't show unexpanded
  subcells\\ \hline
\et Objects Shown & \vt objs & none & Object display control\\ \hline
\et Subthreshold Boxes & \vt tinyb & none & Show outline of subthreshold
  cells\\ \hline
\et No Top Symbolic & \vt nosym & none & Electrical only, don't show top
  cell as symbolic\\ \hline
\et Set Grid & \vt grid & {\cb Grid Setup} & Set grid parameters\\ \hline
\end{tabular}


% -----------------------------------------------------------------------------
% xic:updat 071010
\section{The {\cb Save Tech} Button: Update Technology File}
\index{Save Tech button}
\index{updating technology file}
The {\cb Save Tech} button in the {\cb Attributes menu} will pop up a
small panel, from which one may write an updated technology file to
disk.  This file provides setup information to {\Xic}, and is read on
program start-up.  The file will reflect the current settings of the
configurable attributes, layers, etc.

The {\cb Write Tech File} panel has three radio buttons that select
how parameters that are currently set to program defaults will be
handled.  By default, keywords which would specify a default value are
omitted from the file, providing a more compact file.  However, it may
be useful to include the defaults in the file, to facilitate future
hand-editing.  If {\cb Comment default definitions} is selected, these
lines will be added to the technology file as comments.  If {\cb
Include default definitions} is selected, these lines will be added as
active text.

An entry area allows the user to change the name of the file to write. 
The default is to use ``{\vt xic\_tech}'' with an extension, if any,
that was given to {\Xic} on startup with the {\vt -T} option, written
in the current directory.  The base name must be ``{\vt xic\_tech}''
and the file must be located in the library search path in order to
load the new file when {\Xic} is restarted.

The radio buttons track and set the status of the
{\et TechPrintDefaults} variable.   


% -----------------------------------------------------------------------------
% xic:keymp 011014
\section{The {\cb Key Map} Button: Create Key Mapping File}
\index{Key Map button}
\index{key mapping}
Several of the keys which {\Xic} uses are not found on all keyboards,
or they may return a different code than {\Xic} expects.  {\Xic}
contains a built-in facility for remapping keys from the keyboard to
the functions expected in {\Xic}, through the {\cb Key Map} button in
the {\cb Attributes Menu}.  The user is prompted to press keys on the
keyboard that will correspond to the various special keys such as {\kb
Page Up}, {\kb Page Down}, {\kb Home}, and numeric keypad {\kb Add}
and {\kb Subtract}.  Only special function keys can be remapped, not
the standard character entry keys, or modifier keys (e.g., {\kb
Shift}, {\kb Control}) and not the function keys 1--12.  If there is
no response when a key is pressed and the pointer is in a drawing
window, then that key can not be mapped.  To skip mapping of a key,
press {\kb Enter}.  To abort, press {\kb Esc}.  When the prompting is
finished, a file will be created in the current directory named {\vt
xic\_keymap} by default.  The user is given the option to alter this
name.

The file can be moved to a safe place if desired.  The keyboard
mapping is not actually performed until this file is read by {\Xic},
which must be done explicitly.  This differs from {\Xic} releases
earlier than 4.0.0, which attempted to automatically read a created
mapping file.  There are two ways to accomplish reading the mapping
file:

\begin{enumerate}
\item{The {\cb !kmap} command:  {\vt !kmap} {\it mapfile}\\
This command can be typed into the prompt line and will cause the
named file to be read and the mapping asserted immediately.}

\item{The {\vt ReadKeymap} script function:
  {\vt ReadKeymap(}{\it mapfile\/}{\vt )}\\
The function can be run from any script.  In particular, this line can
be added to a startup script ({\vt .xicstart} or {\vt .xicinit} file)
to assert the mapping when {\Xic} starts.}
\end{enumerate}

In either case, the {\it mapfile} is a path to a file as created with
this command.  If the path is not rooted, the file is searched for
relative to the current directory, the user's home directory, and
through the library search path, in that order.  The commands can be
re-run with different mapping files at any time, the current mapping
state reflects the most recent mapping applied.

\subsection{Key Mapping File}
\index{xic\_keymap file}

The file produced in this manner contains all key mapping and action
translation tables used by {\Xic}.  Although it is not really
recommended, this file can be customized by the user, with a text
editor.  The recommended way to alter a key sequence response is with
a macro (see below), but there may be occasions where the mapping file
should be changed to achieve a desired effect.  Contact Whiteley
Research for assistance.

The name of the file is {\vt xic\_keymap.}{\it hostname}, where {\it
hostname} is the name of the machine.  When {\Xic} starts, it looks
for a file of this name in the current directory, then in the user's
home directory, then in the library search path.  If a file is found,
it is read and processed.

The first section of the mapping file contains a listing of the
keysyms for the special keys mapped with this command ({\kb Home},
{\kb Page Down}, etc.).  The keysym is a system-defined number
assigned to that key.  In Unix/Linux, keysyms are defined in the X
include file {\vt X11/keysymdef.h}.

The next section of the file consists of the definition of the ``macro
suppression'' character (described below).  This must be a printable
ASCII character, surrounded by single quotes.

The third section contains a mapping table for keysyms to an internal
code.  The first column contains the keysyms; a numeric value in
Windows, or the standard name under Unix/Linux.  The second column is
the internal code.  The third column is a subcode used only for
function keys.

The next section contains an action table which maps actions before
the keypress code is sent to any internal command in {\Xic}.  In
action tables, the first column is a code which is either an internal
code, or the ASCII value of the keysym.  This is operating system
dependent.  Under Unix/Linux, the keysym code for a printing character
is simply the ASCII value of that character.  The internal code is
interpreted numerically as a value (hex) 0 -- 1f.  Ascii values are
(hex) 20 -- fe.  Other values are taken as keysyms.  Under Windows, the
code is similar, but the Windows keysym is used.  The Windows keysym
differs in that (1) the upper-case alpha characters must be specified
rather than lower case, (2) for punctuation which is the Shift of a
numeric key, the numeric key should be given, (3) for punctuation
which is not a Shift value for a numeric key, a special keysym for
that key should be given.

The second column is the modifier state.  The actions should be listed
in order of most specific to least specific with regard to modifiers. 
A value ``0'' in this column indicates ``don't care''.  Otherwise, the
entry consists of one or more of ``SHIFT'', ``CTRL'', and ``ALT'',
separated by a minus sign `{\vt -}'.  The actual modifier should be
listed, even if a Shift state is implied by the value in column one.

For example, to select the `!' character, in Unix/Linux one has
\begin{quote}
\begin{verbatim}
'!'        SHIFT
\end{verbatim}
\end{quote}
in Windows, this would be
\begin{quote}
\begin{verbatim}
'1'        SHIFT
\end{verbatim}
\end{quote}

The third column in the action tables is the name of an action, which
is defined internally in {\Xic}.  This is the action performed when
the keypress combination specified in columns one and two is detected.

The next section in the file is an action table that specifies actions
to perform after the present {\Xic} command processes the keypress. 
Many of the {\Xic} commands look for specific keys, and if that key is
seen, further mapping is inhibited.  If the keypress is not used by
the command, it is available for translation by this final table.

The last section of the file contains a ``{\vt $<$Buttons$>$}'' field
which maps the functions of the mouse buttons under Unix/Linux.  This
allows the functions of the buttons within {\Xic} to be permuted. 
However, the functions of the buttons with respect to the user
interface, such as the mouse button used to engage user interface
buttons and menu items, will not change.  This field is ignored under
Windows.


% -----------------------------------------------------------------------------
% xic:macro 011114
\section{The {\cb Define Macro} Button: Assign a Macro to a Key}
\index{Define Macro button}
\index{macros}
Pressing the {\cb Define Macro} button in the {\cb Attributes Menu}
allows the user to enter a macro.  The generated macros are stored in
a file named {\vt .xicmacros} or {\vt .xicmacros}.{\it ext}, where
{\it ext} is the current technology file extension.  In all cases,
when a macro file is produced, which occurs after any new macro is
defined, the file is written in the current directory as {\vt
.xicmacros}.  The new file contains definitions for all current
macros.  The user can add the suffix and move the file to their home
directory if desired.

When {\Xic} starts, it looks for a macro file in the following
sequence, and inputs the first one (and only one) found:\\
\begin{tabular}{ll}\\
1. &  current directory using same extension as tech file.\\
2. &  home directory using same extension as tech file.\\
3. &  current directory with no extension.\\
4. &  home directory with no extension.
\end{tabular}

Under Windows, the home directory is obtained from environment
variables, in particular the value of {\et HOME}.

Macros can be attached to any combination of a keyboard key and
modifier key(s), with the exceptions of {\kb Enter}, {\kb Esc}, {\kb
Backspace}, and the ``bare'' modifier keys ({\kb Shift}, {\kb Ctrl},
etc).  However, not all key combinations will work.  For example, the
expansion of keyboard menu accelerators occurs before macro expansion,
so menu accelerator key combinations can not be used as macros.  Also,
key combinations that are intercepted by the window manager, which may
include some combinations involving the {\kb Alt} key, can not be used
as macros.  Otherwise, macro definitions have higher priority than
most other functions in {\Xic}.

Macros are {\it not} expanded when the prompt line is in text editing
mode.  The macro expansion can be suppressed for the next key
combination by pressing the macro suppression character first.  The
macro suppression character is the backquote ({\vt `}).  The macro
suppression character is eliminated from the keypress buffer after the
next key is typed.  For example, suppose `{\vt o}' is mapped to
something, but you want to enter a literal `{\vt o}' to trigger the
{\cb Open} command (``{\vt op}'').  One would type ``{\vt `op}'',
after which the keypress buffer would contain ``{\vt op}'' triggering
the {\cb Open} command.

While recording a macro, button and key presses will not have the
normal effect, but the events are stored in the macro and will have
the normal effect when the macro is invoked.  Button presses will open
a menu in the normal way, and selected menu commands will become
active, but subsequent events will be swallowed by the macro recorder. 
In most cases, one can send events to a pop-up by performing the
actions, which won't be carried out but will be recorded in the macro.

Note that while recording a macro, if a command is initiated that uses
the prompt line, the macro string display will be overwritten.  It
will come back after the first event.  However, if the command uses
the prompt line for input, the macro definition will be terminated as
if {\kb Esc} was entered.

Pressing {\kb Enter} while a drawing window has the keyboard focus
terminates the macro definition.  Pressing {\kb Esc} terminates a
macro without saving it.  {\kb Backspace} removes the last event when
defining a macro.  To enter {\kb Esc}, {\kb Backspace}, or {\kb Enter}
into the macro, use {\kb Ctrl-Esc}, {\kb Ctrl-Backspace}, {\kb
Ctrl-Enter}.

For example, Suppose we want to define {\kb Ctrl-x} to ``press'' the
{\cb Expand} button.  In the {\cb Define Macro} command, press {\kb
Ctrl} and {\kb x} and release.  In response to the next prompt, select
the {\kb Expand} button in the {\cb View Menu}, move the pointer into
a drawing window, and press the {\kb Enter} key.  From then on, {\kb
Ctrl-x} will be equivalent to pressing the {\kb Expand} button.

To undefine a macro, define it with a null definition for the body.

See the description of the ``!!'' interface in Chapt.  \ref{bangcmds}
for information on using script functions in macros.

In the present GTK-based user interface, there is less need for
macros due to the rich set of keyboard accelerators available.

\subsection{Macro File Format}
\index{.xicmacros file format}
The {\vt .xicmacros} file can contain any number of macro definitions. 
It is not expected that most users will have a need to work with this
file directly, though the possibility exists.  Each macro consists of
a block of lines in the following form:
\begin{quote}
{\vt \#macro}\\
{\vt KeyDown(... , NULL)}\\
{\it statements}\\
{\vt \#end}
\end{quote}

Lines that start with `{\vt \#}' that are not script preprocessor
keywords (see \ref{preproc}) are taken as comments.  The body of the
macro consists entirely of calls to four script functions {\vt
KeyDown}, {\vt KeyUp}, {\vt BtnDown}, and {\vt BtnUp}.  The first line
must be a {\vt KeyDown} command which specifies the character mapped
to the macro.  The widget string is the value ``{\vt NULL}''.  The
remaining lines contain calls to these functions, which simulate the
button and key presses recorded in the macro.

The events processed while {\Xic} is in use are recorded in the {\vt
xic\_run.log} file (see \ref{logfiles}).  This file and other log
files are stored in a temporary directory, which is deleted when
{\Xic} terminates normally.  To access the {\vt xic\_run.log} file
from {\Xic}, press the {\cb Log Files} button in the {\cb Help Menu},
then select the {\vt xic\_run.log} entry in the resulting file
selector, then press the green octagon on the file selector.  Advanced
users can cut/paste sequences of the commands into script files or
macros.


% -----------------------------------------------------------------------------
% xic:attr 101016
\section{The {\cb Set Attributes} Button: Set Window Attributes}
\index{Set Attributes button}
\index{attributes of window}
\index{window attributes}
The {\cb Set Attributes} button in the {\cb Attributes Menu} brings up
the {\cb Window Attributes} panel.  The panel contains controls which
affect presentation attributes in all drawing windows.  The panel
contains five tabbed pages:  {\cb General}, {\cb Selections}, {\cb
Phys Props}, {\cb Terminals}, and {\cb Labels}

The controls found in the {\cb General} page are as follows.

\begin{description}

\index{Set Cursor button}
\index{cursor mode}
\item{\cb Cursor}\\
This menu provides a choice of cursors.  These include the system
default cursor (which is probably the same as the left arrow), cross
cursor, left and right arrows.  Under Windows, there is no right
arrow, so an up arrow is used instead (but it is ugly and useless). 
Also, the default cross cursor for Windows 7 service pack 1 is huge
and grotesque, but can be switched for a better looking cross cursor
through the selections in the Windows {\cb System Preferences}
panel.

\item{\cb Use full-window cursor}\\
When this check box is checked, the mouse cursor will be represented
by horizontal and vertical lines which extend across the entire width
and height of the drawing window containing the cursor.  The lines
intersect at the nearest snap point in the current window.
    
When not checked, the cursor is the normal small cross.

This tracks the state of the {\et FullWinCursor} variable.

\item{\cb Subcell visibility threshold (pixels)}\\
This entry area specifies a pixel size threshold for display of
expanded subcells.  Subcells with height or width less than this
threshold will not be displayed, or displayed as an unfilled bounding
box, according to the setting of the {\cb Subthreshold Boxes} menu
button in the {\cb Main Window} sub-menu of the {\cb Attributes Menu},
and the {\cb Attributes} menu of sub-windows.  The value can be in the
range 0--100.  If 0, subcells will always be rendered when in expanded
state, which can greatly increase drawing time when zoomed out.  This
setting applies to all drawing windows.

This entry tracks the setting of the {\et CellThreshold} variable.

\item{\cb Push context display illumination percent}\\
When the {\cb Push} command is active, and the ``context'' (the
features surrounding the pushed-to subcell) is being displayed, the
intensity of colors used to render the context is reduced.  This
visually differentiates objects in the current cell from those in the
context.  The percentage intensity of the context can be set from this
input area.  If set to 100, the context is rendered with the same
coloring as the current cell.

This entry tracks the setting of the {\et ContextDarkPcnt} variable.

\item{\cb Pixels between pop-ups and prompt line}\\
For windows that are automatically placed just above the prompt line,
giving this entry a positive integer value will position these windows
toward the top of the screen by that many pixels.  This is useful when
using ``plasma'' displays (such as Mac or KDE), where the shadow falls
on the prompt line, which can be distracting.  It might also be
helpful if the window positioning is incorrect, which might occur with
some window managers.

This tracks the state of the {\et LowerWinOffset} Variable.
\end{description}

The {\cb Selections} page contains the following two check boxes.

\begin{description}
\item{\cb Show origin of selected physical instances}\\
When this check box is set, selected physical instances will have the cell
origin marked with a cross.  This applies to the selection
highlighting, as well as to the ghost rendition which is attached to
the mouse pointer during a move or copy operation.

Showing the origin may seem trivial, but marking the origin requires a
bit of overhead since it requires running a transformation and keeping
track of an additional redisplay area since the origin may be outside
of the cell bounding box.  Thus, the default is to not show the mark.

This tracks the state of the {\et MarkInstanceOrigin} variable, and
applies to all physical drawing windows.

\item{\cb Show centroids of selected physical objects}\\
In mathematics, the centroid or geometric center of a two-dimensional
region is the arithmetic mean of all the points in the shape.  When
this check box is set, selected objects will mark the centroid with a
cross.  This applies to the selection highlighting, as well as to the
ghost rendition which is attached to the mouse pointer during a move
or copy operation.

This tracks the state of the {\et MarkObjectCentroid} variable, and
applies to all physical drawing windows.
\end{description}

The {\cb Phys Props} page contains the following controls.

\begin{description}
\item{\cb Erase behind physical properties text}\\
When this check box is set, in windows where physical properties are
being displayed, the area around the physical property text is erased,
providing improved visibility.

This tracks the state of the {\et EraseBehindProps} variable.

\item{\cb Physical property text size (pixels)}\\
This entry sets the height, in pixels, of the text used to render
physical properties on-screen, when physical property text is being
displayed.

This tracks the state of the {\et PhysPropTextSize} variable.
\end{description}

The {\cb Terminals} page contains the following controls.

\begin{description}
\item{\cb Erase behind physical terminals}\\
This will cause the area under terminals in physical windows to be
erased, to promote visibility.  One can choose to not erase, to erase
only under the cell's terminals, or to erase under all terminals. 
This tracks the setting of the {\et EraseBehindTerms} variable.

\item{\cb Terminal text pixel size}\\
This sets the text height, in pixels, of the text associated with
terminals in both physical and electrical windows.  This tracks the
setting of the {\et TermTextSize} variable.  The default text height
is 14 pixels.

\item{\cb Terminal mark size}\\
This sets the pixel size of the mark used to indicate terminal
locations in both physical and electrical windows.  It tracks the
value of the {\et TermMarkSize} variable.  The default mark size is 10
pixels.
\end{description}

Finally, the {\cb Labels} page contains controls related to label
presentation.

\begin{description}
\item{\cb Hidden label scope}\\
By default, all labels participate in a protocol whereby clicking on
the label with the {\kb Shift} key held will ``hide'' the label,
displaying a small box instead.  {\kb Shift}-clicking on the box will
return to the display of the label text.  This menu limits the labels
which will participate in this protocol.  The choices are {\cb all
labels} (the default), {\cb all electrical labels}, {\cb electrical
property labels}, and {\cb no labels}.

This menu tracks the setting of the {\et LabelHiddenMode} variable.

\item{\cb Default minimum label height}\\
This sets the minimum label height, in microns, for new text labels. 
The actual initial height may be larger, depending on the zoom factor
of the window, but it can not be smaller. 

This tracks the setting of the {\et LabelDefHeight} variable.

\item{\cb Maximum displayed label length}\\
This entry sets the maximum width, in default-sized character cells,
of a displayed label.  If the label exceeds this width,
it is not shown, and a small box at the text origin is shown instead. 
The default is 256.

The ``hidden'' status of a property label can be toggled by clicking
the text or box with button 1 with the {\kb Shift} key held.
See \ref{labelbut} for more information.

This entry tracks the setting of the {\et LabelMaxLen} variable.

\item{\cb Label optional displayed line limit}\\
Label text strings may have embedded newline characters which cause
them to be displayed on multiple lines.  This setting, when set to a
positive integer value, provides a limit on the number of lines that
are actually displayed, in labels that respect this limit.  Only the
first {\it N} lines would actually appear in the display, where {\it
N} is the given number.  If {\it N} is zero, there is no limit.
 
Labels observe this limit only if an internal flag is set in the
label.  Presently, this is set internally for the labels associated
with {\et value} and {\et param} properties.  The user can apply the
limit to any label by setting the {\vt LIML} flag in the {\vt
XprpXform} pseudo-property.
 
The setting tracks the value of the {\et LabelMaxLines} variable.
\end{description}


% -----------------------------------------------------------------------------
% xic:dots 022712
\section{The {\cb Connection Dots} Button: Show Connections}
\index{dots button}
\index{connection points!show}
The {\cb Connection Dots} button in the {\cb Attributes Menu} brings
up the {\cb Connection Points} dialog, which contains three ``radio''
buttons which specify how connection points are to be indicated in
schematics shown in electrical mode windows.  This appears in {\Xic}
only.

If {\cb Don't show dots} is selected, there will not be any indication
of connection points.  This is the default.

If the {\cb Show dots normally} button is selected, a ``dot'' will be
shown at ambiguous connections points.  These are wire vertices common
to two or more wires (except for common end vertices of two wires),
non-endpoint wire vertices common with device or subcircuit terminals,
and any point common to three or more terminals or wire vertices.

A dot will also be shown if a cell connection point lies on an
internal (non-endpoint) wire vertex.  In instances, this marks the
connection location, and also ensures that a dot will be shown at this
location, as one likely would not appear otherwise due to the logic
used.

If the {\cb Show dot at every connection} button is selected, a dot
will be shown at every intersection recognized as a connection by
{\Xic}.  This can sometimes be useful for debugging connection
problems in drawings.

The {\cb Connection Points} choices track and set the state of the
{\et ShowDots} variable.  This variable can be set in the technology
file or a startup file to initialize the connection point indication
mode.


% -----------------------------------------------------------------------------
% xic:font 032513
\section{The {\cb Set Font} Button: Set Window Fonts}
\index{Set Font button}
\index{font}
The {\cb Set Font} button in the {\cb Attributes Menu} brings up the
{\cb Font Selection} panel, which allows selection of the fonts used
in the graphical interface.  A drop-down menu provides selection of
the various font targets.  Pressing the {\cb Apply} button will
immediately apply the selected font to all visible windows which use
the font.

Although many of the fonts used in the graphical interface can be set
from this panel, the main font, the one used for menu text and control
labels, must be set externally (it must be available before the
graphical interface is created).  This is probably set by your desktop
environment, and there should be a tool available for customization. 
One can also likely set various "themes" which alter the appearance of
the window decorations.

This is not true in Windows.  Perhaps there is a better way, but one
can do the following:  In the directory {\vt c:}$\backslash${\vt Documents
and Settings}$\backslash${\it your\_username}, create a file named {\vt
.gtkrc-2.0} containing lines similar to

\begin{quote}
{\vt style "win32-font"} \textbraceleft\\
\hspace*{2em}{\vt font\_name = "tahoma 12"}\\
\textbraceright\\
{\vt class "*" style "win32-font"}
\end{quote}

You may need a DOS or Cygwin window to create the file, as
Windows Explorer cannot create a file starting with a period.

The {\cb Dump Vector Font} button in the {\cb Font Selection} panel
will dump the vector font used for text labels in the drawing areas to
a file.  The user will be prompted for the name of a file to use, the
default name is ``{\vt xic\_font}''.  Any existing file with the same
name will be backed up with a {\vt .bak} extension.  A font file with
this name found along the library search path will be read on program
startup, and will define the label font, overriding the internal font. 
The user can start by dumping the internal font, and tweak this to
their taste.  The format of the vector font file is discussed in
\ref{vectorfont}.

The drop-down font targets list contains the following entries:
\begin{description}
\item{\cb Fixed Pitch Text Window Font}\\
This sets the font used in pop-up multi-line text windows, such as the
{\cb Files Listing} and {\cb Cells Listing}, where the names are
formatted into columns.

\item{\cb Proportional Text Window Font}\\
This sets the font used in pop-up multi-line text windows where text
is not formatted, such as the info and error message pop-ups.

\item{\cb Fixed Pitch Drawing Window Font}\\
This is the font used in the coordinate readout, the status line,
layer table, and the prompt line.  It is not the font used to render
label text in the drawing windows, which is a vector font generated by
other means.

\item{\cb Text Editor Font}\\
This is the font used in the {\cb Text Editor} pop-up.

\item{\cb HTML Viewer Proportional Font}\\
This is the base font used for proportional text in the HTML viewer
(help windows).  If set, this will override the font set in the {\vt
.mozyrc} file, if any.

\item{\cb HTML Viewer Fixed Pitch Font}\\
This is the base fixed-pitch font used by the HTML viewer.  If set,
this will override the font set in the {\vt .mozyrc} file, if any.
\end{description}

The {\cb Font} button in the {\cb Options} menu of the text editor
brings up a similar panel, as does the {\cb Font} button in the {\cb
Options} menu of the help viewer.

These fonts can be set in the technology file, and are updated to
the technology file when a {\cb Save Tech} command is given.


% -----------------------------------------------------------------------------
% xic:color 022215
\section{The {\cb Set Color} Button: Set Colors Panel}
\index{Set Color button}
\index{color panel}
\index{set layer colors}
The {\cb Set Color} button in the {\cb Attributes Menu} brings up the
{\cb Color Selection} panel.  The panel can also be displayed by
clicking on a layer in the layer table or layer palette with button 3
while holding down the {\kb Ctrl} key.  The current layer will become
the clicked-on layer, and its color will be loaded into the editor.

The panel contains controls that are manipulated to set the components
of the color of the currently selected layer or other drawing
attribute.  If the {\cb Print Control} panel (induced by the {\cb
Print} button in the {\cb File Menu}) is visible, the color set will
be used for rendering the plot, if the plot driver supports definable
colors.

Near the top of the panel is a pull-down menu plus a ``radio group''
of three buttons.  The three buttons allow the pull-down menu to
contain different sets of entries.  Each entry represents a color that
can be adjusted.
    
The three buttons provide the following sets of menu entries:

\begin{description}
\item{{\cb Attributes} button}\\
These are the colors used in the drawing windows.  Most of these
colors can be separately set while in electrical or physical mode.

\begin{tabular}{|l|l|} \hline
\cb Current Layer & Current layer color\\ \hline
\cb Background & Drawing window background color\\ \hline
\cb Coarse Grid & Color used for coarse grid lines\\ \hline
\cb Fine Grid & Color used for fine grid lines\\ \hline
\cb Ghosting & Color used for ``sprites'' attached to
  the mouse pointer\\ \hline
\cb Highlighting & Color used for highlighting, such as
  for DRC errors\\ \hline
\cb Selection Color 1 & One of two alternating colors used
  for selections\\ \hline
\cb Selection Color 2 & One of two alternating colors used
  for selections\\ \hline
\cb Terminals & Electrical terminals\\ \hline
\cb Instance Boundary & Boundary color of unexpanded
  instance\\ \hline
\cb Instance Name Text & Name text color in unexpanded
  instance\\ \hline
\cb Instance Size Text & Size text color in unexpanded
  instance, physical mode only\\ \hline
\end{tabular}
\vspace*{3mm}

\item{{\cb Prompt} button}\\
These are the colors used in the prompt line and status line.

\begin{tabular}{|l|l|} \hline
\cb Text & Normal prompt line text\\ \hline
\cb Prompt Text & Text color used for prompting\\ \hline
\cb Highlight Text & Text color used for hypertext
  references\\ \hline
\cb Cursor & Text cursor color\\ \hline
\cb Background & Normal background color\\ \hline
\cb Edit Background & Background color while editing\\ \hline
\end{tabular}

\item{{\cb Plot Marks} button}\\
These are the colors of the plot point marks used in electrical mode
to indicate a node or current being plotted by {\WRspice}.  The
default colors are the same as the trace colors used by {\WRspice}
for plotting.

The entries are:  {\cb Plot Mark 1} to {\cb Plot Mark 18}.
\end{description}

The rest of the panel consists of the stock GTK-2 color selection
widgets.  There are six up/down buttons which can adjust the red,
green, and blue values, or the hue, saturation, and intensity values. 
To the left is a color wheel, with a triangle inside.  One can drag
the marked triangle vertex around the outer ring to set the color, and
drag the small circle in the triangle to set the lightness.  All
widgets automatically track the current color setting. 

There is also a palette containing several colors, and a rectangular
color display area for the current and previous colors.  Colors can be
dragged between the palette locations and the current color location. 
The eye-dropper button allows setting the current color from a
clicked-on screen object.

The {\cb Colors} button brings up a listing of color names and RGB
values.  Clicking on a list entry will load that color into the color
selector.

The {\cb Apply} button must be pressed to actually transfer the new
color to {\Xic}.  For layer colors, drag/drop of a color to the
entries in the layer table or layer palette can be done as well.  When
changing layer colors, or screen attribute colors such as grid colors,
the main window and similar (same display mode) sub-windows will be
redrawn.  The {\et NoPhysRedraw} variable, if set, will suppress
automatic redraw of physical-mode windows.

When the {\cb Print Control Panel} panel is visible, i.e., in hard
copy mode, the colors set will be used in that mode only, and in the
plots if the printer driver supports it.


% -----------------------------------------------------------------------------
% xic:fill 022215
\section{The {\cb Set Fill} Button: Fill Pattern Edit Panel}
\index{Set Fill button}
\index{fill panel}
\index{fill patter editing}
The {\cb Set Fill} button in the {\cb Attributes Menu} brings up the
{\cb Fill Pattern Editor} panel.  The panel can also be displayed by
clicking on a layer in the layer table or layer palette with button 3
while holding down the {\kb Shift} key.  The current layer will become
the clicked-on layer, and its fill pattern will be loaded into the
editor.

This panel initially displays an array of sample fill patterns.  By
pressing the {\cb Pixel Editor} button, the pop-up will instead
display a large pixel editor window, from which stipple patterns can
be created and modified.  The pixel editor is pre-loaded with the
pattern of the current layer when the pop-up was invoked.  When
viewing the pixel editor, the {\cb Show Stores} button reverts the
display to the sample patterns.

Patterns are moved between the various boxes and layers in the layer
table by drag/drop.  Press and hold the left mouse button while the
pointer is over the box or layer entry that is the source of a
pattern, and release the button after moving the mouse pointer over
the destination box or layer table entry.

There are also {\cb Load} and {\cb Apply} buttons that will load the
pattern of the current layer into the editor, or set the pattern of
the current layer from the {\cb Sample}, respectively.

The {\cb Sample} box, which is visible in both display modes, is the
entry and exit point for the pixel editor.  Dropping a pattern into
the {\cb Sample} area will load the pattern into the editor.  The
current pattern in the {\cb Sample} area, which matches that currently
in the editor, can be dragged and dropped onto a layer, or into one of
the storage areas.  When the {\cb Pixel Editor} window is visible, it
is also a drop receiver, equivalent to the {\cb Sample} area.

When a pattern is applied to a layer by any means, the main window and
any sub-windows showing the same display mode will be redrawn.  The
automatic redraw can be suppressed in physical mode by setting the
{\cb NoPhysRedraw} variable.

When displaying the pattern array, there are 18 pattern windows
visible.  The first two of these are immutable, containing empty and
solid fills.  The remaining 16 are registers, preset with default
patterns.  These patterns can be changed by dropping a new pattern
into the display box.

There are 64 pattern registers available, in four pages.  When the
patterns are visible, the {\cb Page} spin button can be used to cycle
between the four sets of registers.

To summarize, the following transfers are possible with drag/drop:
\begin{itemize}
\item{To the {\cb Sample} box from any pattern register box, or the
empty and solid boxes, or from any layer in the layer table or layer
palette.  Drag patterns into the {\cb Sample} area to allow editing
with the pixel editor.  The {\cb Pixel Editor} window is also a drop
receiver equivalent to the {\cb Sample} area.}

\item{From the {\cb Sample} box to any pattern register box, or to
any layer in the layer table or layer palette.  Drag patterns into
layers to set the pattern used for that layer.  A pattern dragged
into a pattern register box will replace the existing pattern with
the one from the source.}

\item{From any pattern register box, or the solid and empty fill
boxes, to any layer in the layer table or layer palette, or to the
{\cb Sample} box.}

\item{To any pattern register box, from any layer from the layer table
or layer palette, or the {\cb Sample} box.  Note that it is not
possible to drag/drop patterns between pattern register boxes
directly, one can drag from a register to the {\cb Sample} box, then
from the {\cb Sample} box to another register box.}

\item{From any layer in the layer table or layer palette to any of
the default pattern boxes except solid and empty, or to the
{\cb Sample} box.}

\item{To any layer in the layer table or layer palette, from any of
the pattern register boxes, or the solid and empty boxes, or the {\cb
Sample} box.}
\end{itemize}

If the {\cb Print Control} panel (induced by the {\cb Print} button in
the {\cb File Menu}) is visible, when a new fill pattern is applied to
a layer, the new fill pattern will be used for rendering the plot, if
the plot driver supports definable fill patterns.

Note that the new pattern set for a layer will not be visible in the
drawing windows until they are next redrawn (press {\kb Ctrl-r}, or
click with button 2 near the center of the window to redraw the
window).

The color used to display patterns in the pop-up is the color of the
current layer.  Initiating a drag from a layer in the layer table or
layer palette will change the current layer (and hence the color) to
that layer.

The 64 ``default'' fill patterns can be saved in an {\vt xic\_stipples}
file in the current directory with the {\cb Dump Defs} button, which
is visible when the pattern array is being displayed.  If this file is
found in the library search path, it will be used to initialize the
pattern registers when {\Xic} starts.  A system default {\vt
xic\_stipples} file is provided in the startup directory.

When a pattern (including empty) is dropped on a layer, the {\cb
Outline}, {\cb Fat}, and {\cb Cut} buttons at the bottom of the pop-up
set additional attributes relating to the pattern display.

When a pattern from a layer is dropped into the {\cb Sample} box, the
buttons will change state to that saved in the layer.  The pattern
registers, however, do not have attributes, so that the buttons remain
unchanged when a pattern is dragged from a register box, except to
gray the {\cb Fat} button if the new pattern is not empty.

For empty fill, there are three available outline styles:
\begin{enumerate}
\item{A thin solid line boundary.}
\item{A thin dashed line boundary.}
\item{A thick solid line boundary for Manhattan boxes and polygons,
and a thin solid line boundary for other objects.}
\end{enumerate}

If the {\cb Outline} button is not in the pressed state, the thin
solid line boundary (style 1) will be used.  If {\cb Outline} is in
the pressed state, and {\cb Fat} is not in the pressed state, a thin
dashed outline (style 2) will be used.  If {\cb Fat} is pressed, thick
segments (style 3) will be used, but only for edges of boxes and
Manhattan polygons.  A thin solid outline will be used elsewhere.

If the {\cb Cut} button is in the pressed state, boxes are rendered
with thin lines along the diagonals, forming an X over the box. 
Polygons (even four-sided rectangular ones) and wires are drawn
normally.  The ``cut'' attribute is often used to signify vias.

If {\cb Cut} and {\cb Outline} are pressed with empty fill, but {\cb
Fat} is not pressed, the diagonal ``cut'' lines will be drawn as
dashed, as for the outline.

When the pattern is not empty, the options are slightly different.  If
{\cb Outline} is not pressed, objects will not be outlined.  If {\cb
Outline} is pressed and {\cb Fat} is not pressed, the patterned areas
will have a thin solid outline.  When {\cb Fat} is pressed, boxes and
Manhattan polygons will be shown with thick outlines.  These are the
only boundaries available with stippled fill.  The {\cb Cut} button
will produce diagonals over boxes, as in the empty-fill case.  If the
pattern is solid, none of the attributes will be used, and the buttons
are grayed.

When a pattern is dropped on a layer, the state of these buttons set
the attributes for the layer.  This applies whether the pattern source
is the {\cb Sample} box, or one of the pattern register boxes.  In
particular, to set up a desired empty-fill presentation for a layer,
one would set the buttons, then drag the ``pattern'' from the
empty-fill box to the layer to set.

In the pixel editor, the {\cb NX x NY} spin buttons control the size
of the map.  Each coordinate can range from 2 to 32.  The pixel editor
window changes to accommodate the different aspect ratios, keeping the
actual pixels square.

The three buttons just below operate on the pixel map.
\begin{description}
\item{\cb Rot90}\\
Pressing this button will rotate the pixel map by 90 degrees.  Note that
this swaps the Nx and Ny values.

\item{\cb X}\\
This will flip the map right-to-left (mirroring about the Y axis).

\item{\cb Y}\\
This will flip the map top-to-bottom (mirroring about the X axis).
\end{description}

The basic operation in the pixel editor is to toggle the state of a
pixel by clicking on it with button 1, but more complex possibilities
exist.  A hold and drag will operate on all of the pixels enclosed in
or intersecting the defined rectangle, which is ghost-drawn.  The
operations are indicated in the table below.

\begin{tabular}{|l|l|l|l|l|} \hline
\bf Button & \bf Figure & & \bf Shift & \bf Ctrl\\ \hline
1 & solid box & toggle & set & unset\\ \hline
2 & open box & toggle & set & unset\\ \hline
3 & line & toggle & set & unset\\ \hline
\end{tabular}

If {\kb Shift} or {\kb Ctrl} is held down {\it before} button 1 is
pressed, the action will be as for button 2.  The three columns from
the right indicate the state of the modifier keys on button release
which produces the stated effect on the pixels.  The {\kb Ctrl} press
overrides {\kb Shift} if both are pressed.

Pressing the arrow keys while the pop-up has the keyboard focus
permutes the pixel editor bitmap in the opposite direction of the
arrow.  This is valuable for allowing layers with similar patterns to
show through one another.

Pressing the {\cb Dismiss} button in the {\cb Fill Pattern Editor}
will retire the editor.  This has the same effect as pressing the {\cb
Set Fill} menu button a second time.


% -----------------------------------------------------------------------------
% xic:edlyr 022115
\section{The {\cb Edit Layers} Button: Edit Layer Table}
\index{Edit Layers button}
\index{edit layer table}
The {\cb Edit Layers} button in the {\cb Attributes Menu} brings up
the {\cb Layer Editor} pop-up, which contains buttons for adding and
removing layers from the layer table, plus a drop-down menu of removed
layers which can be added back.  When the pop-up first appears, the
text area will be blank, or will contain the name of the last removed
layer.  The text area can be edited to provide the name of a new layer
to add, or the name of a removed layer can be selected in the
drop-down menu.  After pressing the {\cb Add Layer} button and
clicking in the layer table, the new layer will be added at the
location of the layer entry clicked on.  One can also click beyond the
end of the listed layers to put the new layer above the existing
layers.  With the {\cb Remove Layer} button pressed, layers clicked on
will be removed from the layer table and added to the list in the
drop-down menu.
 
When layers are removed, the geometry on the layer is not affected,
however it will be invisible on-screen (after the first redraw) and to
all commands, and the geometry will not be included if the cell is
updated to disk.
    
Layers can also be added, removed, and renamed with the {\cb !ltab}
command.


% -----------------------------------------------------------------------------
% xic:lpedt 022215
\section{The {\cb Edit Tech Params} Button: Edit Tech Keywords}
\index{Edit Tech Params button}

The {\cb Edit Tech Params} button in the {\cb Attributes Menu} brings
up the {\cb Tech Parameter Editor}.  The editor can also be displayed
by clicking on a layer in the layer table or layer palette with button
3 while holding down the {\kb Shift} and {\kb Ctrl} keys.  The current
layer will become the clicked-on layer, and it will become the target
layer for the editor.

From the editor, many of the technology file keywords, mostly
associated with layers, can have their specifications added, deleted,
or edited.  Keywords that are not adjustable in the editor have an
alternate means of control, such as the separate panels for setting
layer colors and fill patterns.  After modification, the {\cb Save
Tech} button in the {\cb Attributes Menu} can be used to generate a
new technology file that incorporates the changes.

The present editor applies to all keywords.  In earlier {\Xic}
releases, there were separate keyword editors in the {\cb Extract} and
{\cb Convert} menus, as well as an editor in the {\cb Attributes
Menu}.  The different keyword classes, previously available in
separate editors, are now available by selecting one of the four tabs
at the top of the editor window.  Each of these keywords is normally
applied to a layer in the technology file.

The editor is configured for the class of keywords selected, meaning
that the pull-down menu contains entries for those keywords, and the
listing details those keywords only.  The classes, and keywords that
can be manipulated, are listed below.

\begin{description}
\item{\cb Layer}\\
These are miscellaneous layer attributes, as described in \ref{misclattr}.
\begin{quote} \vt
LppName\\
Description\\
Symbolic\\
Invisible\\
NoSelect\\
Invalid\\
NoMerge\\
CrossThick\\
WireActive\\
NoInstView\\
WireWidth
\end{quote}

\item{\cb Extract}\\
These are parameters that are used by the extraction system, but have
relevance elsewhere, such as in the cross section viewer.  These
keywords are described in \ref{exkwords}.
\begin{quote} \vt
Conductor\\
Routing\\
GroundPlane\\
GroundPlaneClear\\
Contact\\
Via\\
Dielectric\\
DarkField
\end{quote}

\item{\cb Physical}\\
These generally set a physical property of the layer material,
primarily in support of the extraction system.  These keywords are
described in \ref{phkwords}.
\begin{quote} \vt
Planarize\\
Thickness\\
Rho\\
Sigma\\
Rsh\\
EpsRel\\
Capacitance\\
Lambda\\
Tline\\
Antenna
\end{quote}

\item{\cb Convert}\\
These are mostly for establishing the mapping between {\Xic} layers
and GDSII layer and datatype numbers, as described in \ref{convlattr}.
\begin{quote} \vt
StreamIn\\
StreamOut\\
NoDrcDatatype
\end{quote}
\end{description}

In addition to the per-layer keywords, there is a {\cb Global
Attributes} menu that allows a few miscellaneous non-layer parameters
to be manipulated.  These were defined as technology file parameters
in {\Xic} releases earlier than 4.1.7.  In present releases, these are
simply variables.  The editor provides a ``front end'' for setting the
variables.

\begin{description}
\item{\cb Global Attributes}\\
\begin{quote} \vt
BoxLineStyle\\
LayerReorderMode\\
NoPlanarize\\
AntennaTotal\\
SubstrateEps\\
SubstrateThickness
\end{quote}
\end{description}

The editor is similar to the {\cb Design Rule Editor} found in the
{\cb DRC Menu}.  When the editor first appears, the keyword
specifications for the current layer in the selected class are listed. 
The specifications appear as they would in the technology file. 
Changing the current layer will update the listing to the parameters
for the new current layer.  Selecting a different keyword class will
update the display to show the keywords in the selected class.  The
user can add new keyword lines or modify existing lines as desired.

To add a keyword specification, one selects the desired keyword in the
keywords menu of the editor.  The available keywords are listed in the
drop-down menu, and the set available depends on the class tab
currently selected.  After clicking on a keyword in the menu, the user
will be asked to enter the associated text in the prompt line, if the
keyword requires it.  The keyword will be applied internally and
appear in the listing if there are no errors.  A status message will
appear, indicating success, or providing an error message.

When adding a keyword, redundant and inconsistent keywords that are
already in the list, such as a previous instance of the keyword, are
removed.  In other cases, a pop-up message will appear if inconsistent
keywords are found.

Clicking on a line in the listing will select the line.  The text for
the selected line can be edited, or the line deleted, with the {\cb
Edit} and {\cb Delete} buttons in the editor's {\cb Edit} menu.  The
{\cb Edit} menu also contains an {\cb Undo} button, allowing the last
operation to be undone.


% xic:attrsubmenu 011916
% -----------------------------------------------------------------------------
\section{The {\cb Main Window} Button: Attributes sub-menu}
\label{attrsubmenu}
\index{Main Window button}
\index{Attributes sub-menu}
The {\cb Main Window} button in the {\cb Attributes Menu} brings up a
sub-menu which is identical to the {\cb Attributes} menu in the
sub-windows produced by the {\cb Viewport} button in the {\cb View
Menu}.  The menu contains attribute settings which apply to the main
window.  When a new sub-window appears, its attributes are inherited
from the main window, but can be reset for the sub-window through its
{\cb Attributes} menu.

\begin{tabular}{|l|l|l|p{2.25in}|} \hline
\multicolumn{4}{|c|}{\kb Main Window Sub-Menu}\\ \hline
\kb Label & \kb Name & \kb Pop-up & \kb Function\\ \hline\hline
\et Freeze Display & \vt freez & none & Suppress redisplay\\ \hline
\et Show Context in Push & \vt cntxt & none & Show context in subedit\\ \hline
\et Show Phys Properties & \vt props & none & Show physical properties\\ \hline
\et Show Labels & \vt labls & none & Show labels\\ \hline
\et Label True Orient & \vt larot & none & Show labels transformed\\ \hline
\et Show Cell Names & \vt cnams & none & Show cell names\\ \hline
\et Cell Name true Orient & \vt cnrot & none & Show cell names transformed\\
  \hline
\et Don't Show Unexpanded & \vt nouxp & none & Don't show unexpanded
  subcells\\ \hline
\et Objects Shown & \vt objs & none & Object display control\\ \hline
\et Subthreshold Boxes & \vt tinyb & none & Show outline of subthreshold
  cells\\ \hline
\et No Top Symbolic & \vt nosym & none & Electrical only, don't show top
  cell as symbolic\\ \hline
\et Set Grid & \vt grid & {\cb Grid Setup} & Set grid parameters\\ \hline
\end{tabular}

% xic:freez 042909
\subsection{The {\cb Freeze Display} Button: Suppress Redisplay}
\index{Freeze Display button}
\index{frozen window}
\index{viewing huge cells}
When the {\cb Freeze Display} button in the {\cb Main Window} sub-menu
of the {\cb Attributes Menu} or the sub-window {\cb Attributes} menu is
active, no cell structure is drawn in the window, only the grid and
the cell bounding box.  This is for use when working on a large,
complex design when it is not necessary to see the structure and it is
inconvenient to wait for the display.  When active, ``FROZEN'' appears
in the upper left corner of the window.

In a frozen window, certain other highlighting features may appear. 
In particular rulers and the viewport location indicators as used by
the {\cb Show Location} function will be displayed.  Also, the
outlines of selected objects will appear in these windows.
    
Frozen sub-windows have an additional feature:  frozen sub-windows
display the viewport of the main window, the reverse of the {\cb Show
Location} function for sub-windows (which displays sub-window viewports
in the main window).

This allows a useful trick for viewing huge cells.  Suppose that one
has a large design which takes a long time to render, and one wishes
to examine a small part of this design (the approximate coordinates
are known).  One can employ the following procedure.  Freeze the main
window and read in the design.  Bring up a sub-window by clicking twice
outside of the cell boundary, so that the sub-window is empty.  Freeze
the sub-window, then press the {\kb Home} key with the cursor in the
sub-window to center and fully view the top cell boundary.  Use the
grid and/or rulers to determine the region of interest.  Drag with
button 3 to define a rectangle in the sub-window surrounding the region
of interest, then click with button 3 in the center of the main
window.  Un-freeze the main window, and this region will be displayed. 
The region shown in the main window is shown with a dotted yellow
outline in the frozen sub-window.


% xic:cntxt 042909
\subsection{The {\cb Show Context in Push} Button: Control Context Display}
\index{Show Context in Push button}
When the {\cb Show Context in Push} button in the {\cb Main Window}
sub-menu of the {\cb Attributes Menu} or the sub-window {\cb
Attributes} menu is active, the context is displayed in the window
when the {\cb Push} command in the {\cb Cell Menu} is active.  The
context is the surrounding geometry in cells other than the instance
of the cell that was ``pushed'' into. 


% xic:props 042909
\subsection{The {\cb Show Phys Properties} Button: Show Physical-Mode
 Properties}
\label{xicprops}
\index{Show Phys Properties button}
The {\cb Show Phys Properties} button in the {\cb Main Window}
sub-menu of the {\cb Attributes Menu} or the sub-window {\cb
Attributes} menu enables the display of object properties on-screen
while the window is in physical mode.  The property text is placed
near the leftmost vertex with largest y value of a polygon, the
leftmost end of a wire, or in the upper left corner of the object's
bounding box.  Properties of the cell itself are not displayed.

Properties can be assigned with the {\cb Properties} command in the
{\cb Edit Menu}.

Outside of any command, clicking on a visible physical property string
allows the property to be edited.  The property string must be close
to or overlap the object.  If the string is too far away from the
object (if there are a large number of properties, or the object size
is small) it can not be selected in this way.  The user is first
prompted for new text, then for a new property number.  The clicked-on
property will be replaced with the new values.  If {\kb Ctrl-d} is
pressed at any time while responding to the prompts, the property will
be deleted, with no replacement.

Properties with property numbers 7000--7104 are not displayed.  These
numbers are reserved for internal use and should not be assigned.

The {\cb Erase behind physical properties text} check box in the {\cb
Window Attributes} panel, or equivalently the {\et EraseBehindProps}
variable can be set to erase around the property strings, enhancing
visibility.


% xic:labls 042909
\subsection{The {\cb Show Labels} Button: Control Label Display}
\index{Show Labels button}
When the {\cb Show Labels} button in the {\cb Main Window} sub-menu of
the {\cb Attributes Menu} or the sub-window {\cb Attributes} menu is
active, labels will be displayed in the window, otherwise label
rendering is suppressed.  It is unlikely that it would be necessary to
turn off the display of labels, unless a layout has so many labels
that important features are obscured.  Labels are shown in legible
orientation by default, however if the {\cb Label True Orient} button
is active, labels will be shown with all transformations applied.


% xic:larot 042909
\subsection{The {\cb Label True Orient} Button: Set Label Orientation}
\index{Label True Orient button}
The {\cb Label True Orient} button in the {\cb Main Window} sub-menu
of the {\cb Attributes Menu} or the sub-window {\cb Attributes} menu is
sensitive only when labels are being displayed (the {\cb Show Labels}
button is active).  When active, labels will be shown in true
orientation, i.e., all transformations are applied to the text before
display.  If not active, labels will always be shown in ``legible''
orientation.


% xic:cnams 042909
\subsection{The {\cb Show Cell Names} Button: Display Cell Names}
\index{Show cell Names button}
When the {\cb Show Cell Names} button in the {\cb Main Window}
sub-menu of the {\cb Attributes Menu} or the sub-window {\cb
Attributes} menu is active, cell names and other information are
printed within the bounding box outline of unexpanded subcells.  The
text is shown in a legible orientation, but if the {\cb Cell Name True
Orient} button is set, the text will be transformed in the same way as
the cell.


% xic:cnrot 042909
\subsection{The {\cb Cell Name True Orient} Button: Set Cell Name Orientation}
\index{Cell Name true orient button}
The {\cb Cell Name True Orient} button in the {\cb Main Window}
sub-menu of the {\cb Attributes Menu} or the sub-window {\cb
Attributes} menu is sensitive when cell names are being displayed in
unexpanded instances (the {\cb Show Cell Names} button is active). 
When set, the name text is transformed in the same way as the subcell. 
When not set, the name text is always shown in a ``legible''
orientation.  It is sometimes convenient to invoke this option, as one
can see at a glance which subcells are rotated, mirrored, etc.


% xic:nouxp 042909
\subsection{The {\cb Don't Show Unexpanded} Button: Don't Show Unexpanded
 Subcells}
\index{Don't Show Unexpanded button}
Normally, unexpanded subcells are shown as a bounding box, containing
a cell name label and perhaps other text.  When the {\cb Don't Show
Unexpanded} button in the {\cb Main Window} sub-menu of the {\cb
Attributes Menu} or the sub-window {\cb Attributes} menu is active,
unexpanded subcells will not be displayed at all, i.e., they will be
invisible.


% xic:objs 053110
\subsection{The {\cb Objects Shown} Button: Object Display menu}
\index{Objects Shown button}
This button in the {\cb Main Window} sub-menu of the {\cb Attributes
Menu} and the {\cb Attributes} menu of subwindows brings up a sub-menu
containing three checkable entries:  {\cb Boxes}, {\cb Polys}, and
{\cb Wires}.  All three entries are checked by default.  If unchecked,
objects of that type will not be shown in the display in the
corresponding window.  The window will have to be redrawn to see the
effect, the redraw is not automatic.
 
Display of labels and instances is controlled by other buttons in the
same menu.


% xic:tinyb 042909
\subsection{The {\cb Subthreshold Boxes} Button: Outline Tiny Subcells}
\index{Subthreshold Boxes button}
When the {\cb Subthreshold Boxes} button in the {\cb Main Window}
sub-menu of the {\cb Attributes Menu} or the sub-window {\cb
Attributes} menu is active, subcells with a displayed size smaller
than a threshold will be shown in unexpanded form, as an unfilled box. 
Otherwise, the cell will not be shown at all.

The threshold pixel size can be adjusted with the {\cb Subcell
visibility threshold (pixels)} entry area in the {\cb Main Window
Attributes} panel, or equivalently by setting the {\et CellThreshold}
variable.


% xic:nosym 090814
\subsection{The {\cb No Top Symbolic} Button: Enforce Schematic View}
\index{No Top Symbolic button}
This button will appear in the {\cb Main Window} sub-menu of the {\cb
Attributes Menu} or the sub-window {\cb Attributes} menu only when the
window is displaying in electrical mode.  Thus, it never appears in
{\XicII} or {\Xiv} feature sets.

When set, the top-level cell will be displayed as a schematic, whether
or not the top-cell has an active symbolic representation.

Unlike the {\cb symbl} button in the electrical side menu, this does
not change the internal state of the cell (thus triggering the
``modified'' flag), and applies only to the window where set.  It is
available in electrical sub-windows in physical mode, when the
electrical side menu is hidden.

All editing capability is available, so it is possible to edit the
schematic and symbolic views of the same cell simultaneously, in
different windows.


% xic:grid 022215
\subsection{The {\cb Set Grid} Button: Set Grid Parameters}
\label{gridpop}
\index{Set Grid button}
\index{Grid Setup panel}
\index{snap grid}
\index{grid style}
\index{keyboard!Ctrl-g}
The {\cb Set Grid} button in the {\cb Main Window} sub-menu of the
{\cb Attributes Menu} or the sub-window {\cb Attributes} menu brings
up the {\cb Grid Setup} panel.  Pressing {\kb Ctrl-g} when a drawing
window has keyboard focus will also produce this panel.  The panel
provides control of the grid display in the associated drawing window. 
Separate grid styles are available for electrical and physical mode,
and in the main drawing window when the {\cb Print Control Panel} is
visible.

The panel is divided into two pages:  {\cb Snapping} and {\cb Style}. 
The {\cb Snapping} page is shown when the panel first appears.  It
provides control of the grid spacing and snapping, plus is the main
control point for the edge snapping feature to be described.  The {\cb
Style} page provides control over the visual presentation of the grid.

In most cases, the grid is not actually changed unless/until the {\cb
Apply} button, at the bottom-left of the panel, is pressed.  This is
not true of the {\cb Edge Snapping} group, or the {\cb All Windows}
controls in the {\cb Style} page, which work immediately, but do not
force a screen redraw.

The following controls appear in the {\cb Snapping} page.

At the top of the panel are entries which control the grid spacing and
snapping.  There is a coarse grid, and a fine grid, that may be
displayed in drawing windows and hard-copy output.  The coarse grid is
an integer multiple of fine grid increments, the multiplier can be set
from the {\cb Style} page.  If a grid would be too fine, it is not
shown.

The snap grid represents points where the cursor is allowed to reside. 
These are related to the fine grid interval, there can be an integer
number of snap points per fine grid interval, or an integer number of
fine grid lines between snap points.

\begin{description}
\index{MfgGrid keyword}
\index{DatabaseResolution variable}
\item{\cb Snap Spacing}\\
The {\cb Snap Spacing} entry area will set the spacing, in microns,
between snap points.  The mouse pointer is constrained to fall only on
snap points when geometry is being created.  If the {\et MfgGrid}
parameter has been set in the technology file, the {\cb Snap Spacing}
value is constrained to be a multiple of this value.  The {\et
MfgGrid} value is printed below the text entry area if set, or the
word ``{\vt unset}'' is printed if not.

The {\cb Snap Spacing} entry displays the actual snap spacing in
microns.  If a {\vt MfgGrid} has been defined, and one enters a snap
spacing that is not an integer multiple of this value, the actual snap
spacing will ``snap'' to the closest multiple before use.

When assigning a {\vt MfgGrid}, one must consider the internal
resolution, and the {\vt MfgGrid} should be representable in the
resolution in use.  For example, the default resolution is 1000 per
micron, or 1nm.  If one attempts to use a {\vt MfgGrid} of 2.5nm,
round-off error will occur.  To support this {\vt MfgGrid}, a
resolution of 2000 would be required.  The resolution is set with the
{\et DatabaseResolution} variable.

When the panel first appears, the initial keyboard focus is to the
{\cb Snap Spacing} entry.  Thus, text typed will go to this entry. 
Pressing {\kb Enter} when a text entry has the focus calls the {\cb
Apply} callback and sets focus to the {\cb Dismiss} button, where
another {\cb Enter} press will dismiss the panel.  So, to quickly
change the grid spacing from the keyboard only, one can type

\begin{quote}
{\kb Ctrl-g} (adjust number) {\kb Enter Enter}
\end{quote}

\item{\cb SnapPerGrid} or {\cb GridPerSnap}\\
To the right is another text entry, which accepts an integer in the
range 1--10.  The entry label indicates {\cb SnapPerGrid} or {\cb
GridPerSnap}, depending on whether the {\cb GridPerSnap} check box,
below the text entry, is checked.  This controls the placement of the
fine grid lines in the window.

If the label indicates {\cb SnapPerGrid}, then the fine grid is spaced
at the given integer times the {\cb Snap Spacing} value.  For example,
if the integer is 3, then a fine grid line (or dot) will be drawn at
every third snap point.  there will be three snap points per fine grid
interval.

If the label indicates {\cb GridPerSnap}, then the fine grid is spaced
such that the integer will give the number of fine grid lines per snap
interval.  For example, it the integer is 3, fine grid lines (or dots)
will appear at the snap points, as well as the 1/3 and 2/3
proportional distances between snap points.

Note that when the integer is 1, there is no difference between the
two cases.

In electrical mode, the snap interval should be a multiple of one
micron, to avoid connectivity errors due to numerical roundoff. 
However, this was not enforced in older releases of {\Xic}. 
Presently, sub-micron snapping on tenth-micron intervals is accepted,
but with a warning issued.  This allows older files to be
``repaired'', i.e., objects moved to a one micron grid.  This is
recommended for files that require it.  A sub-micron snapping interval
should not be used otherwise, and will not be saved in the technology
file produced with the {\cb Save Tech} button in the {\cb Attributes
Menu}.
\end{description}

The {\cb Edge Snapping} group appears below the grid snapping
controls.  This is different from grid snapping.  The edge snapping
will snap the cursor to the edge of a nearby object, which may or may
not be off grid (``off grid'' means that the coordinate is not on a
multiple of the snap grid interval).  The edge snapping is used in,
and the controls apply to, physical mode only.  In electrical mode,
the cursor will always snap to and indicate when near a connection
point.

When snapped in this manner to an edge, a small dotted box transiently
appears around the mouse pointer.  If snapped to vertex of an object,
the box will have a double outline.

Snapping will apply to visible objects at any level of the cell
hierarchy.  The edges of unexpanded subcells will also be snapped to.

The group consists of an {\cb Edge Snapping} menu, and four check
boxes, as described below.  These may be set independently in the main
window and sub-windows.  When a new sub-window appears, it will
inherit the edge snapping settings from the main window, but these can
then be changed in the sub-window if desired.

Unlike other controls in the {\cb Grid Setup} panel, these controls
operate immediately.  They do not require pressing the {\cb Apply}
button.

\begin{description}
\item{\cb Edge Snapping Menu}\\
The menu has three choices, to set the scope of the edge snapping
mode.  If {\cb DISABLED} is selected, then there is no edge snapping
in the window.  The default entry, {\cb Enabled in some commands},
enables the edge snapping in commands where it may be useful.  These
include the side menu commands with the following keywords:

\begin{quote} \vt
arc\\
box\\
break\\
donut\\
erase\\
polyg\\
round\\
wire\\
xor
\end{quote}

as well as the {\cb Rulers} command in the {\cb View Menu}.

The third choice is {\cb Enabled always}, which provides the edge
snapping at all times, in commands or outside of any command.

\item{\cb Allow off-grid edge snapping}\\
When this check box is checked, the cursor will snap to nearby edges,
whether or not they are on grid.  When not checked, only snapping to
on-grid locations is done.  This is unchecked by default.

\item{\cb Include non-Manhattan edges}\\
When checked, non-Manhattan (meaning not horizontal or vertical) edges
will be snapped to.  The snap points are the intersection of the snap
grid and the edge, plus the endpoints (vertices).  If not checked,
non-Manhattan edges ar not snapped to.  This is unchecked by default.

\item{\cb Include wire edges}\\
If checked, the edges of wires are considered for edge snapping.  If
not checked, wire edges are ignored.  This mode is enabled by default.

\item{\cb Include wire path}\\
A wire definition consists of a sequence of vertices, with an implied
line path connecting them.  This path would appear as a line running
along the center of a drawn wire.  When this box is checked, the wire
path is considered for ``edge'' snapping.  By default, this box is
unchecked, so that the wire path is not snapped to.
\end{description}

The {\cb Style} page contains controls which alter the presentation of
the grid visually.

\begin{description}
\item{\cb Show}\\
If the {\cb Show} button is active, the grid will be visible when
gradations are adequately large.  Otherwise, the grid will not be
visible in the window.

\item{\cb On Top}\\
If the {\cb On Top} button is active, the grid will be drawn last,
after all geometry.  Otherwise, it will be drawn first, in which case
it is more likely to be obscured by the geometry.

\item{\cb Store} and {\cb Recall}\\
The {\cb Store} and {\cb Recall} menus allow a set of grid parameters
to be saved in an internal register, to be recalled as needed.  There
are separate physical and electrical grid registers.  One will
automatically save to and recall from the register associated with the
window display mode.

The grid registers save all per-window parameters that can be set from
the {\cb Grid Setup} panel.  There are seven registers available, as
indicated on the menu produced by the {\cb Store} button.  The {\cb
Recall} button produces a menu with two additional entries:

\begin{description}
\item{\cb revert}\\
Revert the {\cb Grid Setup} panel to the current window settings.

\item{\cb last appl}\\
Recall the last settings that were applied with the {\cb Apply}
button.
\end{description}

Grid registers are saved to the technology file when a technology file
is written, and are loaded when a technology file is read.

See also the description of the {\cb !rg} and {\cb !sg} commands. 
These can be used to save and restore the grid from registers.

\item{\cb Axes} group\\
Below the {\cb Show} button, a radio button group is provided to set
the presentation style of the axes in physical mode.  The choices are
{\cb No Axes}, {\cb Plain Axes}, and {\cb Mark Origin}.  The {\cb Mark
Origin} choice is the default.  The {\cb Plain Axes} choice does away
with the small box at the origin, showing the axes as simple lines. 
The {\cb No Axes} choice suppresses the axes entirely.  In electrical
mode, the axes are always suppressed.

\item{\cb Coarse Mult}\\
To the right of the radio group is an integer entry for the coarse
multiple.  This is the number of fine grid lines or dots per coarse
grid line or dot.  Acceptable values are 1 through 50.  When set to
one, the coarse grid replaces the fine grid, which is shown with the
coarse grid color.  The default value is 5.

\item{\cb Line Style Editor}\\
This group sets the line or dot style used to render the grid.  There
are three ``radio buttons'', {\cb Solid}, {\cb Dots}, and {\cb
Textured} that set the basic grid style.  Choosing {\cb Solid} will
cause the grid to use continuous lines.  The {\cb Dots} option will
use a grid consisting of a small dot or cross at each grid point. 
When this selection is active, a {\cb Cross Size} entry area appears. 
This can be set to values 0--6, indicating the number of pixels to
light up around the central dot in the four compass directions.  If
zero, only the central pixel is lit, which can be difficult to see on
high-resolution displays.  The value 1 generally looks like a much
brighter dot.  Larger values will appear as a small cross.

If {\cb Textured} is chosen, a user-specified patterned line will be
used, and the line style editing areas become visible.  The line style
editor allows the user to specify the patterning of the lines used to
form the grid.  The upper window is a sample of the current line
style.  The lower window allows the user to set the line style by
clicking.

The line pattern starts at the left set bit (blue area) and extends to
the right of the display.  The pattern is used to ``tile'' the line. 
The left part of the display is shown in gray to indicate that it is
not part of the line style mask.  Clicking in this window with button
1 will toggle the bit.  Button 2 will clear the bit, and button 3 will
set the bit.  Multiple bits can be set or toggled by dragging.  The
line in the preview window will reflect changes in the pattern.

The sample window is a drag source for a piece of text giving the line
style mask in {\vt 0x}{\it hhhh} (hexadecimal) notation.  The mask is
the integer being represented by the lower window, with set bits in
blue.  This may be useful for creating line styles for entry
elsewhere.

\item{\cb All Windows}\\
This control group appears only in the {\cb Grid Setup} panel for the
main drawing window.  The two controls in the group are different from
other controls in the panel in that they apply globally to all drawing
windows.  They also differ from other controls in that they operate
immediately without the need to press the {\cb Apply} button, however
an explicit screen redraw is necessary to see the effect.

The {\cb No coarse when fine invisible} check box applies in physical
mode only.  When the check box is not checked, as one zooms out, when
the fine grid becomes too closely spaced it won't be shown, however
the coarse grid will be shown, unless it too is too finely spaced.  If
the check box is checked, the coarse grid will not be shown by itself,
it will be suppressed when the fine grid is suppressed.  This tracks
the state (set or unset) of the {\et GridNoCoarseOnly} variable.


The {\cb Visibility Threshold} entry sets the minimum number of pixels
between grid lines or dots.  The grid will be suppressed if it would
be smaller.  This applies to all drawing windows, both physical and
electrical.  This tracks the setting of the {\et GridThreshold}
variable.
\end{description}

Pressing the {\cb Apply} button will actually save the new grid
parameters in {\Xic}, and redraw the window if something has changed. 
Changes in the {\cb Edge Snapping} and {\cb All Windows} groups do not
need the {\cb Apply} button, changes take effect immediately.  All
other controls require an {\cb Apply} button press to assert the
change, and changes will {\bf not} be saved unless {\cb Apply} is
pressed.

