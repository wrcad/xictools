% -----------------------------------------------------------------------------
% MRouter Manual file: mrman2.tex
% Stephen R. Whiteley, Whiteley Research Inc., Sunnyvale CA
% No copyright, open-source
% $Id: mrman2.tex,v 1.5 2017/02/17 23:23:01 stevew Exp $
% -----------------------------------------------------------------------------

\chapter{Command Interface}

The {\MRouter} has a built-in set of command handlers that implement a
basic command-line interface.  Scripts can be produced by writing the
commands in a file, one command per line.  The scripts can be used for
router setup, and initiation and control of routing jobs.  Script files
can read other script files to arbitrary depth.

The command processing is implemented hierarchically.  At lowest level
are the database commands, which are impolemented in the database class.
Command processing for the router class will pass through unresolved
operations to the database command processor.  Finally, when using
the {\MRouter} plug-in, an application's command processing can
pass through unresolved operations to the {\MRouter} command processing.
This allows the application to provide its own commands, or modified
versions of the {\MRouter} commands.  For example, the {\Xic} program
adds commands to display the physical layout, and to move technology
information to and from the {\MRouter}.

The {\it Qrouter} has built-in TCL/TK support, and a comparable
command interface.  This support is not directly available in the
{\MRouter} presently.  However, the command interface provided can be
wrapped into TCL/TK, Python, or other script language functions, per
the user's preference.  Direct support for common languages such as
TCL/TK and Python may be added in future releases.

\section{The database commands}

These are the commands handled by the database, and are available
from an instance of the database class.  They are also available from
the router class, as a fallback.

\subsection{Database commands: {\vt reset}}
\index{reset command}

{\bf Syntax:} {\vt reset}

This command takes no further arguments.  It clears the database, and
resets database parameter settings to default values.

\subsection{Database commands: {\vt set}}
\index{set command}

{\bf Syntax:} {\vt set} [{\it paramname} [{\it value\/}]]

The {\vt set} command is used to set or query values for parameters used by the
database and router.  If no further arguments are given, a list of the parameters
and their present values is printed.  If a parameter name is given but
no value, that parameter name and its current value (or its default) will be printed. 
Otherwise, the parameter whose name is given will be set to the given
value.

The known parameters are as follows.
\begin{description}

\item{\vt debug}
The {\it value} is a hex number whose bits are flags that turn on
various debugging options.  Presently undocumented.

\item{\vt verbose}\\
This sets or displays the verbosity level for messages emitted by the
database and router.  If no {\it value} is given, the present
verbosity level is printed.  This is an integer in the range 0--4,
where 4 is the most verbose.  The {\it value} should otherwise be an
integer in this range.  The default verbosity is 0.

\item{\vt global}\\
Up to six ``global'' nets can be specified.  These are usually power
or ground nets, to be treated specially by the router.  Values are
space-separated net names, which when specified are added to the
internal list of global nets incrementally.  That is, multiple calls
can be made to define the global nets, or equivalently all of the
global nets can be listed for a single call.  The internal list can
be cleared with {\vt set clear global}.  If no net names are given,
the current global net name list is printed.

\item{\vt vdd}\\
This is for {\it Qrouter} compatibility, for setting the name of the
power net.  This is actually equivalent to {\vt global}.

\item{\vt gnd}\\
This is for {\it Qrouter} compatibility, for setting the name of the
ground net.  This is actually equivalent to {\vt global}.

\item{layers}\\
The {\it value} is an integer that represents the number o layers
that the router will use.  This can be set to a number less than or
equal to the number of layers that have been defined (from LEF or
technology input).  For example, if given a value 4, only the first 4
routing layers defined will be used by the router.  If no number is
given, the number of layers that the router would use is printed.

\item{\vt maxnets}\\
The {\it value} is the maximum number of nets allowed in a design. 
This can be set to limit the number of nets accepted for routing by
the router.  Jobs with too many nets will be rejected.  If no number
is given, the current maximum is printed.

\item{\vt lefresol}\\
The {\it value} is the LEF resolution, as is normally provided in a
LEF file.  Acceptable values (from the LEF specification) are 100,
200, 400, 800, 1000, 2000, 4000, 8000, 10000, 20000.  The internal
coordinates use this many points per micron.  The resolution is 100
if not otherwise given.  If not given, the current value is printed. 
Otherwise, the resolution may be set to the value given, however once
the LEF resolution is set, whether by LEF file or this keyword, it
can not be changed except by resetting the database.

\item{\vt mfggrid}\\
The {\it value} is the manufacturing grid usually obtained from a LEF
file.  It is provided as a real number in microns.  The value must be
exactly representable in the LEF resolution, and will quantize
coordinate values.  The {\it value} can be 0, in which case no
manufacturing grid will be used, which is the default.  If no {\it
value} is given, the present manufacturing grid, in microns, is
printed.

\item{\vt definresol}\\
The {\it value} is the resolution as normally read from a DEF file. 
Acceptable values are as listed for {\vt lefresol}, and furthermore
the value must exactly divide the {\vt lefresol}.  The resolution is
100 if not otherwise given.  If not given, the current value is
printed.  Otherwise, the resolution may be set to the value given,
however once the DEF input resolution is set, whether by DEF file or
this keyword, it can not be changed except by resetting the database.

\item{\vt defoutresol}\\
The {\it value} is the resolution used when writing DEF files with
{\vt write def}, which defaults to the {\vt definresol}.  Setting to
0 clears the overriding.  If no {\it value}is given, the present DEF
output resolution override is printed.

\item{\vt netorder}\\
The {\it value} can be set to one of the following integer values. 
It controls the order in which nets are routed.
\begin{description}
]item{0}\\
Use the default ordering, where nets with the most taps are routed
first.
\item{1}\\
Use alternate ordering, where the nets with the largest minimum box
containing taps are routed first.
\item{2}\\
Retain the ordering as nets were defined.
\end{description}
If no {\it value} is given, the present ordering enumerator is
printed.

\item{\vt passes}\\
The {\it value} is an integer that is the maximum number of attempted
passes of the route search algorithm, where the routing constraints
(maximum cost and route area mask) are relaxed on each pass.  The
default number of passes is 10.  With no {\it value}, the maximum
number of passes is printed.  Otherwise, a positive integer is
expected for the {\it value}.

\item{\vt increments}\\
The {\it value} is a list of positive integers, each giving the
incremental halo width in the routing mask for the corresponding
routing pass.  The mask is used to limit the area where a route can
be implemented.  On each pass, this area is expanded by the
corresponding {\vt increments} value.  The final increments value is
used for any subsequent passes.  The default is effectively the
single integer ``1'', so that the masking area is bloated by one
channel on each pass.  If no integers are given, the internal list of
increments is printed.

\item{\vt via\_stack}\\
The {\it value} sets the maximum number of vias that may be stacked
directly on top of each other at a single point.  A {\it value} of
``{\vt none}'' or ``{\vt 0}'' or ``{\vt 1}'' will disallow stacked
vias.  A {\it value} of ``{\vt all}'' will allow all vias to be
stacked (which is the default).  Otherwise, the {\it value} should be
an integer less than or equal to the number of via layers.

\item{\vt via\_pattern}\\
If vias are not square, then they are placed in a checkerboard
pattern, with every other via rotated 90 degrees.  If inverted, the
rotation is swapped relative to the grid positions used in the
non-inverted case.  If the {\it value} is 0 or a word starting with
``{\vt n}'', the pattern is not inverted.  If the {\it value} is
nonzero or a word starting with ``{\vt i}'', the pattern is inverted.
\end{description}

\subsection{Database commands: {\vt setcost}}
\index{setcost command}

{\bf Syntax:} {\vt setcost} [{\it paramname} [{\it value\/}]]

This is similar to the {\vt set} command, but is exclusively for the
cost parameters used to optimize routing.  Advanced users and
experimenters may appreciate the ability to access these parameters. 
The cost parameters are the following.  For most of these, only the
first character is checked.  If the first character is `{\vt c}', the
second character is also checked.

\begin{description}
\item{\vt segcost}\\
Add description.
The default is 1.

\item{\vt viacost}\\
Add description.
The default is 5.

\item{\vt jogcost}\\
Add description.
The default is 10.

\item{\vt xvercost} or {\vt crossovercost}\\
Add description.
The default is 4.

\item{\vt blockcost}\\
Add description.
The default is 25.

\item{\vt offsetcost}\\
Add description.
The default is 50.

\item{\vt conflictcost}\\
Add description.
The default is 50.
\end{description}

\subsection{Database commands: {\vt unset}}
\index{unset command}

{\bf Syntax:} {\vt unset} {\it keyword} [{\it keyword} ...]

The arguments are the keywords that are accepted by the {\bf set} or
{\bf setcost} commands.  At least one must be given, but any number
can appear in the argument list.  For each keyword, the default
values are set, undoing any {\bf set} operation.


\subsection{Database commands: {\vt ignore}}
\index{ignore command}

{\bf Syntax:} {\vt ignore} [{\it netname} ...] [{\vt -u} [{\vt all}]]
 [{\it netname} ...]

The database contains a list of net names that will be ignored by the
router.  The list is maintained by this command.

With no options, The entire list or names of ignored nets is printed. 
Otherwise, the arguments consist of net names or the special token
{\vt -u}.  Net names given before or without {\vt -u} will be added to
the ignored net list (if not already present).  Net names given after
{\vt -u} will be removed from the ignore net list, if found.

A special case is when the special keyword ``{\vt all}'' follows {\vt
-u}.  In this case, the entire list is cleared.  This is not undoable
so proceed with caution.

\subsection{Database commands: {\vt critical}}
\index{critical command}

{\bf Syntax:} {\vt critical} [{\it netname} ...] [{\vt -u} [{\vt all}]]
 [{\it netname} ...]

The database contains a list of net names that will be considered
critical by the router.  Such nets will be afforded higher priority by
the router.  The list is maintained by this command.  With no options,
The entire list or names of critical nets is printed.

With no options, The entire list or names of ignored nets is printed. 
Otherwise, the arguments consist of net names or the special token
{\vt -u}.  Net names given before or without {\vt -u} will be added to
the critical net list (if not already present).  Net names given after
{\vt -u} will be removed from the critical net list, if found.

A special case is when the special keyword ``{\vt all}'' follows {\vt
-u}.  In this case, the entire list is cleared.  This is not undoable
so proceed with caution.

\subsection{Database commands: {\vt obstruction}}
\index{obstruction command}

{\bf Syntax:} {\vt obstruction} [{\vt -u}] [{\it layer}] [{\vt all}]
   [{\it L} {\it B} {\it R} {\it T\/}]

The database contains a list of rectangular areas for each routing
layer that are given to the router as obstructions.  These are
user-defined areas where routes are not allowed to intersect.  The
obstruction list is maintained by this command.

If no arguments appear, the current list of user-defined obstructions
is printed.  If the {\vt -u} option is given, the command will
potentially remove obstructions from the list, otherwise obstructions
will be added (if an identical obstruction is not already present). 
The {\it layer} can be either the name of a routing layer, or its
1-based index number.

The special keyword {\vt all} can appear only when {\vt -u} is given.
When it appears, if a {\it layer} was given, all obstructions on that
layer will be removed from the list.  If no {\it layer} is given,
then the entire obstruction list is cleared.

Otherwise, if a {\it layer} is given, but the coordinates are omitted,
a list of the obstructions on the given layer is printed.  If the box
coordinates are included (given in microns), the indicated rectangle
is added to the list if {\vt -u} was not given, or removed from the
list (if it exists) if {\vt -u} was given.

\subsection{Database commands: {\vt layer}}
\index{layer command}

{\bf Syntax:} {\vt layer} [{\it layer} [{\it option\/}] [{\it args\/}]]

This command sets or queries some basic parameters of routing layers. 
If no arguments are given, a listing of the routing layers with their
present parameter values is printed.

The first argument is a layer name or 1-based routing layer index.  If
only this is given, the parameters for the given layer are printed.

The {\it option} that may follow indicates the parameter.  It can be
one of the following.

\begin{description}
\item{\vt -n}\\
This indicates the layer name.  If a word follows, the layer will be
given this name.  If no word follows, the present layer name will be
printed.

\item{\vt -l}\\
This indicates the user layer number.  This may indicate the GDSII
layer number.  It is a user parameter and not used by the router.  It
must be a non-negative integer if given.  If a number follows, it will
be assigned, otherwise the present value is printed.

\item{\vt -t}\\
This indicates the user type number.  This may indicate the GDSII
datatype number.  It is a user parameter and not used by the router. 
It must be a non-negative integer if given.  If a number follows, it
will be assigned, otherwise the present value is printed.

\item{\vt -w}\\
This indicates the default wire width used for routing, in microns. 
It must be a positive number.  If a number follows, it will be
assigned, otherwise the current value is printed.

\item{\vt -p}\\
This indicates the pitch of the routing, meaning the wire width plus
minimum spacing between wires.  This must be a positive value larger
than the wire width.  If a number follows, it will be assigned,
otherwise the current value is printed.

\item{\vt -d}\\
This indicates the preferred route direction, which can be horizontal
of vertical.  If a word follows, the first character of which is `{\vt
h}' or `{\vt v}', the direction is set.  Otherwise, the current
direction is printed.
\end{description}

\subsection{Database commands: {\vt newlayer}}
\index{newlayer command}

{\bf Syntax:} {\vt !mr newlayer} {\it name}

This creates a new layer with the given name and appends it to the
layers list.  The {\bf layer} command can then be used to fill in the
routing parameters.

\subsection{Database commands: {\vt boundary}}
\index{boundary command}

{\bf Syntax:} {\vt boundary} [{\it L} {\it B} {\it R} {\it T\/}]

This sets or prints the bounding box of the routed area in microns. 
This is ordinarily set from the cell and technology data.

\subsection{Database commands: {\vt read}}
\index{read command}

{\bf Syntax:} {\vt read} {\it source} [{\it filename\/}]

This command will read a file into the database.  The {\it source}
keyord must be one of the following.
\begin{description}
\item{\vt config}\\
This will read a configuration file in the format used by the {\it
Qrouter} from which {\MRouter} was derived.  If no {\it filename} is
given, the name ``{\vt route.config}'' is assumed.  If given, the {\it
filename} should be a path to a Qrouter-style config file.  This
functionality is deprecated and may be removed at some point.

\item{\vt script}\\
A script is a text file containing commands, as described in this
section, one per line.  The {\it filename} should be a path to such a
file.  The file will be read, and the commands executed in order,
unless a fatal error occurs, which terminates execution.  Calls to
read scripts can recurse to any depth.

\item{\vt lef}\\
The {\it filename} should be a path to a file in the Cadence LEF
format (Layout Exchange Format).  This is an industry-standard format
used to provide routing information about the standard cells to be
used by the router.  This includes technology information about the
routing layers (e.g., routing direction, pitch, width), and/or
definitions of the standard cells including size information and
terminal locations.  It may also provide definitions for the standard
vias defined for the process.

\item{\vt def}\\
The {\it filename} should be a path to a file in the Cadence DEF
format (Design Exchange Format).  This is an industry-standard format
used to pass netlist information.  This will include placement
locations and orientations of the cell instances used in the design,
plus a list of nets which define the circuit.  The DEF file provides
the nets in abstract form, the router implements these nets in
physical form.
\end{description}

\subsection{Database commands: {\vt write}}
\index{write command}

{\bf Syntax:} {\vt write} {\it target} [{\it filename\/}]

The {\vt write} command will create a file.  The {\it target} can be
one of the following.
\begin{description}
\item{\vt lef}\\
The {\it filename} is the name of a LEF file to write.  The file will
contain the technology and standard-cell information found in the
database.

\item{\vt def}\\
The {\it filename} is the name of a DEF file to write.  The file will
contain the components, pins, and nets as saved in the database.
\end{description}

\subsection{Database commands: {\vt append}}
\index{append command}

{\bf Syntax:} {\vt append} {\it deffile1} {\it deffile2}

This will copy the DEF file {\it deffile1}, while adding the physical
routing information, to {\it deffile2}.  This is pretty much the same
as the {\it Qrouter} output method.  The {\it deffile1} must be the
original input DEF file, or contain the same instance placements and
abstract routes.  When the router is run, the physical routes are
written back to the database, where they can be used to update the
original source as described.


\section{The router commands}

When called through the router object, the list of available commands
is a superset of the database commands listed in the previous section.
The {\vt read script} command is overloaded to handle scripts containing
router commands.

\subsection{Router commands: {\vt reset}}
\index{reset command}

{\bf Syntax:} {\vt reset} [{\vt all}]

This overrides the database command of the same name.  If the ``{\vt
all}'' argument is given, the router and database are both reset to a
pristine state.  If {\vt all} is not given, only the router is reset,
database information is retained.  In this context, {\vt true}, {\vt
yes}, and {\vt 1} (one) are synonyms for {\vt all}, case insensitive,
and only the leading character is actually read.


\subsection{Router commands: {\vt stage1}}
\index{stage1 command}

{\bf Syntax:} {\vt stage1} [{\it options\/}]

Execute stage1 routing.  This works through the entire netlist,
routing as much as possible but not doing any rip-up and re-route. 
Nets that fail to route are put in the failed nets list.  The number
of failed routes is printed.

The following options are understood.
\begin{description}
\item{\vt -d}\\
Draw the area being searched in real-time.  This slows down the
algorithm and is intended only for diagnostic use.  {\bf Not
implemented.}

\item{\vt -s}\\
Single-step stage one, i.e., do one route per call, incrementally.

\item{\vt -m}[ ]{\it value}\\
The {\it value} follows the option character with or without
intervening white space.  It can be an integer 0 or larger, and will
set the search mask size.  It can also be set to one of the following
keywords, where only the first character is significant.
\begin{description}
\item{\vt n}[{\vt one}]\\
Don't limit the search area.
\item{\vt a}[{\vt uto}]\\
Select the mask automatically.
\item{\vt b}[{\vt box}]\\
Use the net bounding box as a mask.
\end{description}

\item{\vt -f}\\
Force a terminal to be routable.

\item{\it netname}\\
Route the named net only.
\end{description}

\subsection{Router commands: {\vt stage2}}
\index{stage2 command}

{\bf Syntax:} {\vt stage2} [{\it options\/}]

Execute stage2 routing.  This stage works through the failed nets
list, routing with collisions, and then ripping up the colliding nets
and appending them to the failed nets list.

The number of failed nets remaining is printed on exit.

The following options are understood.
\begin{description}
\item{\vt -d}\\
Draw the area being searched in real-time.  This slows down the
algorithm and is intended only for diagnostic use.  {\bf Not
implemented.}

\item{\vt -s}\\
Single-step stage two, i.e., do one route per call, incrementally.

\item{\vt -m}[ ]{\it value}\\
The {\it value} follows the option character with or without
intervening white space.  It can be an integer 0 or larger, and will
set the search mask size.  It can also be set to one of the following
keywords, where only the first character is significant.
\begin{description}
\item{\vt n}[{\vt one}]\\
Don't limit the search area.
\item{\vt a}[{\vt uto}]\\
Select the mask automatically.
\item{\vt b}[{\vt box}]\\
Use the net bounding box as a mask.
\end{description}

\item{\vt -f}\\
Force a terminal to be routable.

\item{\vt -l} {\it N}\\
Fail route if the solution collides with more than {\it N} nets.

\item{\vt -t} {\it N}\\
Keep trying {\it N} additional times.

\item{\it netname}\\
Route the named net only.
\end{description}

\subsection{Router commands: {\vt stage3}}
\index{stage3 command}

{\bf Syntax:} {\vt stage3} [{\it options\/}]

Execute stage3 routing.  This works through the entire netlist,
ripping up each route in turn and re-routing it.  The number of failed
nets remaining is printed on exit.

The following options are understood.
\begin{description}
\item{\vt -d}\\
Draw the area being searched in real-time.  This slows down the
algorithm and is intended only for diagnostic use.  {\bf Not
implemented.}

\item{\vt -s}\\
Single-step stage three, i.e., do one route per call, incrementally.

\item{\vt -m}[ ]{\it value}\\
The {\it value} follows the option character with or without
intervening white space.  It can be an integer 0 or larger, and will
set the search mask size.  It can also be set to one of the following
keywords, where only the first character is significant.
\begin{description}
\item{\vt n}[{\vt one}]\\
Don't limit the search area.
\item{\vt a}[{\vt uto}]\\
Select the mask automatically.
\item{\vt b}[{\vt box}]\\
Use the net bounding box as a mask.
\end{description}

\item{\vt -f}\\
Force a terminal to be routable.

\item{\vt -l} {\it N}\\
Fail route if the solution collides with more than {\it N} nets.

\item{\vt -t} {\it N}\\
Keep trying {\it N} additional times.

\item{\it netname}\\
Route the named net only.
\end{description}

\subsection{Router commands: {\vt ripup}}
\index{ripup command}

{\bf Syntax:} {\vt ripup} [{\vt -a} {\vt |} {\it netmname} ...]

Rip up (remove all physical routing information from) a net or nets,
or all nets in the design.  If {\vt -a} is specified, all nets
will be ripped up.  Otherwise, the arguments are names of nets to
rip up.

\subsection{Router commands: {\vt failed}}
\index{failed command}

{\bf Syntax:} {\vt failed} [{\it option\/}]

The default operation, i.e., with no argument given, is to print the
number of nets currently in the failed nets list.  Otherwise, one of
the following options may be given.
\begin{description}
\item{\vt -a}\\
Move all nets to the failed net list, ordered by the standard metric.
\item{\vt -u}\\
Move all nets to the failed nets list, as originally ordered.
\item{\vt -p}\\
Print a list of failed net names.
\end{description}

\subsection{Router commands: {\vt congest}}
\index{congest command}

{\bf Syntax:} {\vt congest} [{\vt -n} {\it N\/}] [{\it filename\/}]

This command will compute and print a congestion value for each cell
instance.  The values are print in sorted order, highest congestion
to lowest.  By default, all instances are listed, if the {\vt -n}
option is given, only the highest {\it N} instances are listed ({\it
N} being a positive integer).  If a {\it filename} is given, output
goes to that file, otherwise output goes to the standard output. 
This command is intended to identify highly congested parts of a
layout, which may benefit from additional fill in a subsequent
routing run.

